% Appendix A: Key Code
% 附录A:核心代码

\section*{Appendix A: Key Code}
\addcontentsline{toc}{section}{Appendix A: Key Code}
\label{appendix:code}

This appendix presents the core implementation of our sustainability assessment framework. Full code and data are available in our supplementary materials.

%=== A.1 数据预处理与指标计算 ===
\subsection*{A.1 Data Preprocessing and Indicator Calculation}

\begin{lstlisting}[language=Python, caption={Sustainability indicator preprocessing and normalization}]
import pandas as pd
import numpy as np
from scipy.stats import entropy

def load_sustainability_data(filepath):
    """Load and preprocess sustainability indicator data."""
    df = pd.read_csv(filepath)
    
    # Handle missing values via interpolation
    df = df.interpolate(method='linear', limit_direction='both')
    
    return df

def normalize_indicators(df, indicator_cols, directions):
    """
    Normalize indicators using min-max scaling.
    directions: dict mapping indicator -> 'positive' or 'negative'
    """
    normalized = pd.DataFrame()
    
    for col in indicator_cols:
        x = df[col].values
        x_min, x_max = x.min(), x.max()
        
        if directions[col] == 'positive':
            normalized[col] = (x - x_min) / (x_max - x_min + 1e-10)
        else:  # negative indicator (lower is better)
            normalized[col] = (x_max - x) / (x_max - x_min + 1e-10)
    
    return normalized

def entropy_weight(normalized_df):
    """Calculate weights using entropy method."""
    n, m = normalized_df.shape
    
    # Calculate proportion matrix
    P = normalized_df / normalized_df.sum(axis=0)
    P = P.replace(0, 1e-10)  # Avoid log(0)
    
    # Calculate entropy for each indicator
    E = -1 / np.log(n) * (P * np.log(P)).sum(axis=0)
    
    # Calculate weights
    weights = (1 - E) / (1 - E).sum()
    
    return weights.to_dict()
\end{lstlisting}

%=== A.2 AHP权重计算 ===
\subsection*{A.2 AHP Weight Calculation}

\begin{lstlisting}[language=Python, caption={Analytic Hierarchy Process implementation}]
import numpy as np

def ahp_weights(comparison_matrix):
    """
    Calculate weights from AHP pairwise comparison matrix.
    Returns weights and consistency ratio.
    """
    n = comparison_matrix.shape[0]
    
    # Calculate eigenvalues and eigenvectors
    eigenvalues, eigenvectors = np.linalg.eig(comparison_matrix)
    
    # Get principal eigenvalue and eigenvector
    max_idx = np.argmax(eigenvalues.real)
    lambda_max = eigenvalues[max_idx].real
    principal_eigenvector = eigenvectors[:, max_idx].real
    
    # Normalize to get weights
    weights = principal_eigenvector / principal_eigenvector.sum()
    
    # Calculate Consistency Index
    CI = (lambda_max - n) / (n - 1)
    
    # Random Index (for n up to 10)
    RI = {1: 0, 2: 0, 3: 0.58, 4: 0.90, 5: 1.12, 
          6: 1.24, 7: 1.32, 8: 1.41, 9: 1.45, 10: 1.49}
    
    # Consistency Ratio
    CR = CI / RI.get(n, 1.49)
    
    return weights, CR

def hybrid_weights(ahp_w, entropy_w, lambda_param=0.5):
    """Combine AHP and entropy weights."""
    hybrid = {}
    for key in ahp_w:
        hybrid[key] = lambda_param * ahp_w[key] + (1 - lambda_param) * entropy_w[key]
    return hybrid
\end{lstlisting}

%=== A.3 系统动力学模型 ===
\subsection*{A.3 System Dynamics Model}

\begin{lstlisting}[language=Python, caption={System Dynamics simulation core}]
import numpy as np
from scipy.integrate import odeint

class SustainabilitySDModel:
    """System Dynamics model for sustainability assessment."""
    
    def __init__(self, params):
        self.params = params
    
    def derivatives(self, y, t, scenario):
        """
        Calculate derivatives for stock variables.
        y = [E_stock, K, W] (environmental, economic, social)
        """
        E, K, W = y
        p = self.params
        s = scenario
        
        # Environmental subsystem
        R_regen = p['regen_rate'] * E * (1 - E / p['carrying_capacity'])
        D_depletion = p['depletion_rate'] * K
        P_pollution = s['pollution_rate'] * K
        I_restore = s['restoration_investment']
        
        dE_dt = R_regen - D_depletion - P_pollution + I_restore
        
        # Economic subsystem (extended Cobb-Douglas)
        Y = p['tfp'] * (K ** p['alpha']) * (E ** p['gamma'])
        I = p['savings_rate'] * Y
        depreciation = p['depreciation'] * K
        
        dK_dt = I - depreciation
        
        # Social subsystem
        income_effect = p['beta1'] * np.log(max(Y, 1))
        env_effect = p['beta2'] * E
        inequality_penalty = p['beta3'] * (1 - E / p['carrying_capacity'])
        
        dW_dt = income_effect + env_effect - inequality_penalty - p['decay'] * W
        
        return [dE_dt, dK_dt, dW_dt]
    
    def simulate(self, y0, t_span, scenario):
        """Run simulation over time span."""
        t = np.linspace(t_span[0], t_span[1], int(t_span[1] - t_span[0]) + 1)
        solution = odeint(self.derivatives, y0, t, args=(scenario,))
        return t, solution

# Example scenario definitions
SCENARIOS = {
    'BAU': {'pollution_rate': 0.02, 'restoration_investment': 1.0},
    'MOD': {'pollution_rate': 0.015, 'restoration_investment': 2.0},
    'AGG': {'pollution_rate': 0.008, 'restoration_investment': 4.0},
    'STRESS': {'pollution_rate': 0.02, 'restoration_investment': 1.0}
}
\end{lstlisting}

%=== A.4 TOPSIS实现 ===
\subsection*{A.4 TOPSIS Implementation}

\begin{lstlisting}[language=Python, caption={TOPSIS multi-criteria decision analysis}]
import numpy as np

def topsis(decision_matrix, weights, criteria_types):
    """
    TOPSIS method for multi-criteria decision analysis.
    
    Parameters:
    - decision_matrix: numpy array (alternatives x criteria)
    - weights: array of criterion weights
    - criteria_types: list of 'benefit' or 'cost' for each criterion
    
    Returns:
    - closeness: relative closeness to ideal solution
    - ranking: ranked alternatives (1 = best)
    """
    # Step 1: Normalize decision matrix
    norm_matrix = decision_matrix / np.sqrt((decision_matrix ** 2).sum(axis=0))
    
    # Step 2: Apply weights
    weighted_matrix = norm_matrix * weights
    
    # Step 3: Determine ideal and anti-ideal solutions
    ideal = np.zeros(len(weights))
    anti_ideal = np.zeros(len(weights))
    
    for j, ctype in enumerate(criteria_types):
        if ctype == 'benefit':
            ideal[j] = weighted_matrix[:, j].max()
            anti_ideal[j] = weighted_matrix[:, j].min()
        else:  # cost criterion
            ideal[j] = weighted_matrix[:, j].min()
            anti_ideal[j] = weighted_matrix[:, j].max()
    
    # Step 4: Calculate separation measures
    d_plus = np.sqrt(((weighted_matrix - ideal) ** 2).sum(axis=1))
    d_minus = np.sqrt(((weighted_matrix - anti_ideal) ** 2).sum(axis=1))
    
    # Step 5: Calculate relative closeness
    closeness = d_minus / (d_plus + d_minus + 1e-10)
    
    # Ranking (higher closeness = better)
    ranking = closeness.argsort()[::-1].argsort() + 1
    
    return closeness, ranking
\end{lstlisting}
