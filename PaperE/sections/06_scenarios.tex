% Section 6: Scenario Design and Simulation
% 情景设计与模拟:E题核心分析框架

\section{Scenario Design and Simulation}
\label{sec:scenarios}

% 【E题核心】情景分析是O奖论文的必备组件

%=== 6.1 情景设计原则 ===
\subsection{Scenario Design Principles}
\label{subsec:scenario_principles}

Our scenario framework follows established best practices in sustainability assessment:

\begin{enumerate}[itemsep=0.3em]
    \item \textbf{Plausibility:} Each scenario is grounded in realistic policy options and technological trajectories.
    
    \item \textbf{Consistency:} Internal logic is maintained---policy choices align with socioeconomic assumptions.
    
    \item \textbf{Diversity:} Scenarios span the range of possible futures, from conservative to transformative.
    
    \item \textbf{Relevance:} Scenarios address the specific decision context of \TODO{target stakeholder/decision}.
    
    \item \textbf{Challenge:} At least one scenario stress-tests system resilience under extreme conditions.
\end{enumerate}

%=== 6.2 四大情景定义 ===
\subsection{Four Representative Scenarios}
\label{subsec:four_scenarios}

We design four scenarios that span the policy option space:

\begin{table}[H]
    \centering
    \caption{Summary of Scenario Specifications}
    \label{tab:scenarios}
    \begin{tabularx}{\textwidth}{l X X c}
        \toprule
        \textbf{Scenario} & \textbf{Policy Description} & \textbf{Key Parameters} & \textbf{Time Horizon} \\
        \midrule
        \textbf{S1: Business-as-Usual (BAU)} & 
        Continuation of current policies with no additional interventions. Baseline for comparison. & 
        \TODO{e.g., Current emission rates, existing regulations} & 
        \TODO{2025--2050} \\
        \addlinespace[0.5em]
        
        \textbf{S2: Moderate Intervention (MOD)} & 
        Incremental policy adjustments aligned with current pledges (e.g., NDCs, existing commitments). & 
        \TODO{e.g., 30\% emission reduction, moderate investment} & 
        \TODO{2025--2050} \\
        \addlinespace[0.5em]
        
        \textbf{S3: Aggressive Transition (AGG)} & 
        Transformative policy shift toward sustainability. Ambitious targets aligned with 1.5°C pathway or equivalent. & 
        \TODO{e.g., 70\% emission reduction, major investment} & 
        \TODO{2025--2050} \\
        \addlinespace[0.5em]
        
        \textbf{S4: Climate Shock (STRESS)} & 
        Stress test: Baseline policies plus an extreme climate event (e.g., drought, flood, pandemic) occurring at \TODO{year}. & 
        \TODO{e.g., 20\% productivity shock, emergency costs} & 
        \TODO{2025--2050} \\
        \bottomrule
    \end{tabularx}
\end{table}

%=== 6.3 情景参数化 ===
\subsection{Scenario Parameterization}
\label{subsec:scenario_params}

Each scenario modifies the System Dynamics model through specific parameter adjustments:

\subsubsection{Scenario S1: Business-as-Usual}
\begin{align}
    I_{\text{restoration}}(t) &= I_0 \quad \text{(constant at current level)} \label{eq:s1_invest} \\
    P_{\text{pollution}}(t) &= P_0 \cdot (1 + g_p)^t \quad \text{(growing at historical rate } g_p \text{)} \label{eq:s1_pollution}
\end{align}

\subsubsection{Scenario S2: Moderate Intervention}
% [TODO: Set r_mod and rho_mod values based on actual policy scenarios]
\begin{align}
    I_{\text{restoration}}(t) &= I_0 \cdot (1 + r_{\text{mod}})^t \quad \text{(moderate investment growth)} \label{eq:s2_invest} \\
    P_{\text{pollution}}(t) &= P_0 \cdot (1 - \rho_{\text{mod}})^t \quad \text{(gradual reduction)} \label{eq:s2_pollution}
\end{align}

\subsubsection{Scenario S3: Aggressive Transition}
% [TODO: Set r_agg and rho_agg values based on actual policy scenarios]
\begin{align}
    I_{\text{restoration}}(t) &= I_0 \cdot (1 + r_{\text{agg}})^t \quad \text{(ambitious investment growth)} \label{eq:s3_invest} \\
    P_{\text{pollution}}(t) &= P_0 \cdot \exp(-\rho_{\text{agg}} \cdot t) \quad \text{(exponential reduction)} \label{eq:s3_pollution}
\end{align}

\subsubsection{Scenario S4: Climate Shock}
% [TODO: Set sigma_shock and kappa_recovery values based on stress scenarios]
\begin{equation}
    Y(t) = Y_{\text{BAU}}(t) \cdot 
    \begin{cases}
        1 & \text{if } t < t_{\text{shock}} \\
        (1 - \sigma_{\text{shock}}) \cdot \exp(\kappa_{\text{recovery}} \cdot (t - t_{\text{shock}})) & \text{if } t \geq t_{\text{shock}}
    \end{cases}
    \label{eq:s4_shock}
\end{equation}

%=== 6.4 模拟结果 ===
\subsection{Simulation Results}
\label{subsec:simulation_results}

% 核心结果图表占位符
\begin{figure}[H]
    \centering
    \begin{subfigure}[t]{0.48\textwidth}
        \centering
        \fbox{\parbox[c][0.22\textheight][c]{\textwidth}{
            \centering
            \textbf{(a) Environmental Quality Trajectories}
            \par\vspace{0.5em}
            \small \TODO{Time-series plot showing $E_{\text{stock}}(t)$ for all four scenarios. Highlight tipping point if BAU crosses threshold.}
        }}
        \caption{Environmental stock under four scenarios}
        \label{fig:scenario_env}
    \end{subfigure}\hfill
    \begin{subfigure}[t]{0.48\textwidth}
        \centering
        \fbox{\parbox[c][0.22\textheight][c]{\textwidth}{
            \centering
            \textbf{(b) Economic Output Trajectories}
            \par\vspace{0.5em}
            \small \TODO{Time-series plot showing $Y(t)$ for all four scenarios. Show initial cost of AGG but long-term benefits.}
        }}
        \caption{Economic output under four scenarios}
        \label{fig:scenario_econ}
    \end{subfigure}
    
    \vspace{0.5cm}
    
    \begin{subfigure}[t]{0.48\textwidth}
        \centering
        \fbox{\parbox[c][0.22\textheight][c]{\textwidth}{
            \centering
            \textbf{(c) Social Well-being Trajectories}
            \par\vspace{0.5em}
            \small \TODO{Time-series plot showing $W(t)$ for all four scenarios.}
        }}
        \caption{Social well-being under four scenarios}
        \label{fig:scenario_social}
    \end{subfigure}\hfill
    \begin{subfigure}[t]{0.48\textwidth}
        \centering
        \fbox{\parbox[c][0.22\textheight][c]{\textwidth}{
            \centering
            \textbf{(d) Composite Sustainability Score}
            \par\vspace{0.5em}
            \small \TODO{Time-series plot showing $U(t)$ for all four scenarios.}
        }}
        \caption{Composite sustainability score}
        \label{fig:scenario_composite}
    \end{subfigure}
    \caption{System Dynamics simulation results across four scenarios (2025--2050).}
    \label{fig:scenario_results}
\end{figure}

%=== 6.5 关键发现 ===
\subsection{Key Findings}
\label{subsec:scenario_findings}

\TODO{Summarize the key insights from scenario comparison. Example structure:}

\begin{enumerate}[itemsep=0.4em]
    \item \textbf{BAU Unsustainability:} Under Business-as-Usual, environmental quality crosses the tipping point by \TODO{year}, triggering \TODO{consequence}.
    
    \item \textbf{Moderate vs. Aggressive Trade-offs:} MOD achieves \TODO{X\%} improvement in sustainability score but requires \TODO{Y\%} less investment than AGG. However, AGG avoids \TODO{specific risk}.
    
    \item \textbf{Shock Resilience:} The STRESS scenario reveals that systems under AGG recover \TODO{Z times faster} than under BAU, demonstrating the resilience value of sustainability investments.
    
    \item \textbf{Temporal Dynamics:} The ``green paradox'' is observed: AGG shows \TODO{short-term cost description} but yields net benefits by \TODO{breakeven year}.
\end{enumerate}
