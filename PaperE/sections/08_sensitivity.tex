% Section 8: Sensitivity and Robustness Analysis
% 灵敏度分析与鲁棒性检验

\section{Sensitivity and Robustness Analysis}
\label{sec:sensitivity}

% 【E题O奖要求】灵敏度分析必须针对指标权重和关键参数

%=== 8.1 灵敏度分析框架 ===
\subsection{Sensitivity Analysis Framework}
\label{subsec:sensitivity_framework}

Sensitivity analysis examines how variations in model inputs affect outputs, identifying which parameters most influence conclusions. We conduct three types of sensitivity analysis:

\begin{enumerate}[itemsep=0.3em]
    \item \textbf{Local Sensitivity (One-at-a-Time):} Perturb individual parameters while holding others constant.
    
    \item \textbf{Global Sensitivity (Monte Carlo):} Simultaneously vary multiple parameters according to probability distributions.
    
    \item \textbf{Structural Sensitivity:} Test alternative model formulations and assumptions.
\end{enumerate}

%=== 8.2 指标权重灵敏度 ===
\subsection{Indicator Weight Sensitivity}
\label{subsec:weight_sensitivity}

The AHP-Entropy hybrid weights contain inherent subjectivity. We examine sensitivity to the combination coefficient $\lambda$ and pillar-level weights.

\subsubsection{Sensitivity to $\lambda$ (AHP-Entropy Balance)}
\begin{figure}[H]
    \centering
    \fbox{\parbox[c][0.22\textheight][c]{0.85\textwidth}{
        \centering
        \textbf{TOPSIS Rankings as $\lambda$ Varies from 0 (Pure Entropy) to 1 (Pure AHP)}
        \par\vspace{0.5em}
        \small \TODO{Line plot showing $C_j^*$ for each scenario as $\lambda$ varies. Identify the range of $\lambda$ over which the ranking remains stable.}
    }}
    \caption{Sensitivity of scenario rankings to the AHP-Entropy balance parameter $\lambda$.}
    \label{fig:lambda_sensitivity}
\end{figure}

\TODO{Report findings: e.g., ``Rankings remain stable for $\lambda \in [0.3, 0.7]$, suggesting conclusions are robust to reasonable variations in weight methodology.''}

\subsubsection{Sensitivity to Pillar Weights}
We conduct a systematic sweep of pillar weights $(w_{\text{Env}}, w_{\text{Econ}}, w_{\text{Soc}})$ over the simplex and identify ``ranking reversal regions.''

\begin{figure}[H]
    \centering
    \fbox{\parbox[c][0.22\textheight][c]{0.85\textwidth}{
        \centering
        \textbf{Ternary Plot: Optimal Scenario by Pillar Weight Combination}
        \par\vspace{0.5em}
        \small \TODO{Ternary (triangular) diagram with corners representing 100\% weight on each pillar. Color-code regions by which scenario is optimal. Show that the recommended scenario is optimal over a large region.}
    }}
    \caption{Ternary plot showing which scenario is optimal under different pillar weight combinations.}
    \label{fig:ternary}
\end{figure}

%=== 8.3 系统动力学参数灵敏度 ===
\subsection{System Dynamics Parameter Sensitivity}
\label{subsec:sd_sensitivity}

We identify the parameters to which model outputs are most sensitive using normalized sensitivity elasticity:

\begin{equation}
    \eta_{\theta} = \frac{\partial U / U}{\partial \theta / \theta} \approx \frac{\Delta U / U}{\Delta \theta / \theta}
    \label{eq:elasticity}
\end{equation}

\begin{table}[H]
    \centering
    \caption{Sensitivity Elasticity for Key System Dynamics Parameters}
    \label{tab:sd_sensitivity}
    \begin{tabular}{l c c l}
        \toprule
        \textbf{Parameter} & \textbf{Baseline} & \textbf{Elasticity $\eta$} & \textbf{Interpretation} \\
        \midrule
        $R_{\text{regen}}$ (Regeneration rate) & \TODO{value} & \TODO{$\eta$} & \TODO{High/Moderate/Low sensitivity} \\
        $\gamma$ (Env. elasticity of output) & \TODO{value} & \TODO{$\eta$} & \TODO{interpretation} \\
        $\tau_{\text{delay}}$ (Policy delay) & \TODO{value} & \TODO{$\eta$} & \TODO{interpretation} \\
        $r$ (Discount rate) & \TODO{value} & \TODO{$\eta$} & \TODO{interpretation} \\
        $\kappa$ (Carrying capacity) & \TODO{value} & \TODO{$\eta$} & \TODO{interpretation} \\
        \bottomrule
    \end{tabular}
\end{table}

\TODO{Discuss which parameters dominate and implications for data collection priorities.}

%=== 8.4 不确定性下的鲁棒性 ===
\subsection{Robustness under Uncertainty}
\label{subsec:robustness}

We assess model robustness through Monte Carlo simulation with parameter uncertainty:

\subsubsection{Uncertainty Propagation}
\begin{enumerate}[itemsep=0.2em]
    \item Define probability distributions for uncertain parameters based on literature ranges.
    \item Draw $N = 1000$ samples via Latin Hypercube Sampling for efficiency.
    \item Run System Dynamics model and TOPSIS ranking for each sample.
    \item Analyze distribution of outcomes and ranking frequencies.
\end{enumerate}

\begin{figure}[H]
    \centering
    \begin{subfigure}[t]{0.48\textwidth}
        \centering
        \fbox{\parbox[c][0.20\textheight][c]{\textwidth}{
            \centering
            \textbf{(a) Distribution of Sustainability Scores}
            \par\vspace{0.5em}
            \small \TODO{Box plots or violin plots showing the distribution of $U$ for each scenario under parameter uncertainty.}
        }}
        \caption{Uncertainty in composite scores}
        \label{fig:uncertainty_scores}
    \end{subfigure}\hfill
    \begin{subfigure}[t]{0.48\textwidth}
        \centering
        \fbox{\parbox[c][0.20\textheight][c]{\textwidth}{
            \centering
            \textbf{(b) Ranking Frequency Distribution}
            \par\vspace{0.5em}
            \small \TODO{Stacked bar chart showing how often each scenario ranks 1st, 2nd, 3rd, 4th across Monte Carlo runs.}
        }}
        \caption{Robustness of scenario rankings}
        \label{fig:ranking_robustness}
    \end{subfigure}
    \caption{Robustness analysis under parametric uncertainty.}
    \label{fig:robustness}
\end{figure}

\subsubsection{Key Robustness Findings}

\TODO{Report findings, e.g.:}
\begin{itemize}[itemsep=0.2em]
    \item \TODO{``Scenario S3 (AGG) ranks first in X\% of Monte Carlo runs, demonstrating robust superiority under uncertainty.''}
    \item \TODO{``The 90\% confidence interval for sustainability score under S3 is [a, b], non-overlapping with S1 (BAU), confirming statistical significance.''}
    \item \TODO{``Ranking reversals between S2 and S3 occur primarily when [specific condition], suggesting [implication].''}
\end{itemize}

%=== 8.5 结构鲁棒性检验 ===
\subsection{Structural Robustness Checks}
\label{subsec:structural}

Beyond parametric uncertainty, we test robustness to modeling choices:

\begin{table}[H]
    \centering
    \caption{Structural Robustness Tests}
    \label{tab:structural_tests}
    \begin{tabularx}{\textwidth}{l X c}
        \toprule
        \textbf{Test} & \textbf{Modification} & \textbf{Ranking Change?} \\
        \midrule
        Alternative normalization & Use Z-score instead of min-max & \TODO{Yes/No} \\
        Alternative MCDA method & Use VIKOR instead of TOPSIS & \TODO{Yes/No} \\
        Different time horizon & Project to 2040 instead of 2050 & \TODO{Yes/No} \\
        Remove outlier indicators & Exclude most extreme indicator & \TODO{Yes/No} \\
        \bottomrule
    \end{tabularx}
\end{table}

\TODO{Summarize: ``Core conclusions are robust to structural variations, with rankings preserved in X of Y tests.''}
