% Section 11: Conclusion
% 结论

\section{Conclusion}
\label{sec:conclusion}

%=== 11.1 研究总结 ===
\subsection{Summary of Contributions}
\label{subsec:summary}

In this paper, we developed a comprehensive, sustainability-oriented decision framework to address \TODO{specific Problem E topic}. Our key contributions include:

\begin{enumerate}[itemsep=0.4em]
    
    \item \textbf{Three-Pillar Sustainability Indicator System}
    
    We constructed a hierarchical indicator framework spanning environmental, economic, and social dimensions, with weights determined through a hybrid AHP-Entropy method that balances expert judgment and data-driven objectivity.
    
    \item \textbf{System Dynamics Model of Complex Interactions}
    
    Our SD model captures the nonlinear feedback loops, time delays, and tipping points characteristic of sustainability challenges, enabling analysis of long-term system trajectories under different policy regimes.
    
    \item \textbf{Scenario-Based Analysis}
    
    Four carefully designed scenarios---Business-as-Usual, Moderate Intervention, Aggressive Transition, and Climate Shock---provide a comprehensive exploration of the policy option space and stress-test system resilience.
    
    \item \textbf{Multi-Criteria Decision Support}
    
    The TOPSIS method combined with trade-off frontier visualization provides transparent, defensible rankings while explicitly revealing the inherent tensions among sustainability pillars.
    
    \item \textbf{Actionable Policy Recommendations}
    
    We translate technical findings into tiered, stakeholder-specific recommendations with implementation roadmaps and adaptive management provisions.
    
\end{enumerate}

%=== 11.2 核心发现 ===
\subsection{Key Findings}
\label{subsec:key_findings}

Our analysis yields the following principal conclusions:

\begin{itemize}[itemsep=0.4em]
    
    \item \TODO{Finding 1: e.g., ``Business-as-Usual is unsustainable, with environmental quality crossing critical thresholds by [year].''}
    
    \item \TODO{Finding 2: e.g., ``The Aggressive Transition scenario (S3) achieves the highest composite sustainability score while maintaining long-term economic viability.''}
    
    \item \TODO{Finding 3: e.g., ``Short-term economic costs of aggressive action are outweighed by avoided damages and co-benefits by [year].''}
    
    \item \TODO{Finding 4: e.g., ``System resilience under climate shocks is significantly higher for sustainable pathways.''}
    
    \item \TODO{Finding 5: e.g., ``Key leverage points for intervention include [specific factors identified in SD analysis].''}
    
\end{itemize}

%=== 11.3 政策启示 ===
\subsection{Policy Implications}
\label{subsec:implications}

For \TODO{target decision-makers}, our findings suggest:

\begin{quote}
\textit{``Decisive, early action toward sustainability transition is both environmentally necessary and economically rational. Delays increase cumulative costs and risk irreversible damage. An adaptive management approach, guided by robust monitoring and periodic reassessment, can navigate uncertainty while maintaining progress toward long-term goals.''}
\end{quote}

%=== 11.4 研究局限与未来方向 ===
\subsection{Limitations and Future Directions}
\label{subsec:limitations}

While our framework provides valuable decision support, we acknowledge limitations including data gaps, model simplifications, and inherent uncertainties in long-term projections (detailed in Section~\ref{sec:evaluation}). Future research directions include:

\begin{itemize}[itemsep=0.2em]
    \item Spatial disaggregation through GIS integration
    \item Agent-based modeling of stakeholder behavior
    \item Real options analysis for valuing flexibility
    \item Participatory processes for broader stakeholder engagement
    \item Machine learning for enhanced pattern recognition
\end{itemize}

%=== 11.5 结语 ===
\subsection{Closing Remarks}
\label{subsec:closing}

The sustainability challenges of the 21st century demand integrated, systems-oriented approaches that transcend disciplinary silos. Our framework demonstrates that rigorous quantitative analysis, when combined with transparency about assumptions and uncertainties, can provide valuable guidance for navigating the complex trade-offs inherent in environmental-economic-social decision-making.

We hope this work contributes to the growing toolkit available to policymakers, planners, and communities striving to build a more sustainable future.

\vspace{1em}
\begin{center}
    \textit{``We do not inherit the Earth from our ancestors; we borrow it from our children.''}
    
    --- Native American Proverb
\end{center}
