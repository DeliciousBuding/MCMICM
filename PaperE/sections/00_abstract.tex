% Abstract - ICM Problem E (Sustainability-Oriented)
% 摘要:强调可持续性、系统思维、政策相关性

\begin{abstract}

% 开篇:问题背景与挑战
Addressing the complex challenge of \textbf{\TODO{specific environmental/sustainability topic}}, we develop an integrated, sustainability-oriented decision framework that balances environmental integrity, economic viability, and social equity. Our approach emphasizes \textbf{systems thinking}, \textbf{multi-stakeholder perspectives}, and \textbf{policy-actionable recommendations}.

% 第一部分:可持续性指标体系
\noindent\textbf{Sustainability Indicator System.}
We construct a comprehensive three-pillar indicator framework encompassing: (i) \textit{Environmental indicators}---including \TODO{e.g., carbon footprint, biodiversity index, water quality}; (ii) \textit{Economic indicators}---measuring \TODO{e.g., cost-effectiveness, job creation, GDP impact}; and (iii) \textit{Social indicators}---capturing \TODO{e.g., community well-being, equity distribution, public health}. Indicator weights are determined through an integrated \textbf{Analytic Hierarchy Process (AHP)} and \textbf{Entropy Weighting Method}, ensuring both expert judgment and data-driven objectivity.

% 第二部分:系统动力学建模
\noindent\textbf{System Dynamics Modeling.}
To capture the nonlinear feedback loops and temporal dynamics inherent in sustainability challenges, we develop a \textbf{System Dynamics (SD)} model incorporating \TODO{key stocks and flows, e.g., resource depletion, pollution accumulation, economic growth}. The model reveals critical \textbf{tipping points} and \textbf{leverage points} that inform intervention timing and intensity.

% 第三部分:情景分析
\noindent\textbf{Scenario Analysis.}
We design and simulate \textbf{four representative scenarios}: (1) \textit{Business-as-Usual (BAU)}---continuation of current policies; (2) \textit{Moderate Intervention}---incremental policy adjustments; (3) \textit{Aggressive Sustainability Transition}---transformative policy shifts; and (4) \textit{Climate Shock Stress Test}---resilience under extreme events. Results indicate that \TODO{key finding, e.g., Scenario 3 achieves X\% improvement in sustainability score while maintaining economic viability}.

% 第四部分:多准则决策
\noindent\textbf{Multi-Criteria Decision Analysis (MCDA).}
We employ \textbf{TOPSIS} (Technique for Order Preference by Similarity to Ideal Solution) to rank policy alternatives across the three sustainability pillars, complemented by \textbf{trade-off analysis} visualizations. Our framework explicitly quantifies the \textbf{Environmental-Economic Trade-off Frontier}, providing decision-makers with transparent options.

% 第五部分:政策建议
\noindent\textbf{Policy Recommendations.}
Based on our analysis, we provide tiered, actionable recommendations for \TODO{target stakeholders, e.g., government agencies, local communities, international organizations}, emphasizing \textbf{adaptive management strategies} that account for uncertainty and evolving conditions.

% 关键词
\begin{keywords}
Sustainability Assessment; System Dynamics; Multi-Criteria Decision Analysis (MCDA); Scenario Planning; Environmental-Economic Trade-offs; \TODO{1-2 topic-specific keywords}
\end{keywords}

\end{abstract}
