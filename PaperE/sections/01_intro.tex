% Section 1: Introduction
% 引言:问题背景 + 问题重述 + 利益相关者 + 本文工作

\section{Introduction}
\label{sec:intro}

%=== 1.1 问题背景:强调可持续性挑战 ===
\subsection{Problem Background}
\label{subsec:background}

% 【写作指导】E题引言需强调:
% 1. 环境问题的紧迫性与全球性
% 2. 多利益相关者的复杂博弈
% 3. 可持续发展的长期视角

The 21st century confronts humanity with unprecedented environmental challenges that demand integrated, systems-oriented solutions. \TODO{Describe the specific environmental context of Problem E, e.g., ecosystem degradation, climate adaptation, resource scarcity, clean energy transition.}

The complexity of this challenge arises from several intertwined factors:

\begin{itemize}[itemsep=0.3em]
    \item \textbf{Interconnected Systems:} Environmental, economic, and social systems are deeply coupled, with interventions in one domain producing cascading effects across others.
    
    \item \textbf{Temporal Dynamics:} Sustainability outcomes unfold over decades, requiring models that capture both short-term trade-offs and long-term equilibria.
    
    \item \textbf{Stakeholder Heterogeneity:} Different actors---governments, industries, communities, and future generations---hold divergent values and objectives that must be reconciled.
    
    \item \textbf{Deep Uncertainty:} Climate variability, technological disruptions, and policy shifts introduce substantial uncertainty into any planning framework.
\end{itemize}

\TODO{Add 1-2 paragraphs with specific statistics and references relevant to the Problem E topic. Cite authoritative sources such as IPCC, UN SDGs, or peer-reviewed literature.}

%=== 1.2 问题重述:精确界定任务 ===
\subsection{Restatement of the Problem}
\label{subsec:restatement}

To address the requirements of the 2026 ICM Problem E, we are tasked with developing a comprehensive analytical framework to:

\begin{enumerate}[itemsep=0.4em]
    \item \textbf{Task 1: \TODO{First Sub-Problem}}
    
    \TODO{Restate the first specific question from Problem E. Example: ``Develop a sustainability indicator system that captures environmental, economic, and social dimensions of [topic].''}
    
    \item \textbf{Task 2: \TODO{Second Sub-Problem}}
    
    \TODO{Restate the second specific question. Example: ``Model the dynamic interactions between [key system components] over a [time horizon].''}
    
    \item \textbf{Task 3: \TODO{Third Sub-Problem}}
    
    \TODO{Restate the third specific question. Example: ``Evaluate alternative policy scenarios and identify optimal intervention strategies.''}
    
    \item \textbf{Task 4: \TODO{Fourth Sub-Problem / Policy Memo}}
    
    \TODO{Restate the policy communication requirement. Example: ``Prepare a concise policy brief for [target decision-maker] summarizing key findings and recommendations.''}
\end{enumerate}

%=== 1.3 利益相关者与目标分析 ===
\subsection{Stakeholders and Objectives}
\label{subsec:stakeholders}

% 【E题核心】利益相关者分析是O奖论文的标志性要素
A sustainability-oriented decision framework must explicitly acknowledge the diverse stakeholders affected by and influencing the system. Table~\ref{tab:stakeholders} maps key stakeholders to their primary objectives and potential conflicts.

\begin{table}[H]
    \centering
    \caption{Stakeholder Analysis Matrix}
    \label{tab:stakeholders}
    \begin{tabularx}{\textwidth}{l X X X}
        \toprule
        \textbf{Stakeholder} & \textbf{Primary Objectives} & \textbf{Constraints} & \textbf{Potential Conflicts} \\
        \midrule
        \TODO{Government} & 
        \TODO{Environmental protection, economic growth, public welfare} & 
        \TODO{Budget limitations, political cycles} & 
        \TODO{Short-term growth vs. long-term sustainability} \\
        \addlinespace[0.5em]
        
        \TODO{Communities} & 
        \TODO{Livelihoods, health, cultural preservation} & 
        \TODO{Limited resources, information asymmetry} & 
        \TODO{Development vs. conservation} \\
        \addlinespace[0.5em]
        
        \TODO{Private Sector} & 
        \TODO{Profitability, market share, compliance} & 
        \TODO{Competition, capital constraints} & 
        \TODO{Cost externalization vs. responsibility} \\
        \addlinespace[0.5em]
        
        \TODO{NGOs} & 
        \TODO{Advocacy, transparency, equity} & 
        \TODO{Funding, influence} & 
        \TODO{Radical vs. incremental change} \\
        \addlinespace[0.5em]
        
        \TODO{Future Gen.} & 
        \TODO{Intergenerational equity, resources} & 
        \TODO{No direct voice} & 
        \TODO{Present vs. future well-being} \\
        \bottomrule
    \end{tabularx}
\end{table}

% 目标层次结构
Our framework adopts a \textbf{hierarchical objective structure} aligned with the three pillars of sustainability:

\begin{enumerate}[itemsep=0.2em]
    \item \textbf{Environmental Pillar:} \TODO{e.g., Minimize carbon emissions, preserve biodiversity, maintain ecosystem services}
    \item \textbf{Economic Pillar:} \TODO{e.g., Maximize cost-effectiveness, ensure economic viability, create green jobs}
    \item \textbf{Social Pillar:} \TODO{e.g., Promote equitable distribution, protect vulnerable populations, enhance quality of life}
\end{enumerate}

%=== 1.4 本文工作概述 ===
\subsection{Our Work}
\label{subsec:ourwork}

Our study introduces a \textbf{sustainability-oriented decision support framework} with the following methodological contributions:

\begin{enumerate}[itemsep=0.4em]
    \item \textbf{Three-Pillar Indicator System:}
    A comprehensive sustainability assessment framework integrating environmental, economic, and social dimensions through AHP-Entropy hybrid weighting.
    
    \item \textbf{System Dynamics Modeling:}
    A feedback-driven model capturing nonlinear interactions, tipping points, and long-term system trajectories under different policy regimes.
    
    \item \textbf{Scenario-Based Analysis:}
    Four carefully designed scenarios (BAU, Moderate, Aggressive, Stress Test) that span the policy option space and stress-test system resilience.
    
    \item \textbf{Multi-Criteria Decision Analysis:}
    TOPSIS-based ranking complemented by explicit trade-off frontier visualization to support transparent decision-making.
    
    \item \textbf{Actionable Policy Recommendations:}
    Tiered, stakeholder-specific recommendations with implementation roadmaps and adaptive management provisions.
\end{enumerate}

% 【插入框架流程图占位符】
\begin{figure}[H]
    \centering
    \begin{tikzpicture}[
        node distance=0.8cm and 1.2cm,
        block/.style={rectangle, draw, fill=blue!10, text width=2.8cm, minimum height=1cm, align=center, rounded corners},
        arrow/.style={->, >=stealth, thick}
    ]
        % 第一行:问题理解
        \node[block] (problem) {Problem\\Restatement};
        \node[block, right=of problem] (stakeholder) {Stakeholder\\Analysis};
        \node[block, right=of stakeholder] (boundary) {System\\Boundary};
        
        % 第二行:评估框架
        \node[block, below=of problem] (indicators) {Sustainability\\Indicators};
        \node[block, right=of indicators] (sd) {System\\Dynamics};
        \node[block, right=of sd] (scenarios) {Scenario\\Design};
        
        % 第三行:决策支持
        \node[block, below=of indicators] (mcda) {MCDA\\Analysis};
        \node[block, right=of mcda] (tradeoff) {Trade-off\\Visualization};
        \node[block, right=of tradeoff] (policy) {Policy\\Recommendations};
        
        % 连接箭头
        \draw[arrow] (problem) -- (stakeholder);
        \draw[arrow] (stakeholder) -- (boundary);
        \draw[arrow] (boundary) -- (scenarios);
        \draw[arrow] (problem) -- (indicators);
        \draw[arrow] (indicators) -- (sd);
        \draw[arrow] (sd) -- (scenarios);
        \draw[arrow] (indicators) -- (mcda);
        \draw[arrow] (scenarios) -- (mcda);
        \draw[arrow] (mcda) -- (tradeoff);
        \draw[arrow] (tradeoff) -- (policy);
    \end{tikzpicture}
    \caption{Overview of our sustainability-oriented decision framework.}
    \label{fig:framework}
\end{figure}
