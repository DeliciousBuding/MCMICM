% Section 7: Multi-Criteria Decision Analysis
% 多准则决策分析:TOPSIS与权衡可视化

\section{Multi-Criteria Decision Analysis}
\label{sec:mcda}

% 【E题核心】MCDA将技术分析转化为决策支持

%=== 7.1 MCDA方法论 ===
\subsection{MCDA Methodology}
\label{subsec:mcda_methodology}

Multi-Criteria Decision Analysis (MCDA) provides a structured framework for evaluating alternatives against multiple, potentially conflicting objectives~\cite{velasquez2013analysis}. For sustainability decisions involving environmental-economic-social trade-offs, MCDA offers:

\begin{itemize}[itemsep=0.2em]
    \item \textbf{Transparency:} Explicit articulation of criteria and weights.
    \item \textbf{Comprehensiveness:} Simultaneous consideration of all sustainability pillars.
    \item \textbf{Flexibility:} Accommodation of diverse stakeholder preferences through weight adjustment.
    \item \textbf{Defensibility:} Systematic documentation of decision logic.
\end{itemize}

We employ \textbf{TOPSIS} (Technique for Order Preference by Similarity to Ideal Solution) as our primary ranking method, complemented by trade-off frontier visualization.

%=== 7.2 决策矩阵构建 ===
\subsection{Decision Matrix Construction}
\label{subsec:decision_matrix}

The decision matrix $\mathbf{D}$ summarizes scenario performance across all indicators at the end of the projection horizon ($t = T$):

\begin{table}[H]
    \centering
    \caption{Decision Matrix: Scenario Performance on Sustainability Indicators}
    \label{tab:decision_matrix}
    \begin{tabular}{l c c c c}
        \toprule
        \textbf{Indicator} & \textbf{S1: BAU} & \textbf{S2: MOD} & \textbf{S3: AGG} & \textbf{S4: STRESS} \\
        \midrule
        \multicolumn{5}{l}{\textit{Environmental Pillar}} \\
        \TODO{Env. Indicator 1} & \TODO{value} & \TODO{value} & \TODO{value} & \TODO{value} \\
        \TODO{Env. Indicator 2} & \TODO{value} & \TODO{value} & \TODO{value} & \TODO{value} \\
        \TODO{Env. Indicator 3} & \TODO{value} & \TODO{value} & \TODO{value} & \TODO{value} \\
        \midrule
        \multicolumn{5}{l}{\textit{Economic Pillar}} \\
        \TODO{Econ. Indicator 1} & \TODO{value} & \TODO{value} & \TODO{value} & \TODO{value} \\
        \TODO{Econ. Indicator 2} & \TODO{value} & \TODO{value} & \TODO{value} & \TODO{value} \\
        \TODO{Econ. Indicator 3} & \TODO{value} & \TODO{value} & \TODO{value} & \TODO{value} \\
        \midrule
        \multicolumn{5}{l}{\textit{Social Pillar}} \\
        \TODO{Social Indicator 1} & \TODO{value} & \TODO{value} & \TODO{value} & \TODO{value} \\
        \TODO{Social Indicator 2} & \TODO{value} & \TODO{value} & \TODO{value} & \TODO{value} \\
        \TODO{Social Indicator 3} & \TODO{value} & \TODO{value} & \TODO{value} & \TODO{value} \\
        \bottomrule
    \end{tabular}
\end{table}

%=== 7.3 TOPSIS实现 ===
\subsection{TOPSIS Implementation}
\label{subsec:topsis}

The TOPSIS method ranks alternatives based on their geometric distance from ideal ($A^+$) and anti-ideal ($A^-$) solutions.

\subsubsection{Step 1: Normalize the Decision Matrix}
\begin{equation}
    r_{ij} = \frac{x_{ij}}{\sqrt{\sum_{j=1}^{m} x_{ij}^2}}
    \label{eq:topsis_norm}
\end{equation}

\subsubsection{Step 2: Apply Weights}
\begin{equation}
    v_{ij} = w_i \cdot r_{ij}
    \label{eq:topsis_weight}
\end{equation}

\subsubsection{Step 3: Determine Ideal and Anti-Ideal Solutions}
\begin{align}
    A^+ &= \{v_1^+, v_2^+, \ldots, v_n^+\}, \quad v_i^+ = \max_j(v_{ij}) \text{ for benefit criteria} \label{eq:ideal} \\
    A^- &= \{v_1^-, v_2^-, \ldots, v_n^-\}, \quad v_i^- = \min_j(v_{ij}) \text{ for benefit criteria} \label{eq:anti_ideal}
\end{align}

\subsubsection{Step 4: Calculate Separation Measures}
\begin{equation}
    d_j^+ = \sqrt{\sum_{i=1}^{n} (v_{ij} - v_i^+)^2}, \quad d_j^- = \sqrt{\sum_{i=1}^{n} (v_{ij} - v_i^-)^2}
    \label{eq:topsis_distance}
\end{equation}

\subsubsection{Step 5: Compute Relative Closeness}
\begin{equation}
    C_j^* = \frac{d_j^-}{d_j^+ + d_j^-}, \quad C_j^* \in [0, 1]
    \label{eq:topsis_closeness}
\end{equation}

The scenario with the highest $C_j^*$ is the most preferred alternative.

%=== 7.4 TOPSIS结果 ===
\subsection{TOPSIS Results}
\label{subsec:topsis_results}

\begin{table}[H]
    \centering
    \caption{TOPSIS Ranking Results}
    \label{tab:topsis_results}
    \begin{tabular}{l c c c c}
        \toprule
        \textbf{Scenario} & $d_j^+$ & $d_j^-$ & $C_j^*$ & \textbf{Rank} \\
        \midrule
        S1: BAU & \TODO{value} & \TODO{value} & \TODO{value} & \TODO{rank} \\
        S2: MOD & \TODO{value} & \TODO{value} & \TODO{value} & \TODO{rank} \\
        S3: AGG & \TODO{value} & \TODO{value} & \TODO{value} & \TODO{rank} \\
        S4: STRESS & \TODO{value} & \TODO{value} & \TODO{value} & \TODO{rank} \\
        \bottomrule
    \end{tabular}
\end{table}

%=== 7.5 权衡前沿可视化 ===
\subsection{Trade-off Frontier Visualization}
\label{subsec:tradeoff}

% 【E题O奖特征】权衡可视化是展示决策复杂性的关键

Beyond single-score ranking, we visualize the \textbf{Environmental-Economic Trade-off Frontier} to expose the inherent tensions in sustainability decision-making.

\begin{figure}[H]
    \centering
    \begin{subfigure}[t]{0.48\textwidth}
        \centering
        \fbox{\parbox[c][0.22\textheight][c]{\textwidth}{
            \centering
            \textbf{(a) Environmental vs. Economic Trade-off}
            \par\vspace{0.5em}
            \small \TODO{Scatter plot with Environmental Score on Y-axis and Economic Score on X-axis. Mark Pareto frontier. Label each scenario.}
        }}
        \caption{Environmental-Economic Pareto frontier}
        \label{fig:tradeoff_env_econ}
    \end{subfigure}\hfill
    \begin{subfigure}[t]{0.48\textwidth}
        \centering
        \fbox{\parbox[c][0.22\textheight][c]{\textwidth}{
            \centering
            \textbf{(b) Radar Chart Comparison}
            \par\vspace{0.5em}
            \small \TODO{Radar/spider chart showing all scenarios across environmental, economic, and social pillars.}
        }}
        \caption{Multi-pillar radar comparison}
        \label{fig:radar}
    \end{subfigure}
    \caption{Trade-off visualization for multi-criteria decision support.}
    \label{fig:tradeoff}
\end{figure}

%=== 7.6 利益相关者偏好敏感性 ===
\subsection{Stakeholder Preference Sensitivity}
\label{subsec:stakeholder_sensitivity}

Different stakeholders may assign different weights to the three pillars. We examine how rankings change under alternative weight profiles:

\begin{table}[H]
    \centering
    \caption{Scenario Rankings under Alternative Stakeholder Weights}
    \label{tab:stakeholder_weights}
    \begin{tabular}{l c c c c c c}
        \toprule
        \textbf{Weight Profile} & $w_{\text{Env}}$ & $w_{\text{Econ}}$ & $w_{\text{Soc}}$ & \textbf{Best Scenario} & \textbf{Worst Scenario} \\
        \midrule
        Balanced & 0.33 & 0.33 & 0.34 & \TODO{scenario} & \TODO{scenario} \\
        Environmental Priority & 0.50 & 0.25 & 0.25 & \TODO{scenario} & \TODO{scenario} \\
        Economic Priority & 0.25 & 0.50 & 0.25 & \TODO{scenario} & \TODO{scenario} \\
        Social Priority & 0.25 & 0.25 & 0.50 & \TODO{scenario} & \TODO{scenario} \\
        \bottomrule
    \end{tabular}
\end{table}

\TODO{Discuss implications: Under which stakeholder preference profiles does the optimal scenario change? What does this reveal about the robustness of the recommended policy?}
