% Section 9: Conclusion and Recommendations
% 结论与建议

\section{Conclusion and Recommendations}
\label{sec:conclusion}

% 9.1 政策建议
\subsection{Recommendations}
\label{subsec:recommendations}

Based on our quantitative analysis, we propose the following tiered recommendations for National Olympic Committees:

\begin{quote}
\textbf{To: NOC Directors and Sports Ministry Officials}

\vspace{0.5em}
\textbf{Priority A -- Short-Term Actions (1--2 years):}
\begin{itemize}[itemsep=0.2em]
    \item Establish a medal-yield audit mechanism that links funding increments to marginal medal returns and uncertainty.
    \item Prioritize sports with high expected gains and narrow predictive intervals (e.g., swimming, athletics for mid-tier nations).
\end{itemize}

\vspace{0.5em}
\textbf{Priority B -- Medium-Term Investments (3--5 years):}
\begin{itemize}[itemsep=0.2em]
    \item Recruit internationally experienced coaches in high-elasticity sports (wrestling, table tennis, weightlifting).
    \item Build youth development pipelines aligned with breakthrough-probability rankings.
\end{itemize}

\vspace{0.5em}
\textbf{Priority C -- Long-Term Strategy (6+ years):}
\begin{itemize}[itemsep=0.2em]
    \item Invest in sports science infrastructure and data analytics capabilities.
    \item Develop regional cooperation agreements for talent exchange and shared training facilities.
\end{itemize}
\end{quote}

% 9.2 总结
\subsection{Conclusion}
\label{subsec:conclusion_text}

In this paper, we developed a comprehensive, data-driven framework addressing four key challenges in Olympic sports analytics:

\begin{enumerate}[itemsep=0.3em]
    \item \textbf{Medal Forecasting:} Our PCA--LSTM--XGBoost pipeline achieves low MAE/MAPE with calibrated uncertainty intervals, providing reliable national-level predictions.
    
    \item \textbf{Breakthrough Identification:} The Hurdle--Tobit model successfully identifies high-potential NOCs for first-medal achievements, with San Marino and Kuwait flagged as top candidates.
    
    \item \textbf{Causal Policy Evaluation:} The DID analysis provides credible evidence that ``great coach'' interventions yield measurable medal gains, particularly in swimming and athletics.
    
    \item \textbf{Actionable Recommendations:} Our multi-objective allocation framework balances specialization and diversification, offering practical guidance for resource-constrained NOCs.
\end{enumerate}

Overall, our framework balances predictive performance, causal interpretability, and actionable recommendations. We believe this integrated approach can serve as a valuable decision-support tool for NOCs aiming to maximize their Olympic potential in 2028 and beyond.
