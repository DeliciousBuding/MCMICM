% Section 5: Model 2 - Coach Effect Estimation (DID)
% 模型二:教练效应因果推断(任务2核心)

\section{Model 2: Coach Effect Estimation via Difference-in-Differences}
\label{sec:model2}

% 5.1 数据描述与预处理
\subsection{Data Description \& Preprocessing}
\label{subsec:model2_data}

Building upon the dataset described in Section~\ref{subsec:model1_data}, we augment with coaching-related variables:
\begin{itemize}[itemsep=0.2em]
    \item \textbf{Treatment indicator:} Binary variable indicating whether an NOC hired a ``great coach'' (defined as a coach with prior international medal-winning experience).
    \item \textbf{Treatment timing:} The Olympic cycle in which the coaching change occurred.
    \item \textbf{Control covariates:} GDP growth, sports funding changes, and athlete pool size to control for confounding.
\end{itemize}

% 5.2 模型构建
\subsection{Model Formulation}
\label{subsec:model2_formulation}

We employ a Difference-in-Differences (DID) design to isolate the causal effect of coaching interventions from secular trends.

\subsubsection{DID Specification}
The canonical DID regression is:
\begin{equation}
    \label{eq:did_basic}
    Y_{i,t} = \alpha_i + \delta_t + \beta_{\mathrm{DID}} \cdot (\mathrm{Treat}_i \times \mathrm{Post}_t) + \boldsymbol{\gamma}^\top \mathbf{c}_{i,t} + \varepsilon_{i,t},
\end{equation}
where:
\begin{itemize}[itemsep=0.2em]
    \item $Y_{i,t}$ is the outcome (e.g., medal count or growth rate) for NOC $i$ at time $t$,
    \item $\alpha_i$ and $\delta_t$ are NOC and time fixed effects,
    \item $\mathrm{Treat}_i \times \mathrm{Post}_t$ is the interaction capturing the treatment effect,
    \item $\beta_{\mathrm{DID}}$ is the average treatment effect on the treated (ATT).
\end{itemize}

\subsubsection{Parallel Trends Assumption}
The key identification assumption is that treated and control NOCs would have followed parallel trends absent the intervention. We validate this through:
\begin{enumerate}[itemsep=0.2em]
    \item Pre-treatment trend tests (lead coefficients should be insignificant),
    \item Placebo tests with randomized treatment timing.
\end{enumerate}

% 5.3 求解算法
\subsection{Solution Algorithm}
\label{subsec:model2_algorithm}

\begin{algorithm}[H]
    \caption{DID Estimation with Robustness Checks}
    \label{alg:did}
    \DontPrintSemicolon
    \SetAlgoLined
    \KwIn{Panel data $\{(Y_{i,t}, \mathrm{Treat}_i, \mathrm{Post}_t, \mathbf{c}_{i,t})\}$}
    \KwOut{Estimated $\hat{\beta}_{\mathrm{DID}}$ with confidence interval}
    \BlankLine
    Construct treatment-control matching based on pre-treatment characteristics\;
    Estimate Eq.~\eqref{eq:did_basic} via OLS with clustered standard errors\;
    Conduct parallel trends test using event-study specification\;
    Perform placebo tests with fake treatment timing\;
    Report $\hat{\beta}_{\mathrm{DID}}$ and 95\% CI\;
\end{algorithm}

% 5.4 结果与解读
\subsection{Results \& Interpretation}
\label{subsec:model2_results}

% 【提示】此处插入DID结果图表
\begin{figure}[H]
    \centering
    \begin{subfigure}[t]{0.48\textwidth}
        \centering
        \fbox{\parbox[c][0.20\textheight][c]{\textwidth}{\centering\textbf{(a) Event Study Plot}\par\vspace{0.5em}\small Replace with pre/post coefficient plot.}}
        \caption{Parallel trends validation}
        \label{fig:model2_event}
    \end{subfigure}\hfill
    \begin{subfigure}[t]{0.48\textwidth}
        \centering
        \fbox{\parbox[c][0.20\textheight][c]{\textwidth}{\centering\textbf{(b) SHAP Attribution}\par\vspace{0.5em}\small Replace with SHAP summary plot.}}
        \caption{Feature importance via SHAP}
        \label{fig:model2_shap}
    \end{subfigure}
    \caption{DID estimation results for coaching effect.}
    \label{fig:model2_results}
\end{figure}

Results suggest measurable policy effects for Australia, South Korea, and Poland during 2020--2024. SHAP analysis attributes medal gains primarily to swimming and athletics, highlighting these as core drivers of coaching-induced improvement.
