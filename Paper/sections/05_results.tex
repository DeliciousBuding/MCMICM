\section{Results and Visualization}
\label{sec:results}

% 提示:本节专门用于“结果高分化呈现”。优先用对比图 + 置信区间 + 可解释性图(SHAP)。

\subsection{2024 Actual vs. 2028 Forecast (Bar Chart)}
\begin{figure}[H]
    \centering
    \begin{subfigure}[t]{0.32\textwidth}
        \centering
        \fbox{\parbox[c][0.26\textheight][c]{\textwidth}{\centering\textbf{(a) 2024 Actual vs. 2028 Forecast}\par\vspace{0.5em}\small Replace with a grouped bar chart.}}
        \caption{Actual vs. forecast}
        \label{fig:bar_2024_2028}
    \end{subfigure}\hfill
    \begin{subfigure}[t]{0.32\textwidth}
        \centering
        \fbox{\parbox[c][0.26\textheight][c]{\textwidth}{\centering\textbf{(b) SHAP Feature Contributions}\par\vspace{0.5em}\small Replace with a SHAP summary plot.}}
        \caption{SHAP summary}
        \label{fig:shap_summary}
    \end{subfigure}\hfill
    \begin{subfigure}[t]{0.32\textwidth}
        \centering
        \fbox{\parbox[c][0.26\textheight][c]{\textwidth}{\centering\textbf{(c) 95\% Prediction Interval}\par\vspace{0.5em}\small Replace with a line/area chart with shaded CI.}}
        \caption{95\% interval}
        \label{fig:ci_shaded}
    \end{subfigure}

    \caption{Three complementary result views: point forecast, interpretability, and uncertainty quantification.}
    \label{fig:results_triptych}
\end{figure}

\subsection{Interpretation and Discussion}
We interpret the forecasts through two lenses: (i) cross-country comparisons against the 2024 baseline, and (ii) model explainability via SHAP. In particular, the CI visualization in Fig.~\ref{fig:ci_shaded} is used to distinguish statistically meaningful improvements from fluctuations driven by uncertainty.