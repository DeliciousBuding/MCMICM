% ============================================================================
%                     Section 3: Assumptions and Notations
%                          假设条件与符号说明
% ============================================================================
% 写作约定:假设需可追溯(给出理由),符号表用 tabularx 自动换行并保持全篇一致。
% ============================================================================

\section{Assumptions and Notations}
\label{sec:assumptions}

% ============================================================================
%                          3.1 基本假设
% ============================================================================
\subsection{Basic Assumptions}
% 【提示】问题重述不要直接复制原题,要用专业术语(如"多目标优化"、"时间序列预测")进行转述。
% 【格式】使用 H1, H2... 编号,便于后文引用 "由假设 H3 可知..."

To establish a reasonable and solvable mathematical model, we make the following assumptions. Each assumption is justified with a brief rationale.

\begin{enumerate}[label=\textbf{H\arabic*.}, leftmargin=2em, itemsep=0.5em]
    
    \item \textbf{Data Integrity Assumption.}
    The provided dataset is complete and representative of the real-world scenario.
    
    \textit{Rationale:} The official dataset from COMAP is assumed to undergo quality control. Missing values, if any, are handled in our preprocessing pipeline (see Section~\ref{sec:data}).
    
    % -----------------------------------------------------------------------
    
    \item \textbf{Temporal Stability Assumption.}
    The underlying patterns and trends observed in historical data remain consistent during the prediction horizon.
    
    \textit{Rationale:} Short-term forecasting (e.g., 1--5 years) typically does not experience drastic structural changes unless major disruptions occur.
    
    % -----------------------------------------------------------------------
    
    \item \textbf{Independence Assumption.}
    The observations are independently and identically distributed (i.i.d.) within each time window.
    
    \textit{Rationale:} This is a standard assumption for many statistical and machine learning models, enabling the use of classical estimators.
    
    % -----------------------------------------------------------------------
    
    \item \textbf{Policy Lag Assumption.}
    There exists a time delay between policy implementation and observable effects.
    
    \textit{Rationale:} Government policies typically require administrative processes, public communication, and behavioral adaptation periods before measurable impacts emerge.
    
    % -----------------------------------------------------------------------
    
    \item \textbf{External Shock Exclusion.}
    We assume no major external shocks (e.g., pandemics, wars, natural disasters) occur during the modeling period.
    
    \textit{Rationale:} Such events are inherently unpredictable and would require scenario-based analysis beyond the scope of this competition.

\end{enumerate}

% 【技巧】如果假设过多,可添加 "Additional Assumptions" 子节
% \subsubsection{Problem-Specific Assumptions}
% ...

% ============================================================================
%                          3.2 符号说明
% ============================================================================
\subsection{Notations}
\label{subsec:notations}

% 【重要】符号表需覆盖全文所有关键变量,尤其是 PCA--LSTM 与 DID 的核心参数;并补充 Unit 列。

For clarity and consistency throughout this paper, we summarize the key symbols used in the PCA--LSTM forecasting module and the DID causal module in Table~\ref{tab:notations}.

\begin{table}[H]
    \centering
    \caption{Summary of Key Notations (Forecasting + Causal Inference)}
    \label{tab:notations}
    \begin{tabularx}{\textwidth}{c X c}
        \toprule
        \textbf{Symbol} & \textbf{Description} & \textbf{Unit} \\
        \midrule
        $i$ & Index of a National Olympic Committee (NOC) & -- \\
        $t$ & Olympic cycle / year index & year \\
        $\mathcal{I}$ & Set of NOCs, $|\mathcal{I}| = N$ & -- \\
        $\mathcal{T}$ & Set of observed cycles/years, $|\mathcal{T}| = T$ & -- \\
        \midrule
        $M^{\mathrm{gold}}_{i,t}$ & Observed number of gold medals for NOC $i$ at time $t$ & medals \\
        $M^{\mathrm{tot}}_{i,t}$ & Observed number of total medals for NOC $i$ at time $t$ & medals \\
        $\hat{M}_{i,t}$ & Model prediction of medal count (gold or total) & medals \\
        $\mathrm{CI}^{95\%}_{i,t}$ & 95\% predictive interval for $\hat{M}_{i,t}$ & medals \\
        \midrule
        $\mathbf{x}_{i,t} \in \mathbb{R}^{p}$ & Raw feature vector (event-level + macro covariates) for NOC $i$ at time $t$ & mixed \\
        $\tilde{\mathbf{x}}_{i,t}$ & Standardized feature vector of $\mathbf{x}_{i,t}$ & -- \\
        $\mathbf{W} \in \mathbb{R}^{p \times k}$ & PCA loading matrix (top-$k$ principal directions) & -- \\
        $\mathbf{z}_{i,t} \in \mathbb{R}^{k}$ & PCA scores, $\mathbf{z}_{i,t} = \mathbf{W}^\top \tilde{\mathbf{x}}_{i,t}$ & -- \\
        $L$ & LSTM look-back window length & cycles \\
        $\mathbf{h}_{i,t}$ & LSTM hidden representation for NOC $i$ at time $t$ & -- \\
        $\Theta_{\mathrm{LSTM}}$ & Trainable parameters of the LSTM & -- \\
        $\mathbf{f}_{i,t}$ & Fused feature vector used by the downstream regressor (e.g., $[\mathbf{h}_{i,t};\,\mathrm{Host}_{i,t};\,\mathbf{u}_{i,t}]$) & mixed \\
        $\mathcal{B}$ & Number of bootstrap resamples for uncertainty quantification & -- \\
        \midrule
        $Y_{i,t}$ & Outcome variable in DID (e.g., medal growth, efficiency index) & varies \\
        $\mathrm{Treat}_i$ & Treatment indicator: 1 if NOC $i$ adopts a ``great coach'' intervention & -- \\
        $\mathrm{Post}_t$ & Post-intervention indicator: 1 for periods after policy implementation & -- \\
        $\mathrm{Treat}_i\times\mathrm{Post}_t$ & DID interaction term capturing the treatment effect & -- \\
        $\beta_{\mathrm{DID}}$ & DID coefficient (average treatment effect on the treated) & varies \\
        $\boldsymbol{\gamma}$ & Coefficients for control covariates $\mathbf{c}_{i,t}$ & varies \\
        $\alpha_i,\ \delta_t$ & NOC and time fixed effects (if included) & -- \\
        $\varepsilon_{i,t}$ & Error term & -- \\
        \bottomrule
    \end{tabularx}
\end{table}

% 【高级技巧】如果符号太多,可以按类别分成多个小表格
% 例如:Table 2a - Input Variables, Table 2b - Model Parameters
