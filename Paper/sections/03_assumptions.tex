% ============================================================================
%                     Section 3: Assumptions and Notations
%                          假设条件与符号说明
% ============================================================================
% 写作约定:假设需可追溯(给出理由),符号表用 tabularx 自动换行并保持全篇一致。
% ============================================================================

\section{Assumptions and Notations}
\label{sec:assumptions}

% ============================================================================
%                          3.1 基本假设
% ============================================================================
\subsection{Basic Assumptions}
% 【提示】问题重述不要直接复制原题,要用专业术语(如"多目标优化"、"时间序列预测")进行转述。
% 【格式】使用 H1, H2... 编号,便于后文引用 "由假设 H3 可知..."

To establish a reasonable and solvable mathematical model, we make the following assumptions. Each assumption is justified with a brief rationale.

\begin{enumerate}[label=\textbf{H\arabic*.}, leftmargin=2em, itemsep=0.5em]
    
    \item \textbf{Data Integrity Assumption.}
    The provided dataset is complete and representative of the real-world scenario.
    
    \textit{Rationale:} The official dataset from COMAP is assumed to undergo quality control. Missing values, if any, are handled in our preprocessing pipeline (see Section~\ref{sec:data}).
    
    % -----------------------------------------------------------------------
    
    \item \textbf{Temporal Stability Assumption.}
    The underlying patterns and trends observed in historical data remain consistent during the prediction horizon.
    
    \textit{Rationale:} Short-term forecasting (e.g., 1--5 years) typically does not experience drastic structural changes unless major disruptions occur.
    
    % -----------------------------------------------------------------------
    
    \item \textbf{Independence Assumption.}
    The observations are independently and identically distributed (i.i.d.) within each time window.
    
    \textit{Rationale:} This is a standard assumption for many statistical and machine learning models, enabling the use of classical estimators.
    
    % -----------------------------------------------------------------------
    
    \item \textbf{Policy Lag Assumption.}
    There exists a time delay between policy implementation and observable effects.
    
    \textit{Rationale:} Government policies typically require administrative processes, public communication, and behavioral adaptation periods before measurable impacts emerge.
    
    % -----------------------------------------------------------------------
    
    \item \textbf{External Shock Exclusion.}
    We assume no major external shocks (e.g., pandemics, wars, natural disasters) occur during the modeling period.
    
    \textit{Rationale:} Such events are inherently unpredictable and would require scenario-based analysis beyond the scope of this competition.

\end{enumerate}

% 【技巧】如果假设过多,可添加 "Additional Assumptions" 子节
% \subsubsection{Problem-Specific Assumptions}
% ...

% ============================================================================
%                          3.2 符号说明
% ============================================================================
\subsection{Notations}
\label{subsec:notations}

% 【重要】使用 tabularx + X 列类型,让 Description 自动换行
% 【注意】不要手动用 \\ 或 \newline 断行,tabularx 会自动处理

For clarity and consistency throughout this paper, we define the key symbols and their meanings in Table~\ref{tab:notations}.

% --- 符号表 ---
\begin{table}[H]
    \centering
    \caption{Summary of Key Notations}
    \label{tab:notations}
    % 【说明】X 列类型会自动伸缩填充剩余宽度,并自动换行
    \begin{tabularx}{\textwidth}{c X c}
        \toprule
        \textbf{Symbol} & \textbf{Description} & \textbf{Unit} \\
        \midrule
        % --- 集合与索引 ---
        $i, j$ & Indices for regions or entities in the dataset & -- \\
        $t$ & Time index (year, month, or discrete period) & -- \\
        $\mathcal{N}$ & Set of all nodes (regions) in the network, $\mathcal{N} = \{1, 2, \ldots, n\}$ & -- \\
        $\mathcal{T}$ & Time horizon of the study, $\mathcal{T} = \{1, 2, \ldots, T\}$ & -- \\
        \midrule
        % --- 输入变量 ---
        $x_{i,t}$ & Primary feature variable for region $i$ at time $t$ & varies \\
        $P_i$ & Population density of region $i$ & people/km\textsuperscript{2} \\
        $G_i$ & GDP per capita of region $i$ & USD \\
        \midrule
        % --- 输出/目标变量 ---
        $y_{i,t}$ & Target variable (prediction objective) for region $i$ at time $t$ & varies \\
        $\hat{y}_{i,t}$ & Predicted value of $y_{i,t}$ by our model & varies \\
        \midrule
        % --- 模型参数 ---
        $\omega$ & Weighting coefficient for the multi-objective function & -- \\
        $\lambda$ & Regularization parameter to prevent overfitting & -- \\
        $\eta$ & Learning rate in gradient-based optimization & -- \\
        $\Theta$ & Set of all model parameters, $\Theta = \{\theta_1, \theta_2, \ldots\}$ & -- \\
        \midrule
        % --- 性能指标 ---
        $J(\cdot)$ & Objective function to be minimized or maximized & -- \\
        $\epsilon$ & Error tolerance or convergence threshold & -- \\
        $\Delta t$ & Time interval between consecutive observations & years \\
        \bottomrule
    \end{tabularx}
\end{table}

% 【高级技巧】如果符号太多,可以按类别分成多个小表格
% 例如:Table 2a - Input Variables, Table 2b - Model Parameters
