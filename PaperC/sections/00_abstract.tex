\begin{abstract}
We propose an integrated, data-driven framework to support (i) Olympic medal forecasting, (ii) ``first-medal''/breakthrough identification, (iii) causal evaluation of elite-coach interventions, and (iv) evidence-based portfolio recommendations for National Olympic Committees (NOCs). Our pipeline fuses heterogeneous sources and emphasizes both predictive accuracy and interpretability.

\noindent\textbf{Medal forecasting (PCA--LSTM + XGBoost--Bootstrap).}
We first compress high-dimensional event-level indicators via Principal Component Analysis (PCA) and feed the resulting temporal components into a Long Short-Term Memory (LSTM) network to learn dynamic performance trajectories while accounting for host effects. The final predictor is a dual-channel XGBoost model equipped with bootstrap resampling to report uncertainty, producing a \textbf{95\% confidence interval} for each nation-level forecast. Empirically, the model achieves low \textbf{MAE} and \textbf{MAPE}, indicating strong out-of-sample stability.

\noindent\textbf{Breakthrough screening (nonparametric testing).}
To identify potential ``gold-breakthrough'' NOCs, we apply distribution-free hypothesis testing on predicted improvements. We flag San Marino and Kuwait as high-likelihood candidates, with gold-medal breakthrough probabilities of \textbf{84.7\%} and \textbf{68.4\%}, respectively.

\noindent\textbf{Coach-effect estimation (DID + diagnostics).}
We quantify the causal impact of ``great coach'' replacements using a Difference-in-Differences (DID) design, accompanied by statistical significance checks and parallel-trends validation. Results suggest measurable policy effects for Australia, South Korea, and Poland during 2020--2024. To translate causal findings into actionable levers, we further employ SHapley Additive exPlanations (SHAP) to attribute medal gains to event groups, highlighting swimming and athletics as core drivers.

\noindent\textbf{First-ever medal probability (Hurdle--Tobit) and resource planning.}
To handle zero inflation and cross-country heterogeneity, we build a Hurdle--Tobit hybrid model and estimate that Angola and Bangladesh exceed a \textbf{45\%} probability of earning a first medal in the next Games. Finally, a multi-objective allocation module balances ``depth'' (specialization) and ``breadth'' (diversification), recommending coach imports in high-elasticity sports such as wrestling and table tennis for high-potential NOCs.

\begin{keywords}
Olympic medal forecasting; PCA--LSTM; XGBoost with bootstrap; Difference-in-Differences; Hurdle--Tobit; SHAP.
\end{keywords}
\end{abstract}
