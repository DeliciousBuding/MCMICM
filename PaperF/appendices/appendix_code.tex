% appendix_code.tex - 代码附录
% ICM Problem F - Policy Science

\section*{Appendix A: Source Code}
\addcontentsline{toc}{section}{Appendix A: Source Code}
\label{app:code}

% 【写作指导】代码附录需要:
% 1. 核心算法的可执行代码
% 2. 清晰的注释
% 3. 使用说明

\subsection*{A.1 Mechanism Model Implementation}

The following Python code implements the system dynamics model described in \secref{sec:model1}:

\begin{lstlisting}[language=Python, caption={System Dynamics Model Core}, label={lst:mechanism}]
"""
System Dynamics Model for Policy Analysis
ICM 2026 - Problem F
Team #2617892
"""
import numpy as np
from dataclasses import dataclass
from typing import Callable, Tuple

@dataclass
class ModelParameters:
    """Model parameters with default values"""
    alpha_0: float = 0.0    # Intercept
    alpha_1: float = 1.0    # Linear policy effect
    alpha_2: float = -0.1   # Quadratic (diminishing returns)
    beta: np.ndarray = None # Covariate effects
    lambda_fb: float = 0.2  # Feedback strength
    tau: int = 2            # Policy lag (periods)
    delta: float = 0.95     # Discount factor
    
def simulate_dynamics(
    policy: np.ndarray,
    initial_state: float,
    params: ModelParameters,
    T: int
) -> Tuple[np.ndarray, np.ndarray]:
    """
    Simulate system dynamics under given policy trajectory.
    
    Args:
        policy: Policy intensity over time, shape (T,)
        initial_state: Initial stock level
        params: Model parameters
        T: Time horizon
    
    Returns:
        states: Stock trajectory, shape (T+1,)
        outcomes: Outcome trajectory, shape (T,)
    """
    states = np.zeros(T + 1)
    outcomes = np.zeros(T)
    states[0] = initial_state
    
    for t in range(T):
        # Inflow rate (Eq. 2 in paper)
        flow_in = (
            params.alpha_0 
            + params.alpha_1 * policy[t] 
            + params.alpha_2 * policy[t]**2
        )
        
        # Feedback from lagged outcome (if t >= tau)
        if t >= params.tau:
            flow_in += params.lambda_fb * outcomes[t - params.tau]
        
        # Outflow rate (proportional depletion)
        flow_out = 0.1 * states[t]
        
        # State evolution (Eq. 1 in paper)
        states[t + 1] = states[t] + flow_in - flow_out
        
        # Outcome mapping (Eq. 4 in paper)
        outcomes[t] = np.log(1 + max(states[t + 1], 0))
    
    return states, outcomes

# TODO: Add parameter estimation functions
# TODO: Add model validation routines
\end{lstlisting}

\subsection*{A.2 Multi-Objective Optimization}

\begin{lstlisting}[language=Python, caption={NSGA-II Based Policy Optimizer}, label={lst:optimizer}]
"""
Multi-Objective Policy Optimization using NSGA-II
"""
import numpy as np
from typing import List, Tuple
# from pymoo.algorithms.moo.nsga2 import NSGA2  # Uncomment for full implementation

def policy_objectives(
    policy: np.ndarray,
    params: dict
) -> Tuple[float, float, float]:
    """
    Evaluate policy against multiple objectives.
    
    Args:
        policy: Policy decision vector
        params: Dictionary of model parameters
    
    Returns:
        Tuple of (welfare, gini, cost) - note: minimize all
    """
    # Simulate dynamics
    states, outcomes = simulate_dynamics(
        policy, 
        params['initial_state'],
        params['model_params'],
        len(policy)
    )
    
    # Objective 1: Welfare (maximize -> negate for minimization)
    welfare = -np.sum(
        params['discount_factors'] * outcomes
    )
    
    # Objective 2: Gini coefficient (minimize)
    gini = compute_gini(outcomes)
    
    # Objective 3: Total cost (minimize)
    cost = np.sum(params['unit_costs'] * policy)
    
    return welfare, gini, cost

def compute_gini(values: np.ndarray) -> float:
    """Compute Gini coefficient."""
    sorted_vals = np.sort(values)
    n = len(values)
    cumsum = np.cumsum(sorted_vals)
    return (2 * np.sum((np.arange(1, n+1) * sorted_vals))) / (n * np.sum(sorted_vals)) - (n + 1) / n

def run_optimization(
    problem_params: dict,
    pop_size: int = 100,
    n_gen: int = 200
) -> List[np.ndarray]:
    """
    Run NSGA-II optimization to find Pareto-optimal policies.
    
    Returns:
        List of Pareto-optimal policy vectors
    """
    # TODO: Implement full NSGA-II with pymoo
    # Placeholder for demonstration
    print("Running NSGA-II optimization...")
    print(f"Population size: {pop_size}")
    print(f"Generations: {n_gen}")
    
    # Return placeholder
    return [np.random.rand(10) for _ in range(10)]  # Pareto set

# TODO: Add constraint handling
# TODO: Add local search refinement
\end{lstlisting}

\subsection*{A.3 Sensitivity Analysis}

\begin{lstlisting}[language=Python, caption={Sobol Sensitivity Analysis}, label={lst:sensitivity}]
"""
Global Sensitivity Analysis using Sobol Indices
"""
import numpy as np
# from SALib.sample import saltelli
# from SALib.analyze import sobol  # Uncomment for full implementation

def define_problem(param_ranges: dict) -> dict:
    """Define problem for SALib."""
    return {
        'num_vars': len(param_ranges),
        'names': list(param_ranges.keys()),
        'bounds': list(param_ranges.values())
    }

def run_sobol_analysis(
    model_func,
    param_ranges: dict,
    n_samples: int = 1024
) -> dict:
    """
    Perform Sobol sensitivity analysis.
    
    Args:
        model_func: Model function mapping parameters to output
        param_ranges: Dict of {param_name: [low, high]}
        n_samples: Number of samples (will be multiplied by 2*(d+1))
    
    Returns:
        Dict with first-order (S1) and total-order (ST) indices
    """
    problem = define_problem(param_ranges)
    
    # TODO: Generate Saltelli samples
    # param_samples = saltelli.sample(problem, n_samples)
    
    # TODO: Evaluate model
    # Y = np.array([model_func(p) for p in param_samples])
    
    # TODO: Analyze
    # Si = sobol.analyze(problem, Y)
    
    # Placeholder results
    Si = {
        'S1': np.random.rand(len(param_ranges)),  # First-order
        'ST': np.random.rand(len(param_ranges)),  # Total-order
        'names': list(param_ranges.keys())
    }
    
    return Si

# TODO: Add visualization functions
# TODO: Add OAT analysis
\end{lstlisting}

\vspace{1em}
\noindent\textbf{Note:} Complete source code with all dependencies is available at: \TODO{GitHub repository URL or submission attachment reference}.
