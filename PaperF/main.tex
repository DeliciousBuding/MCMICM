% MCM/ICM 2026 LaTeX 主控文件 - Problem F (Policy Science)
% 编译方式:XeLaTeX + Biber(推荐用latexmk自动编译)
% !TEX program = xelatex
% !BIB program = biber

\documentclass{mcmthesis}

% 兼容性补丁:避免旧aux残留的chapter计数器导致hyperref报错
\newcounter{chapter}
\renewcommand{\thechapter}{\arabic{chapter}}

% MCM题目配置(比赛时务必修改tcn为真实队号)
\mcmsetup{
    tcn = {2617892},
    problem = {F},
    sheet = true,
    titleinsheet = true,
    keywordsinsheet = true,
    titlepage = false
}

% 目录开关
\makeatletter
\@ifundefined{showtoctrue}{}{\showtoctrue}
\makeatother

%=== 宏包加载 ===

% 字体与排版
\usepackage{palatino}
\usepackage{microtype}
\setlength{\emergencystretch}{2em}

% 页面与布局
\usepackage{lastpage}
\usepackage{zref-abspage}
\usepackage{float}
\usepackage{geometry}
\setlength{\headheight}{15pt}

% 数学公式
\usepackage{amsmath, amssymb, amsthm}
\usepackage{mathtools}

% 定理环境
\newtheorem{assumption}{Assumption}
\newtheorem{definition}{Definition}
\newtheorem{theorem}{Theorem}
\newtheorem{lemma}{Lemma}

% 表格美化(政策报告常用)
\usepackage{booktabs}
\usepackage{tabularx}
\usepackage{longtable}
\usepackage{multirow}
\usepackage{array}
\usepackage{colortbl}

% 算法伪代码(顶会风格)
\usepackage[ruled, vlined, linesnumbered]{algorithm2e}
\SetAlgorithmName{Algorithm}{algorithm}{List of Algorithms}
\SetKwInput{KwIn}{Input}
\SetKwInput{KwOut}{Output}
\SetKwInput{KwData}{Data}
\SetKwInput{KwResult}{Result}

% 代码高亮
\usepackage{listings}
\lstset{
    basicstyle=\small\ttfamily,
    keywordstyle=\color{blue},
    commentstyle=\color{gray},
    numbers=left,
    numberstyle=\tiny\color{gray},
    frame=single,
    breaklines=true,
    tabsize=4
}

% 图形与颜色
\usepackage{graphicx}
\usepackage{subcaption}
\usepackage{xcolor}
\usepackage{tikz}
\usetikzlibrary{shapes, arrows, positioning, calc, fit, backgrounds, decorations.pathreplacing}

% 列表与枚举(政策要点常用)
\usepackage{enumitem}
\setlist[enumerate]{itemsep=0pt, parsep=0pt}
\setlist[itemize]{itemsep=0pt, parsep=0pt}

% 超链接
\usepackage{hyperref}
\hypersetup{
    colorlinks=true,
    linkcolor=black,
    citecolor=green!50!black,
    urlcolor=blue!80!black
}

% 目录格式调整
\usepackage{tocloft}
\setlength{\cftbeforesecskip}{2pt}
\setlength{\cftbeforesubsecskip}{1pt}

% 参考文献
\usepackage[backend=biber, style=ieee, sorting=none]{biblatex}
\addbibresource{ref.bib}

% 加载共享宏定义
% _shared_macros.tex - 共享宏定义文件
% 用于多人协作时统一符号和变量命名,避免冲突
% 使用说明:在 main.tex 中通过 % _shared_macros.tex - 共享宏定义文件
% 用于多人协作时统一符号和变量命名,避免冲突
% 使用说明:在 main.tex 中通过 % _shared_macros.tex - 共享宏定义文件
% 用于多人协作时统一符号和变量命名,避免冲突
% 使用说明:在 main.tex 中通过 \input{sections/_shared_macros.tex} 引入

%=== 政策分析专用符号 ===

% 政策变量
\newcommand{\policy}{\pi}                    % 政策方案
\newcommand{\policyspace}{\Pi}               % 政策空间
\newcommand{\intervention}{I}                % 干预措施
\newcommand{\baseline}{B}                    % 基准情景

% 效果评估
\newcommand{\outcome}{Y}                     % 结果变量
\newcommand{\treatment}{T}                   % 处理变量
\newcommand{\effect}{\tau}                   % 处理效应
\newcommand{\ate}{\bar{\tau}}               % 平均处理效应 (ATE)
\newcommand{\att}{\tau_{\text{ATT}}}        % 处理组平均效应

% 利益相关者
\newcommand{\stakeholder}{S}                 % 利益相关者
\newcommand{\utility}{U}                     % 效用函数
\newcommand{\welfare}{W}                     % 社会福利

% 时间与动态
\newcommand{\horizon}{T}                     % 规划期
\newcommand{\discount}{\delta}               % 折现因子
\newcommand{\lagtime}{\tau_{\text{lag}}}    % 政策时滞

% 不确定性
\newcommand{\uncertainty}{\epsilon}          % 不确定性项
\newcommand{\robust}{\mathcal{R}}           % 鲁棒性指标

%=== 模型专用符号 ===

% 系统动力学
\newcommand{\stock}{S}                       % 存量
\newcommand{\flow}{F}                        % 流量
\newcommand{\feedback}{\lambda}              % 反馈系数

% 优化
\newcommand{\objective}{\mathcal{J}}         % 目标函数
\newcommand{\constraint}{\mathcal{C}}        % 约束集
\newcommand{\optimal}{^*}                    % 最优解标记

% 评估指标
\newcommand{\efficiency}{\eta}               % 效率
\newcommand{\equity}{\xi}                    % 公平性
\newcommand{\sustainability}{\sigma}         % 可持续性

%=== 常用缩写 ===

\newcommand{\ie}{\textit{i.e.}}
\newcommand{\eg}{\textit{e.g.}}
\newcommand{\etc}{\textit{etc.}}
\newcommand{\wrt}{w.r.t.}
\newcommand{\st}{\text{s.t.}}

%=== 数学环境 ===

% 期望、概率
\newcommand{\E}{\mathbb{E}}                  % 期望
\newcommand{\Prob}{\mathbb{P}}               % 概率
\newcommand{\Var}{\text{Var}}                % 方差

% 集合
\newcommand{\R}{\mathbb{R}}                  % 实数集
\newcommand{\N}{\mathbb{N}}                  % 自然数集
\newcommand{\Z}{\mathbb{Z}}                  % 整数集

%=== 颜色定义(用于图表一致性)===

\definecolor{policyblue}{RGB}{31, 119, 180}
\definecolor{resultgreen}{RGB}{44, 160, 44}
\definecolor{warningorange}{RGB}{255, 127, 14}
\definecolor{riskred}{RGB}{214, 39, 40}
\definecolor{neutralgray}{RGB}{127, 127, 127}
 引入

%=== 政策分析专用符号 ===

% 政策变量
\newcommand{\policy}{\pi}                    % 政策方案
\newcommand{\policyspace}{\Pi}               % 政策空间
\newcommand{\intervention}{I}                % 干预措施
\newcommand{\baseline}{B}                    % 基准情景

% 效果评估
\newcommand{\outcome}{Y}                     % 结果变量
\newcommand{\treatment}{T}                   % 处理变量
\newcommand{\effect}{\tau}                   % 处理效应
\newcommand{\ate}{\bar{\tau}}               % 平均处理效应 (ATE)
\newcommand{\att}{\tau_{\text{ATT}}}        % 处理组平均效应

% 利益相关者
\newcommand{\stakeholder}{S}                 % 利益相关者
\newcommand{\utility}{U}                     % 效用函数
\newcommand{\welfare}{W}                     % 社会福利

% 时间与动态
\newcommand{\horizon}{T}                     % 规划期
\newcommand{\discount}{\delta}               % 折现因子
\newcommand{\lagtime}{\tau_{\text{lag}}}    % 政策时滞

% 不确定性
\newcommand{\uncertainty}{\epsilon}          % 不确定性项
\newcommand{\robust}{\mathcal{R}}           % 鲁棒性指标

%=== 模型专用符号 ===

% 系统动力学
\newcommand{\stock}{S}                       % 存量
\newcommand{\flow}{F}                        % 流量
\newcommand{\feedback}{\lambda}              % 反馈系数

% 优化
\newcommand{\objective}{\mathcal{J}}         % 目标函数
\newcommand{\constraint}{\mathcal{C}}        % 约束集
\newcommand{\optimal}{^*}                    % 最优解标记

% 评估指标
\newcommand{\efficiency}{\eta}               % 效率
\newcommand{\equity}{\xi}                    % 公平性
\newcommand{\sustainability}{\sigma}         % 可持续性

%=== 常用缩写 ===

\newcommand{\ie}{\textit{i.e.}}
\newcommand{\eg}{\textit{e.g.}}
\newcommand{\etc}{\textit{etc.}}
\newcommand{\wrt}{w.r.t.}
\newcommand{\st}{\text{s.t.}}

%=== 数学环境 ===

% 期望、概率
\newcommand{\E}{\mathbb{E}}                  % 期望
\newcommand{\Prob}{\mathbb{P}}               % 概率
\newcommand{\Var}{\text{Var}}                % 方差

% 集合
\newcommand{\R}{\mathbb{R}}                  % 实数集
\newcommand{\N}{\mathbb{N}}                  % 自然数集
\newcommand{\Z}{\mathbb{Z}}                  % 整数集

%=== 颜色定义(用于图表一致性)===

\definecolor{policyblue}{RGB}{31, 119, 180}
\definecolor{resultgreen}{RGB}{44, 160, 44}
\definecolor{warningorange}{RGB}{255, 127, 14}
\definecolor{riskred}{RGB}{214, 39, 40}
\definecolor{neutralgray}{RGB}{127, 127, 127}
 引入

%=== 政策分析专用符号 ===

% 政策变量
\newcommand{\policy}{\pi}                    % 政策方案
\newcommand{\policyspace}{\Pi}               % 政策空间
\newcommand{\intervention}{I}                % 干预措施
\newcommand{\baseline}{B}                    % 基准情景

% 效果评估
\newcommand{\outcome}{Y}                     % 结果变量
\newcommand{\treatment}{T}                   % 处理变量
\newcommand{\effect}{\tau}                   % 处理效应
\newcommand{\ate}{\bar{\tau}}               % 平均处理效应 (ATE)
\newcommand{\att}{\tau_{\text{ATT}}}        % 处理组平均效应

% 利益相关者
\newcommand{\stakeholder}{S}                 % 利益相关者
\newcommand{\utility}{U}                     % 效用函数
\newcommand{\welfare}{W}                     % 社会福利

% 时间与动态
\newcommand{\horizon}{T}                     % 规划期
\newcommand{\discount}{\delta}               % 折现因子
\newcommand{\lagtime}{\tau_{\text{lag}}}    % 政策时滞

% 不确定性
\newcommand{\uncertainty}{\epsilon}          % 不确定性项
\newcommand{\robust}{\mathcal{R}}           % 鲁棒性指标

%=== 模型专用符号 ===

% 系统动力学
\newcommand{\stock}{S}                       % 存量
\newcommand{\flow}{F}                        % 流量
\newcommand{\feedback}{\lambda}              % 反馈系数

% 优化
\newcommand{\objective}{\mathcal{J}}         % 目标函数
\newcommand{\constraint}{\mathcal{C}}        % 约束集
\newcommand{\optimal}{^*}                    % 最优解标记

% 评估指标
\newcommand{\efficiency}{\eta}               % 效率
\newcommand{\equity}{\xi}                    % 公平性
\newcommand{\sustainability}{\sigma}         % 可持续性

%=== 常用缩写 ===

\newcommand{\ie}{\textit{i.e.}}
\newcommand{\eg}{\textit{e.g.}}
\newcommand{\etc}{\textit{etc.}}
\newcommand{\wrt}{w.r.t.}
\newcommand{\st}{\text{s.t.}}

%=== 数学环境 ===

% 期望、概率
\newcommand{\E}{\mathbb{E}}                  % 期望
\newcommand{\Prob}{\mathbb{P}}               % 概率
\newcommand{\Var}{\text{Var}}                % 方差

% 集合
\newcommand{\R}{\mathbb{R}}                  % 实数集
\newcommand{\N}{\mathbb{N}}                  % 自然数集
\newcommand{\Z}{\mathbb{Z}}                  % 整数集

%=== 颜色定义(用于图表一致性)===

\definecolor{policyblue}{RGB}{31, 119, 180}
\definecolor{resultgreen}{RGB}{44, 160, 44}
\definecolor{warningorange}{RGB}{255, 127, 14}
\definecolor{riskred}{RGB}{214, 39, 40}
\definecolor{neutralgray}{RGB}{127, 127, 127}


%=== 自定义命令 ===

% 正文页数计数器(实现 Page X of Y)
\newcounter{mainpages}
\makeatletter
\newcommand{\savemainpages}{%
    \immediate\write\@auxout{\string\setcounter{mainpages}{\the\value{page}}}%
}
\makeatother

% 快捷引用命令
\newcommand{\figref}[1]{Fig.~\ref{#1}}
\newcommand{\tabref}[1]{Table~\ref{#1}}
\newcommand{\eqnref}[1]{Eq.~(\ref{#1})}
\newcommand{\algref}[1]{Algorithm~\ref{#1}}
\newcommand{\secref}[1]{Section~\ref{#1}}

% TODO标记命令(用于标识待填内容)
\newcommand{\TODO}[1]{\textcolor{red}{\textbf{[TODO: #1]}}}

% 政策备忘录专用环境
\newenvironment{memorandum}{%
    \begin{center}
    \large\textbf{MEMORANDUM}
    \end{center}
    \vspace{0.5em}
    \hrule
    \vspace{0.5em}
}{\vspace{0.5em}\hrule}

%=== 文档元信息 ===

\title{A Policy-Driven Modeling Framework for \\
       \TODO{Problem F Topic: e.g., Resource Allocation / Social Intervention / Regulatory Design}}
\author{Team \#2617892}

%=== 正文开始 ===
\begin{document}

% ============================================
% PART 0: Summary Sheet (摘要页,不计入页数)
% ============================================
\pagenumbering{arabic}
\setcounter{page}{0}
\pagestyle{empty}
\begin{abstract}%以下是25年C题的参考Abstract
    
To ensure a systematic and multidimensional approach to Olympic medal prediction, this study constructs a dynamic and coupled comprehensive \textbf{Predictive Framework}. This framework integrates various objectives, including medal distribution prediction, breakthrough country identification, event performance evaluation, and the impact analysis of coaching resource allocation. By employing multi-source heterogeneous data fusion and machine learning integration methods, we achieve an in-depth analysis and reliable prediction of the development patterns in competitive sports.

First, we develop a multidimensional predictive model to analyze Olympic medal patterns. Utilizing \textbf{Principal Component Analysis (PCA)} for dimensionality reduction and noise filtering of the Olympic event dataset, we combine \textbf{Long Short-Term Memory (LSTM)} networks to mine temporal features and integrate home advantage effects. A dual-channel \textbf{XGBoost-Bootstrap} model is established to generate predictions with a 95\% confidence interval. The results indicate an upward trend for countries such as the United States and the United Kingdom, while countries like France and China show a decline. The model exhibites a high accuracy with a low \textbf{MAE} and \textbf{MAPE}, demonstrating strong robustness. Through non-parametric testing, potential breakthrough countries are identified, with San Marino and Kuwait showing gold medal breakthrough probabilities of \textbf{84.7\%} and \textbf{68.4\%}, respectively.

Subsequently, we develop a \textbf{Difference-in-Differences (DID)} model to quantify the competitive benefits of coaching replacement and conduct statistical significance tests as well as parallel trends tests to ensure the reliability of our results. It is found that during the 2020-2024 period, Australia, South Korea, and Poland experienced significant "great coach" policy effects through strategic restructuring of their coaching teams. Based on this, we recommend prioritizing investment in high-elasticity projects. \textbf{SHapley Additive exPlanations (SHAP)} is further utilized to quantify event contributions, revealing that swimming and athletics serve as core contributing events to medal augmentation.

Furthermore, we explore the potential of a country to win its first-ever medal by constructing a \textbf{Hurdle-Tobit} fusion model. This model addresses the zero-inflation characteristics and heterogeneity between countries. The prediction results show that countries like Angola and Bangladesh have a probability of winning their first medal exceeding \textbf{45\%} in the next Games. Finally, we propose a strategic resource allocation plan using \textbf{Multiobjective Optimization} to balance the "depth" and "breadth" of National Olympic Committees (NOCs), suggesting that high-potential NOCs should prioritize multinational coach introductions in wrestling and table tennis.

In conclusion, our integrated framework synthesizes all models and analyses to present new insights and corresponding decision supports for the global sports community.

\begin{keywords}
    Olympic Medal Prediction, PCA-LSTM, XGBoost, DID Model, Hurdle-Tobit, SHAP Analysis.
\end{keywords}

\end{abstract}

\maketitle
\thispagestyle{empty}

% ============================================
% PART 0.5: Policy Memorandum (政策备忘录)
% ============================================
\newpage
\thispagestyle{empty}
% 00_memo.tex - Policy Memorandum (政策备忘录)
% ICM Problem F 核心特色:面向决策者的正式单页摘要

% 【写作指导】
% 1. 本页是F题的标志性产出,评审重点关注
% 2. 语言应专业但可被非技术背景决策者理解
% 3. 控制在1页以内,突出要点
% 4. 使用项目符号和加粗强调关键信息

\section*{Policy Memorandum}
\addcontentsline{toc}{section}{Policy Memorandum}
\label{sec:memo}

\begin{memorandum}

\noindent
\begin{tabular}{@{}l l}
\textbf{To:} & \TODO{Target Decision-Maker, e.g., Director of Policy Planning, Minister of XX} \\[0.3em]
\textbf{From:} & Team \#2617892, ICM 2026 Modeling Analysts \\[0.3em]
\textbf{Date:} & \today \\[0.3em]
\textbf{Subject:} & \TODO{Concise Policy Title: e.g., ``Optimizing Resource Allocation for [Topic]''}
\end{tabular}

\vspace{1em}

%=== 执行摘要 ===
\subsection*{Executive Summary}

\TODO{2-3 sentences summarizing the policy problem and your recommended solution. Example: ``Our analysis indicates that implementing [Policy X] will achieve [Outcome Y] with [Z\%] probability of success. We recommend a phased rollout beginning with [Region/Sector] to maximize impact while managing implementation risks.''}

%=== 政策要点 ===
\subsection*{Key Policy Recommendations}

\begin{enumerate}[leftmargin=*, label=\textbf{\arabic*.}]
    \item \textbf{\TODO{Recommendation 1 Title}}
    
    \TODO{Brief description of the first recommendation. Include: what action, who implements, expected outcome.}
    
    \item \textbf{\TODO{Recommendation 2 Title}}
    
    \TODO{Brief description of the second recommendation.}
    
    \item \textbf{\TODO{Recommendation 3 Title}}
    
    \TODO{Brief description of the third recommendation.}
\end{enumerate}

%=== 预期成效 ===
\subsection*{Expected Outcomes}

\begin{itemize}[leftmargin=*]
    \item \textbf{Primary Impact:} \TODO{e.g., ``15-20\% improvement in [metric] within [timeframe]''}
    \item \textbf{Secondary Benefits:} \TODO{e.g., ``Enhanced equity across [demographic groups]''}
    \item \textbf{Long-term Value:} \TODO{e.g., ``Sustainable system stability over [X]-year horizon''}
\end{itemize}

%=== 风险与缓解 ===
\subsection*{Risks and Mitigation}

\begin{itemize}[leftmargin=*]
    \item \textbf{Implementation Risk:} \TODO{Brief risk description} $\rightarrow$ \textit{Mitigation:} \TODO{mitigation strategy}
    \item \textbf{Stakeholder Resistance:} \TODO{Brief risk description} $\rightarrow$ \textit{Mitigation:} \TODO{mitigation strategy}
    \item \textbf{External Shocks:} \TODO{Brief risk description} $\rightarrow$ \textit{Mitigation:} \TODO{mitigation strategy}
\end{itemize}

%=== 资源需求 ===
\subsection*{Resource Requirements}

\begin{table}[H]
\centering
\small
\begin{tabular}{l l l}
\toprule
\textbf{Resource Category} & \textbf{Estimated Requirement} & \textbf{Timeline} \\
\midrule
Financial Investment & \TODO{\$X million} & \TODO{Years 1-3} \\
Human Capital & \TODO{X FTEs} & \TODO{Ongoing} \\
Infrastructure & \TODO{Description} & \TODO{Year 1} \\
\bottomrule
\end{tabular}
\end{table}

%=== 下一步行动 ===
\subsection*{Recommended Next Steps}

\begin{enumerate}[leftmargin=*, label=\textbf{Step \arabic*:}]
    \item \TODO{Immediate action (0-3 months): e.g., ``Establish cross-departmental task force''}
    \item \TODO{Short-term action (3-12 months): e.g., ``Launch pilot program in [region]''}
    \item \TODO{Medium-term action (1-3 years): e.g., ``Scale successful interventions nationwide''}
\end{enumerate}

\vspace{0.5em}
\noindent\textit{For detailed technical analysis supporting these recommendations, please refer to the full report.}

\end{memorandum}


% ============================================
% Table of Contents (目录页)
% ============================================
\newpage
\setcounter{page}{1}
\pagestyle{empty}
\tableofcontents
\thispagestyle{empty}

% ============================================
% 正文开始(从第3页开始显示页眉页码)
% ============================================
\newpage
\setcounter{page}{3}
\pagestyle{fancy}
\fancyhf{}
\fancyhead[L]{\small Team \#2617892}
\fancyhead[R]{\small Page \thepage\ of \themainpages}
\fancyfoot[C]{}

%=== PART I: Policy Context & Problem Framing ===

% Section 1: Introduction (政策背景与问题重述)
% 01_introduction.tex - Introduction (引言)
% ICM Problem F - 政策背景与问题重述

\section{Introduction}
\label{sec:intro}

% 【写作指导】F题引言需建立政策问题的紧迫性和重要性
% 结构:政策背景 → 问题痛点 → 文献缺口 → 本文贡献

%=== 1.1 政策背景与问题陈述 ===
\subsection{Policy Background and Problem Statement}
\label{sec:intro_background}

% 背景段落1:宏观政策情境
\TODO{Describe the broader policy context. Example: ``In recent years, governments worldwide have faced increasing pressure to address [policy challenge]. The [specific domain, e.g., healthcare, education, environment] sector has experienced significant shifts due to [driving factors], creating both opportunities and challenges for policymakers.''}

% 背景段落2:问题具体化
The specific challenge we address is \TODO{concise problem statement in 1-2 sentences}. This problem manifests across multiple dimensions:

\begin{itemize}
    \item \textbf{Scale:} \TODO{e.g., ``Affecting X million individuals across Y regions''}
    \item \textbf{Urgency:} \TODO{e.g., ``Current trajectory suggests Z\% degradation by [year]''}
    \item \textbf{Complexity:} \TODO{e.g., ``Involves trade-offs between competing stakeholder interests''}
\end{itemize}

% 背景段落3:量化问题严重性
To quantify the severity of this challenge, consider the following statistics:
% TODO: 插入关键统计数据或引用

\begin{figure}[H]
    \centering
    % \includegraphics[width=0.85\textwidth]{figures/problem_overview.pdf}
    \fbox{\parbox{0.8\textwidth}{\centering\vspace{3em}\TODO{Figure 1: Problem Overview Infographic}\vspace{3em}}}
    \caption{Overview of the policy challenge: \TODO{caption description}}
    \label{fig:problem_overview}
\end{figure}

%=== 1.2 文献回顾与研究缺口 ===
\subsection{Literature Review and Research Gap}
\label{sec:intro_literature}

% 文献分类综述
Previous research on \TODO{topic} can be categorized into three main streams:

\textbf{Stream 1: \TODO{Category Name, e.g., Theoretical Foundations}}

\TODO{2-3 sentences summarizing key works in this category, with citations like \cite{example2023policy}. Focus on what has been established.}

\textbf{Stream 2: \TODO{Category Name, e.g., Empirical Evidence}}

\TODO{2-3 sentences summarizing empirical studies. Highlight methodologies used (RCTs, quasi-experiments, observational studies).}

\textbf{Stream 3: \TODO{Category Name, e.g., Policy Modeling Approaches}}

\TODO{2-3 sentences on existing modeling approaches (system dynamics, agent-based, optimization). Identify their strengths and limitations.}

\vspace{0.5em}
\noindent\textbf{Research Gap.} Despite these contributions, the literature exhibits several gaps:

\begin{enumerate}[label=(\arabic*)]
    \item \TODO{Gap 1: e.g., ``Limited integration of mechanism modeling with policy optimization''}
    \item \TODO{Gap 2: e.g., ``Insufficient attention to distributional impacts and equity''}
    \item \TODO{Gap 3: e.g., ``Lack of robustness analysis under uncertainty''}
\end{enumerate}

%=== 1.3 本文贡献 ===
\subsection{Our Contributions}
\label{sec:intro_contributions}

This paper addresses the identified gaps through a \textbf{policy-driven modeling framework} that integrates mechanism analysis, optimization, and robust evaluation. Our specific contributions include:

\begin{enumerate}[label=\textbf{C\arabic*:}, leftmargin=2.5em]
    \item \textbf{System Mechanism Model.} We develop a \TODO{e.g., system dynamics / causal inference} model that captures the pathways through which policy interventions generate outcomes, enabling scenario analysis across varying policy intensities (\secref{sec:model1}).
    
    \item \textbf{Policy Optimization Framework.} We formulate a \TODO{e.g., multi-objective / robust} optimization problem that balances \TODO{competing objectives} subject to practical constraints, yielding Pareto-optimal policy portfolios (\secref{sec:model2}).
    
    \item \textbf{Spillover and Stability Analysis.} Beyond direct impacts, we analyze how policies propagate through interconnected systems, assessing \textbf{cross-domain spillovers} and \textbf{long-term stability} (\secref{sec:impacts}).
    
    \item \textbf{Actionable Policy Recommendations.} We translate quantitative findings into a structured \textbf{Policy Memorandum} with tiered, implementable recommendations aligned with stakeholder capacities.
\end{enumerate}

%=== 1.4 论文结构 ===
\subsection{Paper Organization}
\label{sec:intro_organization}

The remainder of this paper is organized as follows:

\begin{itemize}[leftmargin=*]
    \item \textbf{\secref{sec:assumptions}:} Defines the policy scope, formalizes assumptions, and establishes analytical boundaries.
    \item \textbf{\secref{sec:data}:} Describes data sources, preprocessing, and the notation system.
    \item \textbf{\secref{sec:model1}:} Develops the system mechanism model capturing causal pathways.
    \item \textbf{\secref{sec:model2}:} Formulates the policy optimization problem and solution approach.
    \item \textbf{\secref{sec:algorithm}:} Presents the computational algorithm with complexity analysis.
    \item \textbf{\secref{sec:results}:} Reports results and validates model predictions.
    \item \textbf{\secref{sec:sensitivity}:} Conducts sensitivity and robustness analysis.
    \item \textbf{\secref{sec:impacts}:} Discusses policy impacts, spillovers, and implementation considerations.
    \item \textbf{\secref{sec:strengths}:} Evaluates model strengths and limitations.
    \item \textbf{\secref{sec:conclusion}:} Summarizes findings and outlines future directions.
\end{itemize}


% Section 2: Project Definition and Assumptions (项目定义与假设)
% 02_project_and_assumptions.tex - 项目定义与假设
% ICM Problem F - Policy Science

\section{Project Definition and Assumptions}
\label{sec:assumptions}

% 【写作指导】F题需要明确定义政策问题的边界、利益相关者、约束条件
% 假设应分层次:系统假设、行为假设、数据假设

%=== 2.1 政策问题界定 ===
\subsection{Policy Problem Definition}
\label{sec:assumptions_definition}

% 问题范围
\textbf{Policy Scope.} We define the policy problem within the following boundaries:

\begin{itemize}
    \item \textbf{Spatial Scope:} \TODO{e.g., ``National level with regional disaggregation'' or ``Urban metropolitan areas''}
    \item \textbf{Temporal Scope:} \TODO{e.g., ``Planning horizon of 10 years (2025-2035) with annual decision intervals''}
    \item \textbf{Sectoral Scope:} \TODO{e.g., ``Primary focus on [sector], with linkages to [related sectors]''}
\end{itemize}

% 利益相关者
\textbf{Stakeholder Identification.} Effective policy analysis requires explicit recognition of affected parties:

\begin{table}[H]
\centering
\caption{Stakeholder Analysis Matrix}
\label{tab:stakeholders}
\begin{tabular}{p{3cm} p{4cm} p{4cm} p{3cm}}
\toprule
\textbf{Stakeholder Group} & \textbf{Primary Interest} & \textbf{Influence Level} & \textbf{Model Representation} \\
\midrule
\TODO{Group 1, e.g., Government} & \TODO{e.g., Efficiency, equity} & \TODO{High/Medium/Low} & \TODO{e.g., Decision-maker} \\
\TODO{Group 2, e.g., Citizens} & \TODO{e.g., Service quality} & Medium & \TODO{e.g., Utility function} \\
\TODO{Group 3, e.g., Industry} & \TODO{e.g., Profitability} & \TODO{Level} & \TODO{e.g., Constraint set} \\
\TODO{Group 4} & \TODO{Interest} & \TODO{Level} & \TODO{Representation} \\
\bottomrule
\end{tabular}
\end{table}

% 政策工具
\textbf{Policy Instruments.} The decision-maker has access to the following intervention levers:

\begin{enumerate}[label=\textbf{P\arabic*:}]
    \item \textbf{Regulatory Tools:} \TODO{e.g., ``Standards, mandates, restrictions''}
    \item \textbf{Economic Incentives:} \TODO{e.g., ``Subsidies, taxes, pricing mechanisms''}
    \item \textbf{Information-Based:} \TODO{e.g., ``Disclosure requirements, awareness campaigns''}
    \item \textbf{Direct Provision:} \TODO{e.g., ``Public investment, service delivery''}
\end{enumerate}

%=== 2.2 核心假设 ===
\subsection{Core Assumptions}
\label{sec:assumptions_core}

We organize our assumptions into three categories, each justified by empirical evidence or standard practice in the policy literature.

%--- 2.2.1 系统结构假设 ---
\subsubsection{System Structure Assumptions}

\begin{assumption}[System Boundaries]
\label{asmp:boundaries}
The policy system can be meaningfully bounded to include \TODO{list of included elements} while treating \TODO{excluded elements} as exogenous factors.
\end{assumption}

\noindent\textit{Justification:} \TODO{Explain why this boundary is appropriate for the problem.}

\begin{assumption}[Causal Structure]
\label{asmp:causality}
The causal relationships between policy inputs and outcomes follow the directed acyclic graph (DAG) structure depicted in \figref{fig:causal_dag}, with no unmeasured confounders between key variables.
\end{assumption}

\begin{figure}[H]
    \centering
    \begin{tikzpicture}[
        node distance=2cm,
        box/.style={rectangle, draw, minimum width=2cm, minimum height=0.8cm, align=center},
        arrow/.style={->, >=stealth, thick}
    ]
        % 节点
        \node[box, fill=policyblue!20] (policy) {Policy\\Intervention};
        \node[box, right=of policy, fill=neutralgray!20] (mechanism) {Mechanism\\Variables};
        \node[box, right=of mechanism, fill=resultgreen!20] (outcome) {Policy\\Outcomes};
        \node[box, below=1cm of mechanism, fill=warningorange!20] (confound) {Contextual\\Factors};
        
        % 箭头
        \draw[arrow] (policy) -- (mechanism);
        \draw[arrow] (mechanism) -- (outcome);
        \draw[arrow] (confound) -- (mechanism);
        \draw[arrow] (confound) -- (outcome);
    \end{tikzpicture}
    \caption{Simplified causal DAG of the policy system}
    \label{fig:causal_dag}
\end{figure}

%--- 2.2.2 行为假设 ---
\subsubsection{Behavioral Assumptions}

\begin{assumption}[Rational Bounded Response]
\label{asmp:behavior}
Stakeholders respond to policy signals in a \textit{bounded-rational} manner: they seek to improve their utility but face information constraints and adjustment costs, leading to gradual rather than instantaneous responses.
\end{assumption}

\noindent\textit{Justification:} Consistent with behavioral economics literature \cite{kahneman2011thinking} and empirical observations of policy lag effects.

\begin{assumption}[Compliance Heterogeneity]
\label{asmp:compliance}
Compliance with policy mandates varies across stakeholder groups, with compliance rate $\rho_i \in [0.6, 0.95]$ for group $i$, estimated from \TODO{data source}.
\end{assumption}

%--- 2.2.3 数据与技术假设 ---
\subsubsection{Data and Technical Assumptions}

\begin{assumption}[Data Quality]
\label{asmp:data}
Available data accurately represents population characteristics with measurement error bounded by \TODO{X\%}, and missing data mechanisms are at most \textit{missing at random} (MAR).
\end{assumption}

\begin{assumption}[Parameter Stability]
\label{asmp:stability}
Model parameters estimated from historical data remain valid for the planning horizon, absent structural breaks. We test this assumption via sensitivity analysis in \secref{sec:sensitivity}.
\end{assumption}

%=== 2.3 假设汇总表 ===
\subsection{Summary of Assumptions}
\label{sec:assumptions_summary}

\begin{table}[H]
\centering
\caption{Summary of Model Assumptions}
\label{tab:assumptions_summary}
\small
\begin{tabular}{c l p{6cm} c}
\toprule
\textbf{ID} & \textbf{Category} & \textbf{Statement} & \textbf{Tested?} \\
\midrule
A1 & System & \TODO{Brief statement} & \secref{sec:sensitivity} \\
A2 & System & \TODO{Brief statement} & -- \\
A3 & Behavioral & Bounded-rational stakeholder response & \secref{sec:sensitivity} \\
A4 & Behavioral & \TODO{Brief statement} & \secref{sec:results} \\
A5 & Data & Data quality bounds & -- \\
A6 & Technical & Parameter stability & \secref{sec:sensitivity} \\
\bottomrule
\end{tabular}
\end{table}


% Section 3: Data and Notation (数据与符号说明)
% 03_data_and_notation.tex - 数据与符号说明
% ICM Problem F - Policy Science

\section{Data Description and Notation}
\label{sec:data}

% 【写作指导】F题数据部分需要:
% 1. 清晰说明数据来源的权威性(政府统计、学术数据库等)
% 2. 展示数据预处理流程
% 3. 建立统一的符号体系

%=== 3.1 数据来源 ===
\subsection{Data Sources}
\label{sec:data_sources}

Our analysis draws on multiple authoritative data sources to ensure reliability and cross-validation:

\begin{table}[H]
\centering
\caption{Primary Data Sources}
\label{tab:data_sources}
\small
\begin{tabular}{p{3cm} p{4cm} p{2.5cm} p{2.5cm} p{2cm}}
\toprule
\textbf{Dataset} & \textbf{Description} & \textbf{Source} & \textbf{Coverage} & \textbf{Granularity} \\
\midrule
\TODO{Dataset 1} & \TODO{e.g., Population demographics} & \TODO{e.g., Census Bureau} & \TODO{2015-2024} & \TODO{Regional} \\
\TODO{Dataset 2} & \TODO{e.g., Policy implementation records} & \TODO{e.g., Government reports} & \TODO{Coverage} & \TODO{Level} \\
\TODO{Dataset 3} & \TODO{e.g., Economic indicators} & \TODO{e.g., World Bank} & \TODO{Coverage} & \TODO{Level} \\
\TODO{Dataset 4} & \TODO{e.g., Survey data} & \TODO{e.g., Academic study} & \TODO{Coverage} & \TODO{Level} \\
\bottomrule
\end{tabular}
\end{table}

% 数据质量评估
\textbf{Data Quality Assessment.} We evaluate each source against the following criteria:

\begin{itemize}
    \item \textbf{Completeness:} \TODO{e.g., ``Dataset 1 has <2\% missing values after preprocessing''}
    \item \textbf{Consistency:} \TODO{e.g., ``Cross-validated against alternative sources''}
    \item \textbf{Timeliness:} \TODO{e.g., ``Most recent data from 2023''}
    \item \textbf{Relevance:} \TODO{e.g., ``Directly measures target outcomes''}
\end{itemize}

%=== 3.2 数据预处理 ===
\subsection{Data Preprocessing}
\label{sec:data_preprocessing}

We apply a standardized preprocessing pipeline to ensure data quality and compatibility:

\begin{figure}[H]
    \centering
    \begin{tikzpicture}[
        node distance=1.5cm,
        box/.style={rectangle, draw, rounded corners, minimum width=2.5cm, minimum height=0.8cm, align=center, font=\small},
        arrow/.style={->, >=stealth, thick}
    ]
        % 流程节点
        \node[box, fill=blue!10] (raw) {Raw Data\\Collection};
        \node[box, fill=orange!10, right=of raw] (clean) {Missing Value\\Imputation};
        \node[box, fill=green!10, right=of clean] (normal) {Normalization\\Standardization};
        \node[box, fill=purple!10, right=of normal] (feature) {Feature\\Engineering};
        \node[box, fill=red!10, right=of feature] (final) {Analysis-Ready\\Dataset};
        
        % 箭头
        \draw[arrow] (raw) -- (clean);
        \draw[arrow] (clean) -- (normal);
        \draw[arrow] (normal) -- (feature);
        \draw[arrow] (feature) -- (final);
    \end{tikzpicture}
    \caption{Data preprocessing pipeline}
    \label{fig:preprocessing}
\end{figure}

\textbf{Key Preprocessing Steps:}

\begin{enumerate}
    \item \textbf{Missing Value Treatment:}
    \begin{itemize}
        \item Continuous variables: \TODO{e.g., Multiple imputation using chained equations (MICE)}
        \item Categorical variables: \TODO{e.g., Mode imputation with uncertainty flags}
    \end{itemize}
    
    \item \textbf{Outlier Detection:}
    \begin{itemize}
        \item Method: \TODO{e.g., IQR-based detection with domain expert validation}
        \item Treatment: \TODO{e.g., Winsorization at 1st/99th percentiles}
    \end{itemize}
    
    \item \textbf{Variable Transformation:}
    \begin{itemize}
        \item Skewed distributions: \TODO{e.g., Log transformation for income variables}
        \item Scale standardization: \TODO{e.g., Z-score normalization for comparability}
    \end{itemize}
\end{enumerate}

%=== 3.3 探索性数据分析 ===
\subsection{Exploratory Data Analysis}
\label{sec:data_eda}

% 描述性统计
\textbf{Summary Statistics.} \tabref{tab:descriptive} presents descriptive statistics for key variables:

\begin{table}[H]
\centering
\caption{Descriptive Statistics of Key Variables}
\label{tab:descriptive}
\small
\begin{tabular}{l c c c c c c}
\toprule
\textbf{Variable} & \textbf{N} & \textbf{Mean} & \textbf{Std Dev} & \textbf{Min} & \textbf{Max} & \textbf{Unit} \\
\midrule
\TODO{Variable 1} & \TODO{N} & \TODO{Mean} & \TODO{SD} & \TODO{Min} & \TODO{Max} & \TODO{Unit} \\
\TODO{Variable 2} & \TODO{N} & \TODO{Mean} & \TODO{SD} & \TODO{Min} & \TODO{Max} & \TODO{Unit} \\
\TODO{Variable 3} & \TODO{N} & \TODO{Mean} & \TODO{SD} & \TODO{Min} & \TODO{Max} & \TODO{Unit} \\
\TODO{Variable 4} & \TODO{N} & \TODO{Mean} & \TODO{SD} & \TODO{Min} & \TODO{Max} & \TODO{Unit} \\
\bottomrule
\end{tabular}
\end{table}

% 相关性分析
\textbf{Correlation Analysis.} We examine pairwise correlations to identify potential multicollinearity and causal pathway candidates:

\begin{figure}[H]
    \centering
    % \includegraphics[width=0.6\textwidth]{figures/correlation_heatmap.pdf}
    \fbox{\parbox{0.55\textwidth}{\centering\vspace{4em}\TODO{Figure: Correlation Heatmap}\vspace{4em}}}
    \caption{Correlation matrix of policy-relevant variables}
    \label{fig:correlation}
\end{figure}

%=== 3.4 符号与约定 ===
\subsection{Notation and Conventions}
\label{sec:data_notation}

To facilitate clear communication, we establish the following notation conventions:

\begin{table}[H]
\centering
\caption{Notation System}
\label{tab:notation}
\begin{tabular}{c l p{8cm}}
\toprule
\textbf{Symbol} & \textbf{Type} & \textbf{Description} \\
\midrule
\multicolumn{3}{l}{\textit{Index Sets}} \\
$i \in \mathcal{I}$ & Set & Set of regions/units, $|\mathcal{I}| = I$ \\
$t \in \mathcal{T}$ & Set & Time periods, $\mathcal{T} = \{1, 2, \ldots, T\}$ \\
$k \in \mathcal{K}$ & Set & Stakeholder groups \\
\midrule
\multicolumn{3}{l}{\textit{Decision Variables}} \\
$\policy_{i,t}$ & Variable & Policy intensity for unit $i$ at time $t$ \\
$x_{i,t}$ & Variable & Allocation decision \\
\midrule
\multicolumn{3}{l}{\textit{State Variables}} \\
$\stock_{i,t}$ & Variable & System state (stock) for unit $i$ at time $t$ \\
$\outcome_{i,t}$ & Variable & Outcome measure \\
\midrule
\multicolumn{3}{l}{\textit{Parameters}} \\
$\alpha, \beta$ & Parameter & Model coefficients (estimated) \\
$\lambda$ & Parameter & Feedback strength \\
$\discount$ & Parameter & Discount factor, $\discount \in (0,1)$ \\
\midrule
\multicolumn{3}{l}{\textit{Functions}} \\
$\utility_k(\cdot)$ & Function & Utility function of stakeholder $k$ \\
$\objective(\cdot)$ & Function & Social welfare objective \\
$f(\cdot), g(\cdot)$ & Function & System dynamics functions \\
\bottomrule
\end{tabular}
\end{table}

% 约定
\textbf{Conventions:}
\begin{itemize}
    \item Boldface denotes vectors and matrices: $\mathbf{x} = (x_1, \ldots, x_n)^\top$
    \item Subscripts denote indices; superscripts denote iterations or optimality ($x^*$)
    \item Time indices follow the convention: $t=0$ is baseline, $t=1$ is first policy period
    \item All monetary values are in \TODO{currency, e.g., 2023 USD} unless otherwise noted
\end{itemize}


%=== PART II: Mechanism & Policy Modeling ===

% Section 4: System Mechanism Model (系统机制模型)
% 04_model1_mechanism.tex - 系统机制模型
% ICM Problem F - Policy Science

\section{Model I: System Mechanism Analysis}
\label{sec:model1}

% 【写作指导】F题机制模型需要:
% 1. 明确政策干预如何传导至最终结果
% 2. 识别关键中介变量和反馈回路
% 3. 支持情景分析和政策模拟

%=== 4.1 模型概述 ===
\subsection{Model Overview}
\label{sec:model1_overview}

To understand \textit{how} policy interventions translate into outcomes, we develop a \textbf{\TODO{e.g., System Dynamics / Causal Inference / Structural}} model that captures the mechanism pathways linking policy inputs to societal impacts.

\textbf{Modeling Philosophy.} Our approach is guided by three principles:

\begin{enumerate}
    \item \textbf{Transparency:} All causal relationships are explicitly stated and empirically grounded.
    \item \textbf{Modularity:} Sub-models can be updated independently as new evidence emerges.
    \item \textbf{Policy Relevance:} Model outputs directly inform actionable policy levers.
\end{enumerate}

% 系统边界图
\begin{figure}[H]
    \centering
    \begin{tikzpicture}[
        scale=0.9,
        transform shape,
        node distance=2cm,
        stock/.style={rectangle, draw, thick, minimum width=2.5cm, minimum height=1cm, fill=blue!10, align=center},
        flow/.style={rectangle, draw, minimum width=2cm, minimum height=0.6cm, fill=green!10, align=center},
        cloud/.style={ellipse, draw, dashed, minimum width=2cm, fill=gray!10, align=center},
        arrow/.style={->, >=stealth, thick},
        feedback/.style={->, >=stealth, thick, dashed, red}
    ]
        % 外部输入
        \node[cloud] (external) {External\\Factors};
        
        % 政策干预
        \node[flow, below=of external, fill=policyblue!30] (policy) {Policy\\Intervention};
        
        % 中介变量
        \node[stock, below left=1.5cm and 1cm of policy] (mech1) {\TODO{Mechanism 1}\\Stock};
        \node[stock, below right=1.5cm and 1cm of policy] (mech2) {\TODO{Mechanism 2}\\Stock};
        
        % 结果
        \node[stock, below=3.5cm of policy, fill=resultgreen!30] (outcome) {Policy\\Outcome};
        
        % 箭头
        \draw[arrow] (external) -- (policy);
        \draw[arrow] (policy) -- (mech1);
        \draw[arrow] (policy) -- (mech2);
        \draw[arrow] (mech1) -- (outcome);
        \draw[arrow] (mech2) -- (outcome);
        
        % 反馈
        \draw[feedback] (outcome.west) to[out=180, in=270] (mech1.south);
        
        % 标注
        \node[right=0.3cm of mech2, font=\footnotesize] {Direct Effect};
        \node[left=0.3cm of mech1, font=\footnotesize, red] {Feedback};
    \end{tikzpicture}
    \caption{System structure of the mechanism model}
    \label{fig:system_structure}
\end{figure}

%=== 4.2 核心方程 ===
\subsection{Core Equations}
\label{sec:model1_equations}

%--- 4.2.1 状态演化方程 ---
\subsubsection{State Evolution Dynamics}

The system state evolves according to the following difference equations:

% 状态方程1
\begin{equation}
\stock_{i,t+1} = \stock_{i,t} + \Delta t \cdot \left[ \flow^{in}_{i,t} - \flow^{out}_{i,t} \right]
\label{eq:state_evolution}
\end{equation}

where:
\begin{itemize}
    \item $\stock_{i,t}$: System state (stock) for unit $i$ at time $t$
    \item $\flow^{in}_{i,t}$: Inflow rate, determined by policy and external factors
    \item $\flow^{out}_{i,t}$: Outflow rate, reflecting natural depletion or consumption
\end{itemize}

% 流量方程
The inflow rate is modeled as a function of policy intensity:
\begin{equation}
\flow^{in}_{i,t} = \alpha_0 + \alpha_1 \cdot \policy_{i,t} + \alpha_2 \cdot \policy_{i,t}^2 + \beta \cdot \mathbf{X}_{i,t}
\label{eq:inflow}
\end{equation}
% TODO: 根据具体问题调整函数形式

The quadratic term $\alpha_2 \cdot \policy_{i,t}^2$ captures \textit{diminishing returns} or \textit{threshold effects} common in policy applications.

%--- 4.2.2 反馈机制 ---
\subsubsection{Feedback Mechanisms}

We identify \TODO{N} key feedback loops that shape system behavior:

\textbf{Feedback Loop 1: \TODO{Name, e.g., Reinforcing Growth}}

The first feedback operates through \TODO{mechanism description}:
% TODO: 添加反馈方程
\begin{equation}
\feedback_1 = \lambda_1 \cdot f(\stock_{t-\tau}) \cdot g(\outcome_{t-\tau})
\label{eq:feedback1}
\end{equation}

where $\tau$ represents the policy lag (time delay before effects materialize).

\textbf{Feedback Loop 2: \TODO{Name, e.g., Balancing Constraint}}

\TODO{Description and equation for the second feedback loop}

%--- 4.2.3 结果映射 ---
\subsubsection{Outcome Mapping}

Final policy outcomes are derived from system states through the mapping function:
\begin{equation}
\outcome_{i,t} = h(\stock_{i,t}, \mathbf{Z}_{i,t}; \boldsymbol{\theta})
\label{eq:outcome}
\end{equation}

where $\mathbf{Z}_{i,t}$ represents contextual moderators and $\boldsymbol{\theta}$ is the parameter vector estimated from data.

%=== 4.3 参数估计 ===
\subsection{Parameter Estimation}
\label{sec:model1_estimation}

Model parameters are estimated using a combination of:

\begin{enumerate}
    \item \textbf{Literature Calibration:} Parameters with established empirical estimates are sourced from peer-reviewed studies (see \tabref{tab:param_sources}).
    
    \item \textbf{Statistical Estimation:} For novel parameters, we use \TODO{e.g., maximum likelihood estimation / Bayesian inference / system identification} on the dataset described in \secref{sec:data}.
    
    \item \textbf{Expert Elicitation:} Where data is sparse, we conduct structured expert consultation following \TODO{protocol, e.g., Delphi method}.
\end{enumerate}

\begin{table}[H]
\centering
\caption{Parameter Estimates and Sources}
\label{tab:param_sources}
\small
\begin{tabular}{l c c c l}
\toprule
\textbf{Parameter} & \textbf{Estimate} & \textbf{95\% CI} & \textbf{Method} & \textbf{Source} \\
\midrule
$\alpha_1$ & \TODO{value} & \TODO{[L, U]} & \TODO{MLE} & This study \\
$\alpha_2$ & \TODO{value} & \TODO{[L, U]} & Literature & \cite{example2023policy} \\
$\lambda_1$ & \TODO{value} & \TODO{[L, U]} & Expert & Panel consensus \\
$\tau$ (lag) & \TODO{value} & \TODO{[L, U]} & Empirical & \TODO{source} \\
\bottomrule
\end{tabular}
\end{table}

%=== 4.4 模型验证 ===
\subsection{Model Validation}
\label{sec:model1_validation}

We validate the mechanism model through:

\begin{enumerate}
    \item \textbf{Structural Validity:} Expert review confirms that causal pathways align with domain knowledge.
    
    \item \textbf{Behavioral Validity:} Simulated system behavior reproduces known historical patterns (see \figref{fig:validation}).
    
    \item \textbf{Extreme Condition Testing:} Model behaves plausibly under boundary conditions (e.g., zero policy, maximum policy).
\end{enumerate}

\begin{figure}[H]
    \centering
    % \includegraphics[width=0.8\textwidth]{figures/model_validation.pdf}
    \fbox{\parbox{0.75\textwidth}{\centering\vspace{4em}\TODO{Figure: Model Validation - Historical vs. Simulated}\vspace{4em}}}
    \caption{Model validation: Comparison of historical data (solid) with model simulation (dashed)}
    \label{fig:validation}
\end{figure}


% Section 5: Policy Optimization Model (政策优化模型)
% 05_model2_policy.tex - 政策优化模型
% ICM Problem F - Policy Science

\section{Model II: Policy Optimization Framework}
\label{sec:model2}

% 【写作指导】F题政策优化需要:
% 1. 多目标平衡(效率、公平、可持续性)
% 2. 现实约束(预算、容量、政治可行性)
% 3. 鲁棒性考量(不确定性下的决策)

%=== 5.1 优化问题构建 ===
\subsection{Problem Formulation}
\label{sec:model2_formulation}

Building on the mechanism model (\secref{sec:model1}), we formulate a \textbf{multi-objective optimization} problem that identifies Pareto-optimal policy portfolios.

%--- 5.1.1 目标函数 ---
\subsubsection{Objective Functions}

Effective policy must balance multiple, potentially conflicting objectives. We consider:

\textbf{Objective 1: \TODO{e.g., Social Welfare Maximization}}
\begin{equation}
\objective_1 = \max_{\boldsymbol{\policy}} \sum_{t=1}^{T} \discount^{t-1} \cdot \welfare(\outcome_t, \mathbf{X}_t)
\label{eq:obj1}
\end{equation}

where $\welfare(\cdot)$ is the social welfare function aggregating individual utilities, and $\discount$ is the social discount factor.

\textbf{Objective 2: \TODO{e.g., Equity / Distributional Fairness}}
\begin{equation}
\objective_2 = \min_{\boldsymbol{\policy}} \; \text{Gini}(\outcome_{i,T}) = 1 - 2 \int_0^1 L(p) \, dp
\label{eq:obj2}
\end{equation}

where $L(p)$ is the Lorenz curve of outcomes across units.

\textbf{Objective 3: \TODO{e.g., Cost-Effectiveness / Budget Efficiency}}
\begin{equation}
\objective_3 = \min_{\boldsymbol{\policy}} \sum_{i,t} c_{i,t} \cdot \policy_{i,t}
\label{eq:obj3}
\end{equation}

%--- 5.1.2 约束条件 ---
\subsubsection{Constraints}

Policy decisions are subject to real-world constraints:

\textbf{Budget Constraint:}
\begin{equation}
\sum_{i \in \mathcal{I}} \sum_{t=1}^{T} c_{i,t} \cdot \policy_{i,t} \leq B_{\text{total}}
\label{eq:budget}
\end{equation}

\textbf{Capacity Constraint:}
\begin{equation}
\policy_{i,t} \leq \bar{\policy}_i \quad \forall i \in \mathcal{I}, \; t \in \mathcal{T}
\label{eq:capacity}
\end{equation}

\textbf{Temporal Smoothness (avoid drastic policy changes):}
\begin{equation}
|\policy_{i,t+1} - \policy_{i,t}| \leq \Delta_{\max} \quad \forall i, t
\label{eq:smoothness}
\end{equation}

\textbf{Non-negativity:}
\begin{equation}
\policy_{i,t} \geq 0 \quad \forall i \in \mathcal{I}, \; t \in \mathcal{T}
\label{eq:nonnegativity}
\end{equation}

%--- 5.1.3 综合优化问题 ---
\subsubsection{Integrated Optimization Problem}

The complete multi-objective optimization problem is:

\begin{equation}
\begin{aligned}
& \underset{\boldsymbol{\policy}}{\text{optimize}}
& & \left( \objective_1(\boldsymbol{\policy}), \; \objective_2(\boldsymbol{\policy}), \; \objective_3(\boldsymbol{\policy}) \right) \\
& \text{subject to}
& & \text{Budget constraint \eqnref{eq:budget}} \\
& & & \text{Capacity constraint \eqnref{eq:capacity}} \\
& & & \text{Smoothness constraint \eqnref{eq:smoothness}} \\
& & & \text{Non-negativity \eqnref{eq:nonnegativity}} \\
& & & \text{System dynamics \eqnref{eq:state_evolution}--\eqnref{eq:outcome}}
\end{aligned}
\label{eq:moop}
\end{equation}

%=== 5.2 权衡分析框架 ===
\subsection{Trade-off Analysis Framework}
\label{sec:model2_tradeoff}

Given multiple objectives, we employ the \textbf{$\epsilon$-constraint method} combined with \textbf{Pareto frontier visualization} to explore trade-offs.

\textbf{$\epsilon$-Constraint Formulation:}

Fix objectives 2 and 3 as constraints, optimize objective 1:
\begin{equation}
\begin{aligned}
& \max_{\boldsymbol{\policy}} && \objective_1(\boldsymbol{\policy}) \\
& \text{s.t.} && \objective_2(\boldsymbol{\policy}) \leq \epsilon_2 \\
& && \objective_3(\boldsymbol{\policy}) \leq \epsilon_3 \\
& && \text{Original constraints}
\end{aligned}
\label{eq:epsilon_constraint}
\end{equation}

By systematically varying $(\epsilon_2, \epsilon_3)$, we trace the Pareto frontier.

\begin{figure}[H]
    \centering
    % \includegraphics[width=0.7\textwidth]{figures/pareto_frontier.pdf}
    \fbox{\parbox{0.65\textwidth}{\centering\vspace{4em}\TODO{Figure: Pareto Frontier showing Efficiency-Equity Trade-off}\vspace{4em}}}
    \caption{Pareto frontier illustrating trade-offs between efficiency ($\objective_1$) and equity ($\objective_2$)}
    \label{fig:pareto}
\end{figure}

%=== 5.3 鲁棒优化扩展 ===
\subsection{Robust Optimization Extension}
\label{sec:model2_robust}

To account for parameter uncertainty, we extend the formulation to a \textbf{robust optimization} framework.

\textbf{Uncertainty Set:}
Let $\mathcal{U}$ denote the uncertainty set for key parameters:
\begin{equation}
\mathcal{U} = \left\{ \boldsymbol{\theta} : \| \boldsymbol{\theta} - \hat{\boldsymbol{\theta}} \|_2 \leq \Gamma \right\}
\label{eq:uncertainty_set}
\end{equation}

where $\hat{\boldsymbol{\theta}}$ is the nominal parameter estimate and $\Gamma$ is the uncertainty budget.

\textbf{Robust Counterpart:}
\begin{equation}
\max_{\boldsymbol{\policy}} \min_{\boldsymbol{\theta} \in \mathcal{U}} \objective_1(\boldsymbol{\policy}; \boldsymbol{\theta})
\label{eq:robust}
\end{equation}

This ensures policy performance even under adverse parameter realizations.

%=== 5.4 政策情景设计 ===
\subsection{Policy Scenario Design}
\label{sec:model2_scenarios}

We evaluate policy performance across \TODO{N} representative scenarios:

\begin{table}[H]
\centering
\caption{Policy Scenarios for Comparative Analysis}
\label{tab:scenarios}
\begin{tabular}{l p{5cm} c c}
\toprule
\textbf{Scenario} & \textbf{Description} & \textbf{Policy Intensity} & \textbf{Budget} \\
\midrule
Baseline (S0) & Status quo, no new intervention & -- & \$0 \\
Conservative (S1) & \TODO{Low-intensity intervention} & Low & \TODO{Budget} \\
Moderate (S2) & \TODO{Balanced approach} & Medium & \TODO{Budget} \\
Aggressive (S3) & \TODO{High-intensity intervention} & High & \TODO{Budget} \\
Targeted (S4) & \TODO{Focused on high-impact units} & Variable & \TODO{Budget} \\
\bottomrule
\end{tabular}
\end{table}

Scenario analysis results are presented in \secref{sec:results}.


% Section 6: Solution Algorithm (求解算法)
% 06_solution_and_algorithm.tex - 求解算法
% ICM Problem F - Policy Science

\section{Solution Algorithm}
\label{sec:algorithm}

% 【写作指导】算法部分需要:
% 1. 清晰的算法流程描述
% 2. 复杂度分析
% 3. 收敛性/最优性讨论
% 4. 实现细节(便于复现)

%=== 6.1 算法设计思路 ===
\subsection{Algorithm Design Rationale}
\label{sec:algorithm_rationale}

The optimization problem \eqnref{eq:moop} presents several computational challenges:

\begin{itemize}
    \item \textbf{Non-convexity:} The objective functions involve nonlinear system dynamics.
    \item \textbf{Multi-objective nature:} No single optimal solution exists; we seek the Pareto frontier.
    \item \textbf{Large solution space:} $|\mathcal{I}| \times T$ decision variables with interdependencies.
\end{itemize}

To address these challenges, we develop a \textbf{\TODO{e.g., Hybrid Evolutionary-Local Search / Decomposition-Based / Simulation-Optimization}} algorithm.

%=== 6.2 算法流程 ===
\subsection{Algorithm Description}
\label{sec:algorithm_description}

\begin{algorithm}[H]
\caption{Policy Optimization Algorithm (\TODO{Algorithm Name})}
\label{alg:main}
\KwIn{Initial population $\mathcal{P}_0$, parameters $\{\text{pop\_size}, \text{max\_gen}, \text{crossover\_rate}, \text{mutation\_rate}\}$}
\KwOut{Pareto-optimal policy set $\mathcal{P}^*$}

\BlankLine
\tcp{Initialization}
$\mathcal{P} \leftarrow$ InitializePopulation(pop\_size)\;
EvaluateFitness($\mathcal{P}$) using mechanism model (\secref{sec:model1})\;
$\mathcal{A} \leftarrow \emptyset$ \tcp{Archive for non-dominated solutions}

\BlankLine
\tcp{Main evolutionary loop}
\For{$g = 1$ \KwTo max\_gen}{
    \tcp{Selection}
    $\mathcal{M} \leftarrow$ TournamentSelection($\mathcal{P}$, based on crowding distance)\;
    
    \tcp{Crossover and Mutation}
    $\mathcal{O} \leftarrow \emptyset$\;
    \For{each pair $(\boldsymbol{\policy}_1, \boldsymbol{\policy}_2)$ in $\mathcal{M}$}{
        \If{rand() $<$ crossover\_rate}{
            $(\boldsymbol{\policy}_1', \boldsymbol{\policy}_2') \leftarrow$ SBXCrossover($\boldsymbol{\policy}_1, \boldsymbol{\policy}_2$)\;
        }
        $\boldsymbol{\policy}_1' \leftarrow$ PolynomialMutation($\boldsymbol{\policy}_1'$, mutation\_rate)\;
        RepairConstraints($\boldsymbol{\policy}_1'$) \tcp{Ensure feasibility}
        $\mathcal{O} \leftarrow \mathcal{O} \cup \{\boldsymbol{\policy}_1'\}$\;
    }
    
    \tcp{Evaluation}
    EvaluateFitness($\mathcal{O}$)\;
    
    \tcp{Environmental Selection}
    $\mathcal{P} \leftarrow$ NSGA-II-Selection($\mathcal{P} \cup \mathcal{O}$, pop\_size)\;
    
    \tcp{Archive Update}
    $\mathcal{A} \leftarrow$ UpdateArchive($\mathcal{A}$, non-dominated solutions from $\mathcal{P}$)\;
    
    \tcp{Convergence Check}
    \If{ConvergenceCriterion($\mathcal{A}$)}{
        \textbf{break}\;
    }
}

\BlankLine
\tcp{Local Search Refinement}
$\mathcal{P}^* \leftarrow$ LocalSearch($\mathcal{A}$) \tcp{Gradient-based refinement}

\Return $\mathcal{P}^*$\;
\end{algorithm}

%=== 6.3 算法流程图 ===
\subsection{Algorithm Flowchart}
\label{sec:algorithm_flowchart}

\begin{figure}[H]
    \centering
    \begin{tikzpicture}[
        node distance=1.2cm,
        startstop/.style={rectangle, rounded corners, minimum width=3cm, minimum height=0.8cm, text centered, draw=black, fill=red!20},
        process/.style={rectangle, minimum width=3.5cm, minimum height=0.8cm, text centered, draw=black, fill=blue!10},
        decision/.style={diamond, aspect=2, minimum width=2cm, minimum height=0.8cm, text centered, draw=black, fill=green!10},
        arrow/.style={thick,->,>=stealth}
    ]
        % 节点
        \node (start) [startstop] {Start};
        \node (init) [process, below=of start] {Initialize Population};
        \node (eval) [process, below=of init] {Evaluate Fitness};
        \node (select) [process, below=of eval] {Selection \& Reproduction};
        \node (update) [process, below=of select] {Update Population};
        \node (converge) [decision, below=of update] {Converged?};
        \node (local) [process, below=of converge, yshift=-0.5cm] {Local Search Refinement};
        \node (end) [startstop, below=of local] {Output Pareto Set};
        
        % 箭头
        \draw [arrow] (start) -- (init);
        \draw [arrow] (init) -- (eval);
        \draw [arrow] (eval) -- (select);
        \draw [arrow] (select) -- (update);
        \draw [arrow] (update) -- (converge);
        \draw [arrow] (converge) -- node[anchor=east] {Yes} (local);
        \draw [arrow] (converge.east) -- ++(1.5,0) |- node[anchor=south, pos=0.25] {No} (eval.east);
        \draw [arrow] (local) -- (end);
    \end{tikzpicture}
    \caption{Flowchart of the policy optimization algorithm}
    \label{fig:flowchart}
\end{figure}

%=== 6.4 关键子程序 ===
\subsection{Key Subroutines}
\label{sec:algorithm_subroutines}

\textbf{Constraint Repair.} When genetic operators produce infeasible solutions, we apply:
\begin{equation}
\policy_{i,t}^{\text{repaired}} = \min\left( \max(\policy_{i,t}, 0), \bar{\policy}_i \right)
\label{eq:repair}
\end{equation}

\textbf{Crowding Distance.} To maintain solution diversity:
\begin{equation}
\text{CD}(\boldsymbol{\policy}) = \sum_{m=1}^{M} \frac{f_m^{(i+1)} - f_m^{(i-1)}}{f_m^{\max} - f_m^{\min}}
\label{eq:crowding}
\end{equation}

\textbf{Local Search.} For archive solutions, we apply gradient-based refinement:
\begin{equation}
\boldsymbol{\policy}^{(k+1)} = \boldsymbol{\policy}^{(k)} + \eta \cdot \nabla_{\boldsymbol{\policy}} \objective(\boldsymbol{\policy}^{(k)})
\label{eq:local_search}
\end{equation}

%=== 6.5 复杂度分析 ===
\subsection{Complexity Analysis}
\label{sec:algorithm_complexity}

\begin{table}[H]
\centering
\caption{Computational Complexity Analysis}
\label{tab:complexity}
\begin{tabular}{l l l}
\toprule
\textbf{Component} & \textbf{Complexity} & \textbf{Remarks} \\
\midrule
Fitness evaluation & $O(I \cdot T)$ & Per individual \\
Non-dominated sorting & $O(M \cdot N^2)$ & $M$ objectives, $N$ population \\
Crowding distance & $O(M \cdot N \log N)$ & Sorting required \\
Local search & $O(K \cdot I \cdot T)$ & $K$ iterations \\
\midrule
\textbf{Total per generation} & $O(N \cdot I \cdot T + M \cdot N^2)$ & \\
\bottomrule
\end{tabular}
\end{table}

\textbf{Convergence.} Empirically, the algorithm converges within \TODO{G} generations for problem instances with $|\mathcal{I}| \leq \TODO{I}$ and $T \leq \TODO{T}$. Formal convergence guarantees follow from standard evolutionary algorithm theory \cite{rudolph1998convergence}.

%=== 6.6 实现细节 ===
\subsection{Implementation Details}
\label{sec:algorithm_implementation}

\begin{itemize}
    \item \textbf{Programming Language:} Python 3.10 with NumPy, SciPy
    \item \textbf{Parallelization:} Fitness evaluations parallelized across \TODO{N} CPU cores
    \item \textbf{Parameter Settings:}
    \begin{itemize}
        \item Population size: \TODO{100}
        \item Maximum generations: \TODO{500}
        \item Crossover rate: \TODO{0.9}
        \item Mutation rate: \TODO{1/n} (where $n$ = number of decision variables)
    \end{itemize}
    \item \textbf{Runtime:} Approximately \TODO{X} minutes on \TODO{hardware specification}
\end{itemize}

Source code is provided in Appendix \ref{app:code}.


%=== PART III: Results & Validation ===

% Section 7: Results and Validation (结果与验证)
% 07_results_and_validation.tex - 结果与验证
% ICM Problem F - Policy Science

\section{Results and Validation}
\label{sec:results}

% 【写作指导】F题结果部分需要:
% 1. 情景比较(政策方案效果对比)
% 2. 分群体/分区域影响分析
% 3. 时间动态(政策效果演化)
% 4. 与基准/文献比较

%=== 7.1 主要结果概述 ===
\subsection{Overview of Key Results}
\label{sec:results_overview}

We apply the optimization framework to the \TODO{specific problem context} and present results across the policy scenarios defined in \secref{sec:model2_scenarios}.

\textbf{Key Finding Summary:}
\begin{itemize}
    \item \textbf{Optimal Policy:} Scenario \TODO{SX} achieves the best balance between \TODO{objective 1} and \TODO{objective 2}.
    \item \textbf{Efficiency Gains:} Compared to baseline, the recommended policy improves \TODO{outcome metric} by \TODO{X\%}.
    \item \textbf{Equity Impact:} Distributional analysis shows \TODO{reduction/increase} in inequality (Gini coefficient: \TODO{before} $\rightarrow$ \TODO{after}).
    \item \textbf{Cost-Effectiveness:} \TODO{Y} units of improvement per \TODO{Z currency} invested.
\end{itemize}

%=== 7.2 情景比较分析 ===
\subsection{Scenario Comparison}
\label{sec:results_scenarios}

\begin{table}[H]
\centering
\caption{Policy Scenario Performance Comparison}
\label{tab:scenario_results}
\small
\begin{tabular}{l c c c c c}
\toprule
\textbf{Scenario} & \textbf{Welfare ($\objective_1$)} & \textbf{Gini ($\objective_2$)} & \textbf{Cost ($\objective_3$)} & \textbf{ROI} & \textbf{Rank} \\
\midrule
Baseline (S0) & \TODO{value} & \TODO{value} & \$0 & -- & 5 \\
Conservative (S1) & \TODO{value} & \TODO{value} & \TODO{value} & \TODO{value} & 4 \\
Moderate (S2) & \TODO{value} & \TODO{value} & \TODO{value} & \TODO{value} & \textbf{1} \\
Aggressive (S3) & \TODO{value} & \TODO{value} & \TODO{value} & \TODO{value} & 3 \\
Targeted (S4) & \TODO{value} & \TODO{value} & \TODO{value} & \TODO{value} & 2 \\
\bottomrule
\end{tabular}
\end{table}

\begin{figure}[H]
    \centering
    % \includegraphics[width=0.85\textwidth]{figures/scenario_comparison.pdf}
    \fbox{\parbox{0.8\textwidth}{\centering\vspace{4em}\TODO{Figure: Radar Chart or Bar Chart Comparing Scenarios}\vspace{4em}}}
    \caption{Multi-dimensional comparison of policy scenarios}
    \label{fig:scenario_comparison}
\end{figure}

%=== 7.3 时间动态分析 ===
\subsection{Temporal Dynamics}
\label{sec:results_temporal}

We examine how policy effects unfold over the planning horizon:

\begin{figure}[H]
    \centering
    % \includegraphics[width=0.85\textwidth]{figures/temporal_dynamics.pdf}
    \fbox{\parbox{0.8\textwidth}{\centering\vspace{4em}\TODO{Figure: Time Series of Outcome Variables Under Different Scenarios}\vspace{4em}}}
    \caption{Temporal evolution of policy outcomes: Baseline vs. Recommended policy}
    \label{fig:temporal}
\end{figure}

\textbf{Key Temporal Observations:}
\begin{enumerate}
    \item \textbf{Policy Lag:} Effects begin materializing at $t = \TODO{X}$, consistent with the assumed lag parameter $\tau = \TODO{value}$.
    \item \textbf{Acceleration Phase:} Between $t = \TODO{X}$ and $t = \TODO{Y}$, outcomes improve at rate \TODO{description}.
    \item \textbf{Saturation:} Beyond $t = \TODO{Y}$, diminishing returns are observed as \TODO{mechanism}.
\end{enumerate}

%=== 7.4 分布影响分析 ===
\subsection{Distributional Impact Analysis}
\label{sec:results_distributional}

Policy impacts are not uniform across stakeholder groups. We disaggregate results by \TODO{dimension: region, income level, demographic group}:

\begin{table}[H]
\centering
\caption{Distributional Impacts by \TODO{Dimension}}
\label{tab:distributional}
\small
\begin{tabular}{l c c c c}
\toprule
\textbf{Group} & \textbf{Baseline Outcome} & \textbf{Policy Outcome} & \textbf{Abs. Change} & \textbf{\% Change} \\
\midrule
\TODO{Group 1} & \TODO{value} & \TODO{value} & \TODO{value} & \TODO{+X\%} \\
\TODO{Group 2} & \TODO{value} & \TODO{value} & \TODO{value} & \TODO{+Y\%} \\
\TODO{Group 3} & \TODO{value} & \TODO{value} & \TODO{value} & \TODO{+Z\%} \\
\TODO{Group 4} & \TODO{value} & \TODO{value} & \TODO{value} & \TODO{+W\%} \\
\bottomrule
\end{tabular}
\end{table}

\textbf{Equity Implications:}
The recommended policy demonstrates \TODO{progressive/regressive/neutral} distributional characteristics. \TODO{Explanation of why certain groups benefit more/less.}

%=== 7.5 模型验证 ===
\subsection{Model Validation}
\label{sec:results_validation}

We validate our results through multiple approaches:

\textbf{1. Historical Back-Testing.}
Applying the model to the period \TODO{years} and comparing predictions with observed outcomes:

\begin{table}[H]
\centering
\caption{Back-Testing Results}
\label{tab:backtest}
\begin{tabular}{l c c c c}
\toprule
\textbf{Metric} & \textbf{Predicted} & \textbf{Observed} & \textbf{Error} & \textbf{MAPE} \\
\midrule
\TODO{Metric 1} & \TODO{value} & \TODO{value} & \TODO{value} & \TODO{\%} \\
\TODO{Metric 2} & \TODO{value} & \TODO{value} & \TODO{value} & \TODO{\%} \\
\bottomrule
\end{tabular}
\end{table}

\textbf{2. Cross-Validation.}
Using \TODO{k}-fold cross-validation on regional data, we achieve prediction accuracy of \TODO{R² = value}.

\textbf{3. Benchmark Comparison.}
Our model's policy recommendations align with \TODO{established best practices / comparable international cases}. Specifically, \TODO{comparison details}.

%=== 7.6 Pareto前沿分析 ===
\subsection{Pareto Frontier Analysis}
\label{sec:results_pareto}

The optimization algorithm identifies a set of Pareto-optimal policies representing the efficiency-equity trade-off:

\begin{figure}[H]
    \centering
    % \includegraphics[width=0.75\textwidth]{figures/pareto_results.pdf}
    \fbox{\parbox{0.7\textwidth}{\centering\vspace{4em}\TODO{Figure: Pareto Frontier with Labeled Policy Solutions}\vspace{4em}}}
    \caption{Pareto frontier: Selected solutions labeled with scenario identifiers}
    \label{fig:pareto_results}
\end{figure}

\textbf{Interpretation:} Decision-makers can select policies along this frontier based on their value priorities:
\begin{itemize}
    \item \textbf{Efficiency-Focused:} Select solutions in the upper-left region (high $\objective_1$, higher $\objective_2$).
    \item \textbf{Equity-Focused:} Select solutions in the lower-right region (lower $\objective_1$, low $\objective_2$).
    \item \textbf{Balanced:} The ``knee'' of the frontier (Solution \TODO{ID}) offers the best compromise.
\end{itemize}


% Section 8: Sensitivity and Robustness (灵敏度与鲁棒性)
% 08_sensitivity_and_robustness.tex - 灵敏度与鲁棒性分析
% ICM Problem F - Policy Science

\section{Sensitivity and Robustness Analysis}
\label{sec:sensitivity}

% 【写作指导】F题灵敏度分析是O奖关键得分点
% 需要:参数扰动分析、情景压力测试、模型假设检验

%=== 8.1 灵敏度分析框架 ===
\subsection{Sensitivity Analysis Framework}
\label{sec:sensitivity_framework}

To assess the reliability of our policy recommendations, we conduct comprehensive sensitivity analysis using:

\begin{enumerate}
    \item \textbf{One-At-a-Time (OAT) Analysis:} Vary each parameter independently by $\pm$\TODO{X\%}.
    \item \textbf{Global Sensitivity Analysis:} Sobol indices to identify influential parameters.
    \item \textbf{Scenario Stress Testing:} Evaluate performance under extreme conditions.
\end{enumerate}

%=== 8.2 参数灵敏度 ===
\subsection{Parameter Sensitivity}
\label{sec:sensitivity_parameters}

We identify the following parameters as candidates for sensitivity analysis:

\begin{table}[H]
\centering
\caption{Parameters for Sensitivity Analysis}
\label{tab:sensitivity_params}
\small
\begin{tabular}{l l c c c}
\toprule
\textbf{Parameter} & \textbf{Description} & \textbf{Base Value} & \textbf{Range} & \textbf{Source of Uncertainty} \\
\midrule
$\alpha_1$ & Policy effect coefficient & \TODO{value} & [\TODO{L}, \TODO{U}] & Estimation error \\
$\discount$ & Discount factor & \TODO{value} & [0.90, 0.99] & Value judgment \\
$\tau$ & Policy lag (years) & \TODO{value} & [\TODO{L}, \TODO{U}] & Implementation variation \\
$\rho$ & Compliance rate & \TODO{value} & [0.60, 0.95] & Behavioral uncertainty \\
$B$ & Total budget & \TODO{value} & [\TODO{L}, \TODO{U}] & Political feasibility \\
\bottomrule
\end{tabular}
\end{table}

%--- 8.2.1 OAT分析结果 ---
\subsubsection{One-At-a-Time Analysis}

\begin{figure}[H]
    \centering
    % \includegraphics[width=0.9\textwidth]{figures/tornado_diagram.pdf}
    \fbox{\parbox{0.85\textwidth}{\centering\vspace{4em}\TODO{Figure: Tornado Diagram showing parameter impact on key outcome}\vspace{4em}}}
    \caption{Tornado diagram: Impact of parameter variations on policy outcome ($\outcome$)}
    \label{fig:tornado}
\end{figure}

\textbf{Key Observations:}
\begin{itemize}
    \item \textbf{Most Sensitive:} \TODO{Parameter X} - A \TODO{10\%} change leads to \TODO{Y\%} outcome variation.
    \item \textbf{Moderately Sensitive:} \TODO{Parameters} - \TODO{Description}.
    \item \textbf{Robust:} \TODO{Parameters} - Outcomes remain stable under variation.
\end{itemize}

%--- 8.2.2 全局敏感性分析 ---
\subsubsection{Global Sensitivity Analysis}

Using Sobol variance-based decomposition, we compute first-order ($S_i$) and total-effect ($S_{Ti}$) sensitivity indices:

\begin{table}[H]
\centering
\caption{Sobol Sensitivity Indices}
\label{tab:sobol}
\begin{tabular}{l c c c}
\toprule
\textbf{Parameter} & \textbf{$S_i$ (First-order)} & \textbf{$S_{Ti}$ (Total)} & \textbf{Interpretation} \\
\midrule
$\alpha_1$ & \TODO{value} & \TODO{value} & \TODO{High/Medium/Low influence} \\
$\discount$ & \TODO{value} & \TODO{value} & \TODO{Interpretation} \\
$\tau$ & \TODO{value} & \TODO{value} & \TODO{Interpretation} \\
$\rho$ & \TODO{value} & \TODO{value} & \TODO{Interpretation} \\
\bottomrule
\end{tabular}
\end{table}

\textbf{Interaction Effects:} The gap between $S_i$ and $S_{Ti}$ for \TODO{parameters} indicates significant parameter interactions, suggesting \TODO{implication for policy design}.

%=== 8.3 假设检验 ===
\subsection{Assumption Testing}
\label{sec:sensitivity_assumptions}

We systematically test the robustness of key assumptions identified in \secref{sec:assumptions}:

\begin{table}[H]
\centering
\caption{Assumption Robustness Testing}
\label{tab:assumption_test}
\small
\begin{tabular}{l p{4cm} p{4cm} c}
\toprule
\textbf{Assumption} & \textbf{Alternative Tested} & \textbf{Result} & \textbf{Robust?} \\
\midrule
A3: Bounded rationality & Full rationality model & \TODO{Qualitative difference} & \TODO{Yes/Partial/No} \\
A4: \TODO{Assumption} & \TODO{Alternative} & \TODO{Result} & \TODO{Yes/No} \\
A6: Parameter stability & Regime-switching model & \TODO{Result} & \TODO{Yes/No} \\
\bottomrule
\end{tabular}
\end{table}

%=== 8.4 情景压力测试 ===
\subsection{Scenario Stress Testing}
\label{sec:sensitivity_stress}

We evaluate policy performance under adverse conditions:

\textbf{Stress Scenario 1: Budget Shortfall}
\begin{itemize}
    \item \textit{Condition:} Budget reduced by \TODO{30\%}.
    \item \textit{Result:} Recommended policy remains \TODO{optimal / second-best}; outcome degrades by \TODO{X\%}.
    \item \textit{Adaptation:} Prioritize intervention in \TODO{high-impact regions/groups}.
\end{itemize}

\textbf{Stress Scenario 2: External Shock}
\begin{itemize}
    \item \textit{Condition:} \TODO{e.g., Economic recession reducing baseline by 15\%}.
    \item \textit{Result:} Policy effectiveness \TODO{increases / decreases / unchanged}.
    \item \textit{Adaptation:} \TODO{Recommended adjustment}.
\end{itemize}

\textbf{Stress Scenario 3: Implementation Failure}
\begin{itemize}
    \item \textit{Condition:} Compliance rate drops to \TODO{60\%}.
    \item \textit{Result:} Outcome shortfall of \TODO{X\%} compared to expected.
    \item \textit{Adaptation:} Strengthen enforcement mechanisms; consider incentive redesign.
\end{itemize}

%=== 8.5 鲁棒性综合评估 ===
\subsection{Robustness Summary}
\label{sec:sensitivity_summary}

\begin{figure}[H]
    \centering
    % \includegraphics[width=0.7\textwidth]{figures/robustness_heatmap.pdf}
    \fbox{\parbox{0.65\textwidth}{\centering\vspace{4em}\TODO{Figure: Robustness Heatmap - Policy Performance Across Parameter Combinations}\vspace{4em}}}
    \caption{Policy robustness heatmap: Performance variation across parameter space}
    \label{fig:robustness}
\end{figure}

\textbf{Overall Assessment:}

\begin{center}
\fbox{\parbox{0.9\textwidth}{
\textbf{Robustness Conclusion}\\[0.5em]
The recommended policy (Scenario \TODO{SX}) demonstrates \textbf{\TODO{high/moderate}} robustness:
\begin{itemize}
    \item Remains optimal or near-optimal in \TODO{X\%} of parameter combinations tested.
    \item Performance degradation under worst-case conditions is bounded at \TODO{Y\%}.
    \item Key risk factors requiring monitoring: \TODO{parameters/conditions}.
\end{itemize}
}}
\end{center}


%=== PART IV: Policy Implications & Conclusion ===

% Section 9: Policy Impacts and Discussion (政策影响与讨论)
% 09_policy_impacts_and_discussion.tex - 政策影响与讨论
% ICM Problem F - Policy Science
% 【F题核心特色章节】:溢出效应、系统稳定性、实施考量

\section{Policy Impacts and Discussion}
\label{sec:impacts}

% 【写作指导】F题的独特价值在于:
% 1. 超越直接效果,分析跨领域溢出
% 2. 评估长期系统稳定性
% 3. 提供可操作的实施路径

%=== 9.1 直接政策效果 ===
\subsection{Direct Policy Effects}
\label{sec:impacts_direct}

Building on the quantitative results (\secref{sec:results}), we interpret the direct policy effects:

\textbf{Primary Outcome Achievement:}
The recommended policy achieves \TODO{X\%} of the stated policy goal within the \TODO{T}-year horizon. Specifically:

\begin{itemize}
    \item \textbf{Magnitude:} \TODO{Outcome metric} improves from \TODO{baseline} to \TODO{target}.
    \item \textbf{Timing:} Most gains materialize between years \TODO{X} and \TODO{Y}.
    \item \textbf{Distribution:} Benefits are \TODO{progressively / regressively / evenly} distributed across \TODO{groups}.
\end{itemize}

%=== 9.2 溢出效应分析 ===
\subsection{Spillover Effects Analysis}
\label{sec:impacts_spillover}

Policies rarely operate in isolation. We analyze cross-domain spillovers using a \textbf{systems mapping} approach:

\begin{figure}[H]
    \centering
    \begin{tikzpicture}[
        scale=0.85,
        transform shape,
        node distance=2cm,
        policyarea/.style={ellipse, draw, thick, minimum width=2.5cm, minimum height=1.2cm, align=center},
        arrow/.style={->, >=stealth, thick},
        positive/.style={arrow, green!60!black},
        negative/.style={arrow, red!60!black},
        uncertain/.style={arrow, dashed, gray}
    ]
        % 核心领域
        \node[policyarea, fill=policyblue!30] (core) {Target\\Domain};
        
        % 相关领域
        \node[policyarea, fill=resultgreen!20, above left=1.5cm and 2cm of core] (domain1) {\TODO{Domain 1}\\e.g., Economy};
        \node[policyarea, fill=warningorange!20, above right=1.5cm and 2cm of core] (domain2) {\TODO{Domain 2}\\e.g., Environment};
        \node[policyarea, fill=riskred!20, below left=1.5cm and 2cm of core] (domain3) {\TODO{Domain 3}\\e.g., Social Equity};
        \node[policyarea, fill=neutralgray!20, below right=1.5cm and 2cm of core] (domain4) {\TODO{Domain 4}\\e.g., Public Health};
        
        % 溢出箭头
        \draw[positive] (core) -- node[above, sloped, font=\footnotesize] {+\TODO{effect}} (domain1);
        \draw[negative] (core) -- node[above, sloped, font=\footnotesize] {$-$\TODO{effect}} (domain2);
        \draw[positive] (core) -- node[below, sloped, font=\footnotesize] {+\TODO{effect}} (domain3);
        \draw[uncertain] (core) -- node[below, sloped, font=\footnotesize] {?\TODO{uncertain}} (domain4);
        
        % 反馈
        \draw[positive, bend left=20] (domain1) to (core);
    \end{tikzpicture}
    \caption{Cross-domain spillover map (+ positive, $-$ negative, ? uncertain)}
    \label{fig:spillover_map}
\end{figure}

\textbf{Positive Spillovers:}
\begin{enumerate}
    \item \textbf{\TODO{Domain 1}:} \TODO{e.g., ``Economic stimulus from policy investment creates X jobs''}.
    \item \textbf{\TODO{Domain 3}:} \TODO{e.g., ``Improved equity reduces social unrest risk''}.
\end{enumerate}

\textbf{Negative Spillovers (Trade-offs):}
\begin{enumerate}
    \item \textbf{\TODO{Domain 2}:} \TODO{e.g., ``Increased activity raises environmental footprint by Y\%''}.
    \item \textbf{Mitigation:} \TODO{Recommended complementary policy or design modification}.
\end{enumerate}

\textbf{Uncertain Spillovers:}
\begin{enumerate}
    \item \textbf{\TODO{Domain 4}:} \TODO{e.g., ``Long-term health effects depend on behavioral adaptation''}.
    \item \textbf{Monitoring Required:} We recommend \TODO{indicators to track}.
\end{enumerate}

%=== 9.3 长期系统稳定性 ===
\subsection{Long-term System Stability}
\label{sec:impacts_stability}

We assess whether the recommended policy leads to \textbf{sustainable equilibrium} or potential \textbf{instabilities}:

\textbf{Stability Analysis:}
Using the system dynamics model (\secref{sec:model1}), we analyze equilibrium behavior:

\begin{equation}
\text{Equilibrium condition: } \frac{d\stock}{dt} = 0 \implies \flow^{in}(\policy^*, \stock^*) = \flow^{out}(\stock^*)
\label{eq:equilibrium}
\end{equation}

% 稳定性判据
The Jacobian eigenvalues at the policy-induced equilibrium are:
\begin{equation}
\lambda_1 = \TODO{value}, \quad \lambda_2 = \TODO{value}, \quad \ldots
\label{eq:eigenvalues}
\end{equation}

Since all eigenvalues have \TODO{negative real parts / mixed signs}, the equilibrium is \TODO{asymptotically stable / unstable / conditionally stable}.

\textbf{Phase Diagram:}

\begin{figure}[H]
    \centering
    % \includegraphics[width=0.7\textwidth]{figures/phase_diagram.pdf}
    \fbox{\parbox{0.65\textwidth}{\centering\vspace{4em}\TODO{Figure: Phase Diagram showing system trajectories}\vspace{4em}}}
    \caption{Phase diagram: System trajectories under recommended policy}
    \label{fig:phase}
\end{figure}

\textbf{Stability Implications:}
\begin{itemize}
    \item \textbf{Self-Sustaining:} \TODO{Describe whether the policy creates conditions for continued success without ongoing intervention}.
    \item \textbf{Reversal Risk:} \TODO{Describe conditions under which policy gains might reverse}.
    \item \textbf{Tipping Points:} \TODO{Identify any critical thresholds where system behavior changes qualitatively}.
\end{itemize}

%=== 9.4 公平性与包容性 ===
\subsection{Equity and Inclusion Considerations}
\label{sec:impacts_equity}

Beyond aggregate outcomes, we explicitly address distributional justice:

\textbf{Equity Metrics:}
\begin{table}[H]
\centering
\caption{Equity Assessment}
\label{tab:equity}
\begin{tabular}{l c c c}
\toprule
\textbf{Equity Dimension} & \textbf{Baseline} & \textbf{With Policy} & \textbf{Assessment} \\
\midrule
Geographic (Gini) & \TODO{value} & \TODO{value} & \TODO{Improved/Worsened} \\
Income-based (Theil index) & \TODO{value} & \TODO{value} & \TODO{Assessment} \\
Demographic (\TODO{specific group}) & \TODO{gap} & \TODO{gap} & \TODO{Assessment} \\
Intergenerational & \TODO{metric} & \TODO{metric} & \TODO{Assessment} \\
\bottomrule
\end{tabular}
\end{table}

\textbf{Vulnerable Group Analysis:}
\TODO{Discuss how the policy affects the most vulnerable populations. Identify any groups that may be inadvertently harmed and propose mitigations.}

%=== 9.5 实施可行性 ===
\subsection{Implementation Feasibility}
\label{sec:impacts_implementation}

Translating model recommendations into practice requires attention to:

\textbf{1. Political Feasibility}
\begin{itemize}
    \item \textbf{Stakeholder Alignment:} \TODO{Which stakeholders support/oppose?}
    \item \textbf{Political Capital Required:} \TODO{Low/Medium/High}
    \item \textbf{Coalition Strategy:} \TODO{Recommended approach to build support}
\end{itemize}

\textbf{2. Administrative Capacity}
\begin{itemize}
    \item \textbf{Required Capabilities:} \TODO{List key implementation capacities}
    \item \textbf{Capacity Gaps:} \TODO{Identify potential bottlenecks}
    \item \textbf{Phased Rollout:} We recommend starting with \TODO{pilot regions/sectors} before national scale-up.
\end{itemize}

\textbf{3. Monitoring and Evaluation}
\begin{itemize}
    \item \textbf{Key Performance Indicators (KPIs):} \TODO{List 3-5 measurable indicators}
    \item \textbf{Data Collection:} \TODO{Describe monitoring infrastructure needs}
    \item \textbf{Adaptive Management:} We recommend policy review at \TODO{interval} with pre-specified adjustment triggers.
\end{itemize}

%=== 9.6 局限性讨论 ===
\subsection{Limitations}
\label{sec:impacts_limitations}

We acknowledge the following limitations in our analysis:

\begin{enumerate}
    \item \textbf{Model Simplifications:} \TODO{e.g., ``Assumed homogeneous behavioral response across regions''}.
    
    \item \textbf{Data Constraints:} \TODO{e.g., ``Limited historical data for parameter estimation''}.
    
    \item \textbf{Scope Boundaries:} \TODO{e.g., ``Did not model international spillovers''}.
    
    \item \textbf{Uncertainty Quantification:} \TODO{e.g., ``Deep uncertainty in long-term projections beyond Year X''}.
\end{enumerate}

These limitations suggest caution in \TODO{specific contexts} and point to future research needs (\secref{sec:conclusion}).


% Section 10: Strengths and Weaknesses (优缺点分析)
% 10_strengths_and_weaknesses.tex - 优缺点分析
% ICM Problem F - Policy Science

\section{Strengths and Weaknesses}
\label{sec:strengths}

% 【写作指导】模型优缺点分析需要:
% 1. 客观、平衡地评价
% 2. 对每个缺点提供改进方向
% 3. 展示对方法论的深刻理解

%=== 10.1 模型优势 ===
\subsection{Model Strengths}
\label{sec:strengths_advantages}

Our policy modeling framework offers several methodological and practical advantages:

\begin{table}[H]
\centering
\caption{Summary of Model Strengths}
\label{tab:strengths}
\begin{tabular}{l p{8cm}}
\toprule
\textbf{Strength} & \textbf{Description} \\
\midrule

\textbf{S1: Mechanism Transparency} &
Unlike black-box approaches, our model explicitly represents causal pathways, enabling stakeholders to understand \textit{how} policies generate effects. This supports trust-building and adaptive management. \\[0.5em]

\textbf{S2: Multi-Objective Balance} &
The Pareto optimization framework acknowledges that real policy involves trade-offs. Decision-makers can explore the efficiency-equity frontier rather than receiving a single ``optimal'' solution. \\[0.5em]

\textbf{S3: Robustness Testing} &
Comprehensive sensitivity and stress testing ensures recommendations are not artifacts of specific parameter values. Results remain valid under \TODO{X\%} of plausible conditions. \\[0.5em]

\textbf{S4: Policy Relevance} &
Direct translation of quantitative findings into the Policy Memorandum ensures actionability. Recommendations follow the SMART+A framework (Specific, Measurable, Achievable, Relevant, Time-bound + Adaptive). \\[0.5em]

\textbf{S5: Spillover Analysis} &
Explicit consideration of cross-domain effects prevents unintended consequences and supports integrated policy-making. \\[0.5em]

\textbf{S6: Modularity} &
The framework's modular design allows updating individual components (mechanism model, optimization objectives, constraints) as new evidence emerges. \\

\bottomrule
\end{tabular}
\end{table}

%=== 10.2 模型局限 ===
\subsection{Model Weaknesses}
\label{sec:strengths_weaknesses}

We candidly acknowledge limitations and propose mitigation strategies:

\begin{table}[H]
\centering
\caption{Summary of Model Weaknesses and Mitigations}
\label{tab:weaknesses}
\small
\begin{tabular}{l p{5cm} p{5cm}}
\toprule
\textbf{Weakness} & \textbf{Description} & \textbf{Mitigation / Future Work} \\
\midrule

\textbf{W1: Linearity Assumptions} &
Several model components assume linear or low-order polynomial relationships, which may not capture complex nonlinearities. &
Extend to nonparametric methods (Gaussian processes, neural networks) with interpretability layers. \\[0.5em]

\textbf{W2: Static Parameters} &
Parameters are treated as time-invariant, ignoring potential regime changes or learning effects. &
Incorporate adaptive parameter updating or regime-switching models. \\[0.5em]

\textbf{W3: Limited Behavioral Modeling} &
Stakeholder behavior is modeled at aggregate level; individual heterogeneity and strategic interactions are simplified. &
Develop agent-based extensions for fine-grained behavioral simulation. \\[0.5em]

\textbf{W4: Data Dependencies} &
Results are contingent on data quality and coverage. Sparse data for \TODO{specific variables/regions} increases uncertainty. &
Conduct sensitivity analysis on data-poor parameters; pursue additional data collection. \\[0.5em]

\textbf{W5: Computational Cost} &
Full Pareto frontier exploration requires substantial computational resources, limiting real-time policy updates. &
Develop surrogate models or approximate algorithms for faster iteration. \\[0.5em]

\textbf{W6: External Validity} &
Model calibrated on \TODO{context} may not generalize to significantly different settings. &
Validate on out-of-sample contexts; develop transfer learning approaches. \\

\bottomrule
\end{tabular}
\end{table}

%=== 10.3 与现有方法比较 ===
\subsection{Comparison with Alternative Approaches}
\label{sec:strengths_comparison}

To contextualize our framework, we compare with alternative policy modeling approaches:

\begin{table}[H]
\centering
\caption{Comparison with Alternative Approaches}
\label{tab:comparison}
\small
\begin{tabular}{p{2.5cm} p{3cm} p{3cm} p{3cm} p{2.5cm}}
\toprule
\textbf{Criterion} & \textbf{Our Approach} & \textbf{Traditional Econometric} & \textbf{Pure Optimization} & \textbf{Agent-Based} \\
\midrule
Mechanism insight & High & Medium & Low & High \\
Optimization rigor & High & Low & High & Low \\
Behavioral realism & Medium & Low & Low & High \\
Computational cost & Medium & Low & Medium & High \\
Policy actionability & High & Medium & Medium & Low \\
Data requirements & Medium & High & Low & High \\
\bottomrule
\end{tabular}
\end{table}

\textbf{Positioning:} Our framework occupies a ``middle ground'' that balances mechanism understanding with optimization rigor, making it well-suited for policy contexts requiring both analytical depth and practical actionability.

%=== 10.4 适用范围 ===
\subsection{Scope of Applicability}
\label{sec:strengths_scope}

\textbf{Best Suited For:}
\begin{itemize}
    \item Policy problems with identifiable causal mechanisms
    \item Multi-stakeholder contexts requiring trade-off navigation
    \item Settings with moderate data availability (sufficient for calibration, not requiring big data)
    \item Decision-makers seeking transparent, explainable recommendations
\end{itemize}

\textbf{Less Suited For:}
\begin{itemize}
    \item Highly complex adaptive systems with emergent behavior
    \item Real-time decision-making requiring sub-second response
    \item Contexts with fundamental model uncertainty (unknown unknowns)
    \item Pure prediction tasks without policy intervention interest
\end{itemize}

%=== 10.5 改进路线图 ===
\subsection{Improvement Roadmap}
\label{sec:strengths_roadmap}

For future development, we propose:

\begin{enumerate}[label=\textbf{Phase \arabic*:}]
    \item \textbf{Short-term (1-2 years):} Address W4 through targeted data collection; develop user-friendly decision support interface.
    
    \item \textbf{Medium-term (2-4 years):} Integrate agent-based behavioral modeling (W3); develop transfer learning for cross-context application (W6).
    
    \item \textbf{Long-term (4+ years):} Full adaptive system with real-time parameter updating (W2); integration with digital twin infrastructure for continuous policy optimization.
\end{enumerate}


% Section 11: Conclusion (结论)
% 11_conclusion.tex - 结论
% ICM Problem F - Policy Science

\section{Conclusion}
\label{sec:conclusion}

% 【写作指导】结论需要:
% 1. 简洁回顾问题与方法
% 2. 总结核心发现
% 3. 重申政策建议
% 4. 指出未来研究方向

%=== 11.1 研究总结 ===
\subsection{Summary}
\label{sec:conclusion_summary}

This paper addressed the policy challenge of \TODO{concise problem statement} through an integrated modeling framework combining \textbf{mechanism analysis}, \textbf{multi-objective optimization}, and \textbf{robust policy evaluation}.

\textbf{Methodological Contributions:}
\begin{enumerate}
    \item We developed a \TODO{system dynamics / causal inference} model capturing the pathways through which policy interventions generate societal outcomes (\secref{sec:model1}).
    
    \item We formulated a multi-objective optimization framework balancing \TODO{efficiency, equity, and cost-effectiveness}, enabling decision-makers to navigate inherent trade-offs (\secref{sec:model2}).
    
    \item We conducted comprehensive sensitivity and robustness analysis, demonstrating that recommendations remain valid under \TODO{X\%} of plausible parameter configurations (\secref{sec:sensitivity}).
    
    \item We extended analysis beyond direct effects to examine \textbf{cross-domain spillovers} and \textbf{long-term system stability} (\secref{sec:impacts}).
\end{enumerate}

%=== 11.2 核心发现 ===
\subsection{Key Findings}
\label{sec:conclusion_findings}

Our analysis yields the following principal findings:

\begin{enumerate}[label=\textbf{F\arabic*:}]
    \item \textbf{Policy Effectiveness:} The recommended intervention (Scenario \TODO{SX}) achieves \TODO{X\%} improvement in the primary outcome while maintaining \TODO{constraint satisfaction}.
    
    \item \textbf{Timing Matters:} Early action yields disproportionate benefits due to \TODO{compounding effects / system dynamics}. Delaying intervention by \TODO{T} years reduces effectiveness by \TODO{Y\%}.
    
    \item \textbf{Trade-off Navigation:} The Pareto frontier reveals that achieving \TODO{top 10\%} efficiency requires accepting \TODO{Z level} of inequality. The ``knee'' solution offers balanced performance.
    
    \item \textbf{Spillover Management:} Positive spillovers to \TODO{domains} can be leveraged; negative spillovers to \TODO{domains} require complementary mitigation policies.
    
    \item \textbf{Robustness:} Recommendations are robust to \TODO{most uncertain parameters} but sensitive to \TODO{specific parameter}, warranting monitoring and adaptive management.
\end{enumerate}

%=== 11.3 政策建议 ===
\subsection{Policy Recommendations}
\label{sec:conclusion_recommendations}

Based on our analysis, we offer the following recommendations to \TODO{target decision-maker}:

\begin{center}
\fbox{\parbox{0.92\textwidth}{
\textbf{Core Policy Recommendations}\\[0.5em]
\begin{enumerate}[label=\textbf{R\arabic*:}]
    \item \textbf{Implement \TODO{Policy Name}:} Allocate resources to \TODO{intervention} at intensity \TODO{level}, prioritizing \TODO{target population/region}.
    
    \item \textbf{Adopt Phased Rollout:} Begin with \TODO{pilot phase} to validate assumptions, then scale based on observed effectiveness.
    
    \item \textbf{Establish Monitoring System:} Track KPIs including \TODO{metrics} with \TODO{frequency} reporting cycles.
    
    \item \textbf{Prepare Adaptive Triggers:} If \TODO{indicator} exceeds \TODO{threshold}, activate \TODO{contingency policy}.
    
    \item \textbf{Coordinate Complementary Policies:} Address spillover effects through \TODO{complementary interventions in related domains}.
\end{enumerate}
}}
\end{center}

Detailed implementation guidance is provided in the \textbf{Policy Memorandum} (page \pageref{sec:memo}).

%=== 11.4 未来研究方向 ===
\subsection{Future Research Directions}
\label{sec:conclusion_future}

This work opens several avenues for future research:

\begin{enumerate}
    \item \textbf{Behavioral Extensions:} Incorporate agent-based modeling to capture heterogeneous stakeholder responses and strategic interactions.
    
    \item \textbf{Real-Time Adaptation:} Develop online learning algorithms for continuous policy updating as new data becomes available.
    
    \item \textbf{Cross-Context Validation:} Test framework applicability in \TODO{alternative contexts/regions} to establish external validity.
    
    \item \textbf{Integration with Digital Twins:} Connect policy models with real-time data infrastructure for \TODO{specific application}.
    
    \item \textbf{Deep Uncertainty Methods:} Apply scenario discovery and robust decision-making techniques for contexts with fundamental uncertainty.
\end{enumerate}

%=== 11.5 结语 ===
\subsection{Closing Remarks}
\label{sec:conclusion_closing}

Effective policy-making in the face of complex, interconnected challenges requires analytical frameworks that are simultaneously rigorous and actionable. This paper demonstrates that such frameworks are achievable through careful integration of mechanism modeling, optimization theory, and systems thinking.

We hope this work contributes not only to addressing the specific challenge of \TODO{topic} but also to advancing the broader practice of evidence-based policy design. The tools and approaches developed here are intended to be \textbf{transferable} and \textbf{adaptable} to the diverse policy challenges facing societies worldwide.

\vspace{1em}
\begin{center}
\textit{``The purpose of models is not to fit the data, but to sharpen the questions.''}\\
--- Samuel Karlin
\end{center}


% ============================================
% References (参考文献)
% ============================================
\newpage
\nocite{*}
\printbibliography[heading=bibintoc, title={References}]

% 保存正文总页数
\savemainpages

% ============================================
% Appendices (附录,不计入25页限制)
% ============================================
\clearpage
\appendix
\pagenumbering{gobble}
\pagestyle{plain}

% Appendix A: Key Code
\section*{Appendix A: Key Code}
\addcontentsline{toc}{section}{Appendix A: Key Code}
\label{appendix:code}

Core algorithms are summarized below; full code available in supplementary materials.

\subsection*{A.1 LP Inversion Engine (Simplified)}

\begin{lstlisting}[language=Python, caption={LP fan vote bound computation}]
def compute_bounds(judge_scores, eliminated_idx, eps=0.01):
    n, j_bar = len(judge_scores), np.mean(judge_scores)
    A_ub, b_ub = [], []
    for i in range(n):
        if i != eliminated_idx:
            c = np.zeros(n); c[eliminated_idx], c[i] = 1, -1
            A_ub.append(c)
            b_ub.append((judge_scores[i]-judge_scores[eliminated_idx])/j_bar)
    bounds = [(eps, 1.0) for _ in range(n)]
    results = {}
    for i in range(n):
        c_min = np.zeros(n); c_min[i] = 1
        res = linprog(c_min, A_ub, b_ub, [[1]*n], [1], bounds)
        results[i] = res.fun if res.success else None
    return results
\end{lstlisting}

\subsection*{A.2 Cox Proportional Hazards}

\begin{lstlisting}[language=Python, caption={Survival analysis}]
from lifelines import CoxPHFitter
model = CoxPHFitter()
model.fit(df[['weeks','eliminated','judge_score','fan_vote','celeb_type']],
          duration_col='weeks', event_col='eliminated')
hazard_ratios = np.exp(model.summary['coef'])
\end{lstlisting}

\subsection*{A.3 Weighted Borda Mechanism}

\begin{lstlisting}[language=Python, caption={Borda counterfactual}]
def borda_score(j_scores, v_shares, alpha=0.6):
    j_norm = (j_scores - j_scores.min()) / (j_scores.ptp() + 1e-8)
    v_norm = (v_shares - v_shares.min()) / (v_shares.ptp() + 1e-8)
    return alpha * j_norm + (1-alpha) * v_norm
\end{lstlisting}

\newpage
% appendix_figures.tex - 补充图表附录
% ICM Problem F - Policy Science

\section*{Appendix B: Supplementary Figures and Tables}
\addcontentsline{toc}{section}{Appendix B: Supplementary Figures and Tables}
\label{app:figures}

% 【写作指导】补充材料包括:
% 1. 正文因篇幅限制未能放入的详细图表
% 2. 支持性分析结果
% 3. 额外的数据展示

\subsection*{B.1 Extended Data Description}

\begin{table}[H]
\centering
\caption{Full Variable Definitions and Sources}
\label{tab:app_variables}
\small
\begin{tabular}{p{2.5cm} p{4cm} p{3cm} p{4cm}}
\toprule
\textbf{Variable} & \textbf{Definition} & \textbf{Unit} & \textbf{Source} \\
\midrule
\TODO{Var1} & \TODO{Full definition} & \TODO{Unit} & \TODO{Source with year} \\
\TODO{Var2} & \TODO{Full definition} & \TODO{Unit} & \TODO{Source} \\
\TODO{Var3} & \TODO{Full definition} & \TODO{Unit} & \TODO{Source} \\
\TODO{Var4} & \TODO{Full definition} & \TODO{Unit} & \TODO{Source} \\
\TODO{Var5} & \TODO{Full definition} & \TODO{Unit} & \TODO{Source} \\
\bottomrule
\end{tabular}
\end{table}

\subsection*{B.2 Additional Sensitivity Results}

\begin{figure}[H]
    \centering
    % \includegraphics[width=0.85\textwidth]{figures/app_sensitivity_full.pdf}
    \fbox{\parbox{0.8\textwidth}{\centering\vspace{5em}\TODO{Figure B.1: Complete Sensitivity Analysis Results}\vspace{5em}}}
    \caption{Full parameter sensitivity matrix showing all pairwise interactions}
    \label{fig:app_sensitivity}
\end{figure}

\subsection*{B.3 Regional Disaggregation}

\begin{table}[H]
\centering
\caption{Policy Outcomes by Region}
\label{tab:app_regional}
\small
\begin{tabular}{l c c c c c}
\toprule
\textbf{Region} & \textbf{Baseline} & \textbf{Scenario S1} & \textbf{Scenario S2} & \textbf{Scenario S3} & \textbf{Scenario S4} \\
\midrule
\TODO{Region 1} & \TODO{val} & \TODO{val} & \TODO{val} & \TODO{val} & \TODO{val} \\
\TODO{Region 2} & \TODO{val} & \TODO{val} & \TODO{val} & \TODO{val} & \TODO{val} \\
\TODO{Region 3} & \TODO{val} & \TODO{val} & \TODO{val} & \TODO{val} & \TODO{val} \\
\TODO{Region 4} & \TODO{val} & \TODO{val} & \TODO{val} & \TODO{val} & \TODO{val} \\
\TODO{Region 5} & \TODO{val} & \TODO{val} & \TODO{val} & \TODO{val} & \TODO{val} \\
\midrule
\textbf{National} & \TODO{val} & \TODO{val} & \TODO{val} & \TODO{val} & \TODO{val} \\
\bottomrule
\end{tabular}
\end{table}

\subsection*{B.4 Temporal Evolution Details}

\begin{figure}[H]
    \centering
    % \includegraphics[width=0.9\textwidth]{figures/app_temporal_all.pdf}
    \fbox{\parbox{0.85\textwidth}{\centering\vspace{5em}\TODO{Figure B.2: Year-by-Year Outcome Trajectories for All Scenarios}\vspace{5em}}}
    \caption{Detailed temporal evolution of key outcomes under each policy scenario}
    \label{fig:app_temporal}
\end{figure}

\subsection*{B.5 Algorithm Convergence}

\begin{figure}[H]
    \centering
    % \includegraphics[width=0.7\textwidth]{figures/app_convergence.pdf}
    \fbox{\parbox{0.65\textwidth}{\centering\vspace{4em}\TODO{Figure B.3: Optimization Algorithm Convergence Curve}\vspace{4em}}}
    \caption{Convergence of hypervolume indicator over generations}
    \label{fig:app_convergence}
\end{figure}

\subsection*{B.6 Pareto Front Alternative Views}

\begin{figure}[H]
    \centering
    % \includegraphics[width=0.85\textwidth]{figures/app_pareto_3d.pdf}
    \fbox{\parbox{0.8\textwidth}{\centering\vspace{5em}\TODO{Figure B.4: 3D Pareto Frontier Visualization}\vspace{5em}}}
    \caption{Three-dimensional view of the Pareto frontier showing all three objectives}
    \label{fig:app_pareto_3d}
\end{figure}

\subsection*{B.7 Robustness Heat Maps}

\begin{figure}[H]
    \centering
    % \includegraphics[width=0.85\textwidth]{figures/app_robustness_detailed.pdf}
    \fbox{\parbox{0.8\textwidth}{\centering\vspace{5em}\TODO{Figure B.5: Detailed Robustness Analysis Heat Maps}\vspace{5em}}}
    \caption{Policy performance across the full parameter uncertainty space}
    \label{fig:app_robustness}
\end{figure}

\subsection*{B.8 Stakeholder Impact Summary}

\begin{table}[H]
\centering
\caption{Detailed Stakeholder Impact Assessment}
\label{tab:app_stakeholder}
\small
\begin{tabular}{l c c c p{4cm}}
\toprule
\textbf{Stakeholder} & \textbf{Direct Impact} & \textbf{Indirect Impact} & \textbf{Net Effect} & \textbf{Notes} \\
\midrule
\TODO{Stakeholder 1} & \TODO{+/-} & \TODO{+/-} & \TODO{Net} & \TODO{Brief explanation} \\
\TODO{Stakeholder 2} & \TODO{+/-} & \TODO{+/-} & \TODO{Net} & \TODO{Brief explanation} \\
\TODO{Stakeholder 3} & \TODO{+/-} & \TODO{+/-} & \TODO{Net} & \TODO{Brief explanation} \\
\TODO{Stakeholder 4} & \TODO{+/-} & \TODO{+/-} & \TODO{Net} & \TODO{Brief explanation} \\
\bottomrule
\end{tabular}
\end{table}

\newpage
% 12_ai_tool_report.tex - AI工具使用报告
% ICM Problem F - AI使用合规声明

\section*{Appendix C: AI Tool Usage Report}
\addcontentsline{toc}{section}{Appendix C: AI Tool Usage Report}
\label{app:ai}

% 【写作指导】MCM/ICM 2025+ 要求披露AI使用情况
% 本附录需要:诚实、具体、可追溯

\subsection*{C.1 AI Tools Employed}

The following AI-powered tools were used during the development of this paper:

\begin{table}[H]
\centering
\caption{AI Tools Usage Summary}
\label{tab:ai_tools}
\small
\begin{tabular}{p{2.5cm} p{2.5cm} p{6cm} p{3cm}}
\toprule
\textbf{Tool Name} & \textbf{Provider} & \textbf{Purpose} & \textbf{Usage Extent} \\
\midrule
\TODO{e.g., ChatGPT-4} & \TODO{OpenAI} & \TODO{e.g., Literature review assistance, equation derivation checking} & \TODO{Moderate} \\
\TODO{e.g., GitHub Copilot} & \TODO{Microsoft} & \TODO{e.g., Code completion and debugging assistance} & \TODO{Light} \\
\TODO{e.g., Grammarly} & \TODO{Grammarly Inc.} & \TODO{e.g., Grammar and style checking for English writing} & \TODO{Heavy} \\
\TODO{Tool 4} & \TODO{Provider} & \TODO{Purpose description} & \TODO{Extent} \\
\bottomrule
\end{tabular}
\end{table}

\subsection*{C.2 Detailed Usage Description}

\textbf{Literature Review \& Background Research}

\TODO{Describe how AI was used for initial literature search, concept clarification, or background understanding. Example: ``We used ChatGPT to generate initial keyword suggestions for literature search and to explain unfamiliar technical concepts in system dynamics modeling.''}

\textbf{Model Development}

\TODO{Describe AI assistance in model formulation. Example: ``GitHub Copilot provided code completion suggestions during Python implementation. All mathematical formulations were independently derived and verified by team members.''}

\textbf{Writing \& Editing}

\TODO{Describe writing assistance. Example: ``Grammarly was used to check grammar and improve clarity of English prose. ChatGPT was occasionally consulted for phrasing suggestions, particularly for translating technical concepts for non-specialist readers.''}

\textbf{Visualization}

\TODO{Describe any AI-assisted figure generation. Example: ``DALL-E was not used. All figures were created manually using matplotlib, TikZ, or standard plotting tools.''}

\subsection*{C.3 Human Oversight Statement}

We affirm that:

\begin{enumerate}
    \item \textbf{Original Thinking:} All core ideas, model formulations, and policy recommendations originated from human team members. AI tools provided support but did not generate novel intellectual contributions.
    
    \item \textbf{Verification:} All AI-generated suggestions (code, text, equations) were critically reviewed and verified by team members before inclusion.
    
    \item \textbf{Accountability:} Team members take full responsibility for the accuracy, integrity, and originality of this work.
    
    \item \textbf{No Fabrication:} No AI tool was used to fabricate data, results, or citations. All references are real and were verified by team members.
\end{enumerate}

\subsection*{C.4 Specific AI Interaction Log}

For transparency, we provide representative examples of AI interactions:

\begin{table}[H]
\centering
\caption{Sample AI Interaction Log}
\label{tab:ai_log}
\small
\begin{tabular}{p{1.5cm} p{2cm} p{4.5cm} p{5.5cm}}
\toprule
\textbf{Date} & \textbf{Tool} & \textbf{Query/Task} & \textbf{Outcome \& Human Follow-up} \\
\midrule
\TODO{Date} & \TODO{Tool} & \TODO{e.g., ``Explain Sobol sensitivity analysis''} & \TODO{e.g., ``Used explanation as starting point; read original Sobol (1993) paper for verification''} \\
\TODO{Date} & \TODO{Tool} & \TODO{Query description} & \TODO{Outcome description} \\
\TODO{Date} & \TODO{Tool} & \TODO{Query description} & \TODO{Outcome description} \\
\bottomrule
\end{tabular}
\end{table}

\noindent\textit{Complete interaction logs are maintained by the team and available upon request for audit purposes.}

\subsection*{C.5 Compliance Statement}

This work complies with the MCM/ICM AI usage guidelines. We have:

\begin{itemize}
    \item[$\checkmark$] Disclosed all AI tools used
    \item[$\checkmark$] Described the nature and extent of AI assistance
    \item[$\checkmark$] Ensured human oversight of all AI-generated content
    \item[$\checkmark$] Verified the accuracy of any AI-assisted outputs
    \item[$\checkmark$] Taken full responsibility for the final work product
\end{itemize}

\vspace{1em}
\begin{center}
\textit{Signed: Team \#2617892}\\
\textit{Date: \today}
\end{center}


\end{document}
