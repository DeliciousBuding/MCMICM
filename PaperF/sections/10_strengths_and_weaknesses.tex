% 10_strengths_and_weaknesses.tex - 优缺点分析
% ICM Problem F - Policy Science

\section{Strengths and Weaknesses}
\label{sec:strengths}

% 【写作指导】模型优缺点分析需要:
% 1. 客观、平衡地评价
% 2. 对每个缺点提供改进方向
% 3. 展示对方法论的深刻理解

%=== 10.1 模型优势 ===
\subsection{Model Strengths}
\label{sec:strengths_advantages}

Our policy modeling framework offers several methodological and practical advantages:

\begin{table}[H]
\centering
\caption{Summary of Model Strengths}
\label{tab:strengths}
\begin{tabular}{l p{8cm}}
\toprule
\textbf{Strength} & \textbf{Description} \\
\midrule

\textbf{S1: Mechanism Transparency} &
Unlike black-box approaches, our model explicitly represents causal pathways, enabling stakeholders to understand \textit{how} policies generate effects. This supports trust-building and adaptive management. \\[0.5em]

\textbf{S2: Multi-Objective Balance} &
The Pareto optimization framework acknowledges that real policy involves trade-offs. Decision-makers can explore the efficiency-equity frontier rather than receiving a single ``optimal'' solution. \\[0.5em]

\textbf{S3: Robustness Testing} &
Comprehensive sensitivity and stress testing ensures recommendations are not artifacts of specific parameter values. Results remain valid under \TODO{X\%} of plausible conditions. \\[0.5em]

\textbf{S4: Policy Relevance} &
Direct translation of quantitative findings into the Policy Memorandum ensures actionability. Recommendations follow the SMART+A framework (Specific, Measurable, Achievable, Relevant, Time-bound + Adaptive). \\[0.5em]

\textbf{S5: Spillover Analysis} &
Explicit consideration of cross-domain effects prevents unintended consequences and supports integrated policy-making. \\[0.5em]

\textbf{S6: Modularity} &
The framework's modular design allows updating individual components (mechanism model, optimization objectives, constraints) as new evidence emerges. \\

\bottomrule
\end{tabular}
\end{table}

%=== 10.2 模型局限 ===
\subsection{Model Weaknesses}
\label{sec:strengths_weaknesses}

We candidly acknowledge limitations and propose mitigation strategies:

\begin{table}[H]
\centering
\caption{Summary of Model Weaknesses and Mitigations}
\label{tab:weaknesses}
\small
\begin{tabular}{l p{5cm} p{5cm}}
\toprule
\textbf{Weakness} & \textbf{Description} & \textbf{Mitigation / Future Work} \\
\midrule

\textbf{W1: Linearity Assumptions} &
Several model components assume linear or low-order polynomial relationships, which may not capture complex nonlinearities. &
Extend to nonparametric methods (Gaussian processes, neural networks) with interpretability layers. \\[0.5em]

\textbf{W2: Static Parameters} &
Parameters are treated as time-invariant, ignoring potential regime changes or learning effects. &
Incorporate adaptive parameter updating or regime-switching models. \\[0.5em]

\textbf{W3: Limited Behavioral Modeling} &
Stakeholder behavior is modeled at aggregate level; individual heterogeneity and strategic interactions are simplified. &
Develop agent-based extensions for fine-grained behavioral simulation. \\[0.5em]

\textbf{W4: Data Dependencies} &
Results are contingent on data quality and coverage. Sparse data for \TODO{specific variables/regions} increases uncertainty. &
Conduct sensitivity analysis on data-poor parameters; pursue additional data collection. \\[0.5em]

\textbf{W5: Computational Cost} &
Full Pareto frontier exploration requires substantial computational resources, limiting real-time policy updates. &
Develop surrogate models or approximate algorithms for faster iteration. \\[0.5em]

\textbf{W6: External Validity} &
Model calibrated on \TODO{context} may not generalize to significantly different settings. &
Validate on out-of-sample contexts; develop transfer learning approaches. \\

\bottomrule
\end{tabular}
\end{table}

%=== 10.3 与现有方法比较 ===
\subsection{Comparison with Alternative Approaches}
\label{sec:strengths_comparison}

To contextualize our framework, we compare with alternative policy modeling approaches:

\begin{table}[H]
\centering
\caption{Comparison with Alternative Approaches}
\label{tab:comparison}
\small
\begin{tabular}{p{2.5cm} p{3cm} p{3cm} p{3cm} p{2.5cm}}
\toprule
\textbf{Criterion} & \textbf{Our Approach} & \textbf{Traditional Econometric} & \textbf{Pure Optimization} & \textbf{Agent-Based} \\
\midrule
Mechanism insight & High & Medium & Low & High \\
Optimization rigor & High & Low & High & Low \\
Behavioral realism & Medium & Low & Low & High \\
Computational cost & Medium & Low & Medium & High \\
Policy actionability & High & Medium & Medium & Low \\
Data requirements & Medium & High & Low & High \\
\bottomrule
\end{tabular}
\end{table}

\textbf{Positioning:} Our framework occupies a ``middle ground'' that balances mechanism understanding with optimization rigor, making it well-suited for policy contexts requiring both analytical depth and practical actionability.

%=== 10.4 适用范围 ===
\subsection{Scope of Applicability}
\label{sec:strengths_scope}

\textbf{Best Suited For:}
\begin{itemize}
    \item Policy problems with identifiable causal mechanisms
    \item Multi-stakeholder contexts requiring trade-off navigation
    \item Settings with moderate data availability (sufficient for calibration, not requiring big data)
    \item Decision-makers seeking transparent, explainable recommendations
\end{itemize}

\textbf{Less Suited For:}
\begin{itemize}
    \item Highly complex adaptive systems with emergent behavior
    \item Real-time decision-making requiring sub-second response
    \item Contexts with fundamental model uncertainty (unknown unknowns)
    \item Pure prediction tasks without policy intervention interest
\end{itemize}

%=== 10.5 改进路线图 ===
\subsection{Improvement Roadmap}
\label{sec:strengths_roadmap}

For future development, we propose:

\begin{enumerate}[label=\textbf{Phase \arabic*:}]
    \item \textbf{Short-term (1-2 years):} Address W4 through targeted data collection; develop user-friendly decision support interface.
    
    \item \textbf{Medium-term (2-4 years):} Integrate agent-based behavioral modeling (W3); develop transfer learning for cross-context application (W6).
    
    \item \textbf{Long-term (4+ years):} Full adaptive system with real-time parameter updating (W2); integration with digital twin infrastructure for continuous policy optimization.
\end{enumerate}
