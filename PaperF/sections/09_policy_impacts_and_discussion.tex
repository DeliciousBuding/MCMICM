% 09_policy_impacts_and_discussion.tex - 政策影响与讨论
% ICM Problem F - Policy Science
% 【F题核心特色章节】:溢出效应、系统稳定性、实施考量

\section{Policy Impacts and Discussion}
\label{sec:impacts}

% 【写作指导】F题的独特价值在于:
% 1. 超越直接效果,分析跨领域溢出
% 2. 评估长期系统稳定性
% 3. 提供可操作的实施路径

%=== 9.1 直接政策效果 ===
\subsection{Direct Policy Effects}
\label{sec:impacts_direct}

Building on the quantitative results (\secref{sec:results}), we interpret the direct policy effects:

\textbf{Primary Outcome Achievement:}
The recommended policy achieves \TODO{X\%} of the stated policy goal within the \TODO{T}-year horizon. Specifically:

\begin{itemize}
    \item \textbf{Magnitude:} \TODO{Outcome metric} improves from \TODO{baseline} to \TODO{target}.
    \item \textbf{Timing:} Most gains materialize between years \TODO{X} and \TODO{Y}.
    \item \textbf{Distribution:} Benefits are \TODO{progressively / regressively / evenly} distributed across \TODO{groups}.
\end{itemize}

%=== 9.2 溢出效应分析 ===
\subsection{Spillover Effects Analysis}
\label{sec:impacts_spillover}

Policies rarely operate in isolation. We analyze cross-domain spillovers using a \textbf{systems mapping} approach:

\begin{figure}[H]
    \centering
    \begin{tikzpicture}[
        scale=0.85,
        transform shape,
        node distance=2cm,
        policyarea/.style={ellipse, draw, thick, minimum width=2.5cm, minimum height=1.2cm, align=center},
        arrow/.style={->, >=stealth, thick},
        positive/.style={arrow, green!60!black},
        negative/.style={arrow, red!60!black},
        uncertain/.style={arrow, dashed, gray}
    ]
        % 核心领域
        \node[policyarea, fill=policyblue!30] (core) {Target\\Domain};
        
        % 相关领域
        \node[policyarea, fill=resultgreen!20, above left=1.5cm and 2cm of core] (domain1) {\TODO{Domain 1}\\e.g., Economy};
        \node[policyarea, fill=warningorange!20, above right=1.5cm and 2cm of core] (domain2) {\TODO{Domain 2}\\e.g., Environment};
        \node[policyarea, fill=riskred!20, below left=1.5cm and 2cm of core] (domain3) {\TODO{Domain 3}\\e.g., Social Equity};
        \node[policyarea, fill=neutralgray!20, below right=1.5cm and 2cm of core] (domain4) {\TODO{Domain 4}\\e.g., Public Health};
        
        % 溢出箭头
        \draw[positive] (core) -- node[above, sloped, font=\footnotesize] {+\TODO{effect}} (domain1);
        \draw[negative] (core) -- node[above, sloped, font=\footnotesize] {$-$\TODO{effect}} (domain2);
        \draw[positive] (core) -- node[below, sloped, font=\footnotesize] {+\TODO{effect}} (domain3);
        \draw[uncertain] (core) -- node[below, sloped, font=\footnotesize] {?\TODO{uncertain}} (domain4);
        
        % 反馈
        \draw[positive, bend left=20] (domain1) to (core);
    \end{tikzpicture}
    \caption{Cross-domain spillover map (+ positive, $-$ negative, ? uncertain)}
    \label{fig:spillover_map}
\end{figure}

\textbf{Positive Spillovers:}
\begin{enumerate}
    \item \textbf{\TODO{Domain 1}:} \TODO{e.g., ``Economic stimulus from policy investment creates X jobs''}.
    \item \textbf{\TODO{Domain 3}:} \TODO{e.g., ``Improved equity reduces social unrest risk''}.
\end{enumerate}

\textbf{Negative Spillovers (Trade-offs):}
\begin{enumerate}
    \item \textbf{\TODO{Domain 2}:} \TODO{e.g., ``Increased activity raises environmental footprint by Y\%''}.
    \item \textbf{Mitigation:} \TODO{Recommended complementary policy or design modification}.
\end{enumerate}

\textbf{Uncertain Spillovers:}
\begin{enumerate}
    \item \textbf{\TODO{Domain 4}:} \TODO{e.g., ``Long-term health effects depend on behavioral adaptation''}.
    \item \textbf{Monitoring Required:} We recommend \TODO{indicators to track}.
\end{enumerate}

%=== 9.3 长期系统稳定性 ===
\subsection{Long-term System Stability}
\label{sec:impacts_stability}

We assess whether the recommended policy leads to \textbf{sustainable equilibrium} or potential \textbf{instabilities}:

\textbf{Stability Analysis:}
Using the system dynamics model (\secref{sec:model1}), we analyze equilibrium behavior:

\begin{equation}
\text{Equilibrium condition: } \frac{d\stock}{dt} = 0 \implies \flow^{in}(\policy^*, \stock^*) = \flow^{out}(\stock^*)
\label{eq:equilibrium}
\end{equation}

% 稳定性判据
The Jacobian eigenvalues at the policy-induced equilibrium are:
\begin{equation}
\lambda_1 = \TODO{value}, \quad \lambda_2 = \TODO{value}, \quad \ldots
\label{eq:eigenvalues}
\end{equation}

Since all eigenvalues have \TODO{negative real parts / mixed signs}, the equilibrium is \TODO{asymptotically stable / unstable / conditionally stable}.

\textbf{Phase Diagram:}

\begin{figure}[H]
    \centering
    % \includegraphics[width=0.7\textwidth]{figures/phase_diagram.pdf}
    \fbox{\parbox{0.65\textwidth}{\centering\vspace{4em}\TODO{Figure: Phase Diagram showing system trajectories}\vspace{4em}}}
    \caption{Phase diagram: System trajectories under recommended policy}
    \label{fig:phase}
\end{figure}

\textbf{Stability Implications:}
\begin{itemize}
    \item \textbf{Self-Sustaining:} \TODO{Describe whether the policy creates conditions for continued success without ongoing intervention}.
    \item \textbf{Reversal Risk:} \TODO{Describe conditions under which policy gains might reverse}.
    \item \textbf{Tipping Points:} \TODO{Identify any critical thresholds where system behavior changes qualitatively}.
\end{itemize}

%=== 9.4 公平性与包容性 ===
\subsection{Equity and Inclusion Considerations}
\label{sec:impacts_equity}

Beyond aggregate outcomes, we explicitly address distributional justice:

\textbf{Equity Metrics:}
\begin{table}[H]
\centering
\caption{Equity Assessment}
\label{tab:equity}
\begin{tabular}{l c c c}
\toprule
\textbf{Equity Dimension} & \textbf{Baseline} & \textbf{With Policy} & \textbf{Assessment} \\
\midrule
Geographic (Gini) & \TODO{value} & \TODO{value} & \TODO{Improved/Worsened} \\
Income-based (Theil index) & \TODO{value} & \TODO{value} & \TODO{Assessment} \\
Demographic (\TODO{specific group}) & \TODO{gap} & \TODO{gap} & \TODO{Assessment} \\
Intergenerational & \TODO{metric} & \TODO{metric} & \TODO{Assessment} \\
\bottomrule
\end{tabular}
\end{table}

\textbf{Vulnerable Group Analysis:}
\TODO{Discuss how the policy affects the most vulnerable populations. Identify any groups that may be inadvertently harmed and propose mitigations.}

%=== 9.5 实施可行性 ===
\subsection{Implementation Feasibility}
\label{sec:impacts_implementation}

Translating model recommendations into practice requires attention to:

\textbf{1. Political Feasibility}
\begin{itemize}
    \item \textbf{Stakeholder Alignment:} \TODO{Which stakeholders support/oppose?}
    \item \textbf{Political Capital Required:} \TODO{Low/Medium/High}
    \item \textbf{Coalition Strategy:} \TODO{Recommended approach to build support}
\end{itemize}

\textbf{2. Administrative Capacity}
\begin{itemize}
    \item \textbf{Required Capabilities:} \TODO{List key implementation capacities}
    \item \textbf{Capacity Gaps:} \TODO{Identify potential bottlenecks}
    \item \textbf{Phased Rollout:} We recommend starting with \TODO{pilot regions/sectors} before national scale-up.
\end{itemize}

\textbf{3. Monitoring and Evaluation}
\begin{itemize}
    \item \textbf{Key Performance Indicators (KPIs):} \TODO{List 3-5 measurable indicators}
    \item \textbf{Data Collection:} \TODO{Describe monitoring infrastructure needs}
    \item \textbf{Adaptive Management:} We recommend policy review at \TODO{interval} with pre-specified adjustment triggers.
\end{itemize}

%=== 9.6 局限性讨论 ===
\subsection{Limitations}
\label{sec:impacts_limitations}

We acknowledge the following limitations in our analysis:

\begin{enumerate}
    \item \textbf{Model Simplifications:} \TODO{e.g., ``Assumed homogeneous behavioral response across regions''}.
    
    \item \textbf{Data Constraints:} \TODO{e.g., ``Limited historical data for parameter estimation''}.
    
    \item \textbf{Scope Boundaries:} \TODO{e.g., ``Did not model international spillovers''}.
    
    \item \textbf{Uncertainty Quantification:} \TODO{e.g., ``Deep uncertainty in long-term projections beyond Year X''}.
\end{enumerate}

These limitations suggest caution in \TODO{specific contexts} and point to future research needs (\secref{sec:conclusion}).
