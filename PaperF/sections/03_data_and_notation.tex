% 03_data_and_notation.tex - 数据与符号说明
% ICM Problem F - Policy Science

\section{Data Description and Notation}
\label{sec:data}

% 【写作指导】F题数据部分需要:
% 1. 清晰说明数据来源的权威性(政府统计、学术数据库等)
% 2. 展示数据预处理流程
% 3. 建立统一的符号体系

%=== 3.1 数据来源 ===
\subsection{Data Sources}
\label{sec:data_sources}

Our analysis draws on multiple authoritative data sources to ensure reliability and cross-validation:

\begin{table}[H]
\centering
\caption{Primary Data Sources}
\label{tab:data_sources}
\small
\begin{tabular}{p{3cm} p{4cm} p{2.5cm} p{2.5cm} p{2cm}}
\toprule
\textbf{Dataset} & \textbf{Description} & \textbf{Source} & \textbf{Coverage} & \textbf{Granularity} \\
\midrule
\TODO{Dataset 1} & \TODO{e.g., Population demographics} & \TODO{e.g., Census Bureau} & \TODO{2015-2024} & \TODO{Regional} \\
\TODO{Dataset 2} & \TODO{e.g., Policy implementation records} & \TODO{e.g., Government reports} & \TODO{Coverage} & \TODO{Level} \\
\TODO{Dataset 3} & \TODO{e.g., Economic indicators} & \TODO{e.g., World Bank} & \TODO{Coverage} & \TODO{Level} \\
\TODO{Dataset 4} & \TODO{e.g., Survey data} & \TODO{e.g., Academic study} & \TODO{Coverage} & \TODO{Level} \\
\bottomrule
\end{tabular}
\end{table}

% 数据质量评估
\textbf{Data Quality Assessment.} We evaluate each source against the following criteria:

\begin{itemize}
    \item \textbf{Completeness:} \TODO{e.g., ``Dataset 1 has <2\% missing values after preprocessing''}
    \item \textbf{Consistency:} \TODO{e.g., ``Cross-validated against alternative sources''}
    \item \textbf{Timeliness:} \TODO{e.g., ``Most recent data from 2023''}
    \item \textbf{Relevance:} \TODO{e.g., ``Directly measures target outcomes''}
\end{itemize}

%=== 3.2 数据预处理 ===
\subsection{Data Preprocessing}
\label{sec:data_preprocessing}

We apply a standardized preprocessing pipeline to ensure data quality and compatibility:

\begin{figure}[H]
    \centering
    \begin{tikzpicture}[
        node distance=1.5cm,
        box/.style={rectangle, draw, rounded corners, minimum width=2.5cm, minimum height=0.8cm, align=center, font=\small},
        arrow/.style={->, >=stealth, thick}
    ]
        % 流程节点
        \node[box, fill=blue!10] (raw) {Raw Data\\Collection};
        \node[box, fill=orange!10, right=of raw] (clean) {Missing Value\\Imputation};
        \node[box, fill=green!10, right=of clean] (normal) {Normalization\\Standardization};
        \node[box, fill=purple!10, right=of normal] (feature) {Feature\\Engineering};
        \node[box, fill=red!10, right=of feature] (final) {Analysis-Ready\\Dataset};
        
        % 箭头
        \draw[arrow] (raw) -- (clean);
        \draw[arrow] (clean) -- (normal);
        \draw[arrow] (normal) -- (feature);
        \draw[arrow] (feature) -- (final);
    \end{tikzpicture}
    \caption{Data preprocessing pipeline}
    \label{fig:preprocessing}
\end{figure}

\textbf{Key Preprocessing Steps:}

\begin{enumerate}
    \item \textbf{Missing Value Treatment:}
    \begin{itemize}
        \item Continuous variables: \TODO{e.g., Multiple imputation using chained equations (MICE)}
        \item Categorical variables: \TODO{e.g., Mode imputation with uncertainty flags}
    \end{itemize}
    
    \item \textbf{Outlier Detection:}
    \begin{itemize}
        \item Method: \TODO{e.g., IQR-based detection with domain expert validation}
        \item Treatment: \TODO{e.g., Winsorization at 1st/99th percentiles}
    \end{itemize}
    
    \item \textbf{Variable Transformation:}
    \begin{itemize}
        \item Skewed distributions: \TODO{e.g., Log transformation for income variables}
        \item Scale standardization: \TODO{e.g., Z-score normalization for comparability}
    \end{itemize}
\end{enumerate}

%=== 3.3 探索性数据分析 ===
\subsection{Exploratory Data Analysis}
\label{sec:data_eda}

% 描述性统计
\textbf{Summary Statistics.} \tabref{tab:descriptive} presents descriptive statistics for key variables:

\begin{table}[H]
\centering
\caption{Descriptive Statistics of Key Variables}
\label{tab:descriptive}
\small
\begin{tabular}{l c c c c c c}
\toprule
\textbf{Variable} & \textbf{N} & \textbf{Mean} & \textbf{Std Dev} & \textbf{Min} & \textbf{Max} & \textbf{Unit} \\
\midrule
\TODO{Variable 1} & \TODO{N} & \TODO{Mean} & \TODO{SD} & \TODO{Min} & \TODO{Max} & \TODO{Unit} \\
\TODO{Variable 2} & \TODO{N} & \TODO{Mean} & \TODO{SD} & \TODO{Min} & \TODO{Max} & \TODO{Unit} \\
\TODO{Variable 3} & \TODO{N} & \TODO{Mean} & \TODO{SD} & \TODO{Min} & \TODO{Max} & \TODO{Unit} \\
\TODO{Variable 4} & \TODO{N} & \TODO{Mean} & \TODO{SD} & \TODO{Min} & \TODO{Max} & \TODO{Unit} \\
\bottomrule
\end{tabular}
\end{table}

% 相关性分析
\textbf{Correlation Analysis.} We examine pairwise correlations to identify potential multicollinearity and causal pathway candidates:

\begin{figure}[H]
    \centering
    % \includegraphics[width=0.6\textwidth]{figures/correlation_heatmap.pdf}
    \fbox{\parbox{0.55\textwidth}{\centering\vspace{4em}\TODO{Figure: Correlation Heatmap}\vspace{4em}}}
    \caption{Correlation matrix of policy-relevant variables}
    \label{fig:correlation}
\end{figure}

%=== 3.4 符号与约定 ===
\subsection{Notation and Conventions}
\label{sec:data_notation}

To facilitate clear communication, we establish the following notation conventions:

\begin{table}[H]
\centering
\caption{Notation System}
\label{tab:notation}
\begin{tabular}{c l p{8cm}}
\toprule
\textbf{Symbol} & \textbf{Type} & \textbf{Description} \\
\midrule
\multicolumn{3}{l}{\textit{Index Sets}} \\
$i \in \mathcal{I}$ & Set & Set of regions/units, $|\mathcal{I}| = I$ \\
$t \in \mathcal{T}$ & Set & Time periods, $\mathcal{T} = \{1, 2, \ldots, T\}$ \\
$k \in \mathcal{K}$ & Set & Stakeholder groups \\
\midrule
\multicolumn{3}{l}{\textit{Decision Variables}} \\
$\policy_{i,t}$ & Variable & Policy intensity for unit $i$ at time $t$ \\
$x_{i,t}$ & Variable & Allocation decision \\
\midrule
\multicolumn{3}{l}{\textit{State Variables}} \\
$\stock_{i,t}$ & Variable & System state (stock) for unit $i$ at time $t$ \\
$\outcome_{i,t}$ & Variable & Outcome measure \\
\midrule
\multicolumn{3}{l}{\textit{Parameters}} \\
$\alpha, \beta$ & Parameter & Model coefficients (estimated) \\
$\lambda$ & Parameter & Feedback strength \\
$\discount$ & Parameter & Discount factor, $\discount \in (0,1)$ \\
\midrule
\multicolumn{3}{l}{\textit{Functions}} \\
$\utility_k(\cdot)$ & Function & Utility function of stakeholder $k$ \\
$\objective(\cdot)$ & Function & Social welfare objective \\
$f(\cdot), g(\cdot)$ & Function & System dynamics functions \\
\bottomrule
\end{tabular}
\end{table}

% 约定
\textbf{Conventions:}
\begin{itemize}
    \item Boldface denotes vectors and matrices: $\mathbf{x} = (x_1, \ldots, x_n)^\top$
    \item Subscripts denote indices; superscripts denote iterations or optimality ($x^*$)
    \item Time indices follow the convention: $t=0$ is baseline, $t=1$ is first policy period
    \item All monetary values are in \TODO{currency, e.g., 2023 USD} unless otherwise noted
\end{itemize}
