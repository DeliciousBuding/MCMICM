% 01_introduction.tex - Introduction (引言)
% ICM Problem F - 政策背景与问题重述

\section{Introduction}
\label{sec:intro}

% 【写作指导】F题引言需建立政策问题的紧迫性和重要性
% 结构:政策背景 → 问题痛点 → 文献缺口 → 本文贡献

%=== 1.1 政策背景与问题陈述 ===
\subsection{Policy Background and Problem Statement}
\label{sec:intro_background}

% 背景段落1:宏观政策情境
\TODO{Describe the broader policy context. Example: ``In recent years, governments worldwide have faced increasing pressure to address [policy challenge]. The [specific domain, e.g., healthcare, education, environment] sector has experienced significant shifts due to [driving factors], creating both opportunities and challenges for policymakers.''}

% 背景段落2:问题具体化
The specific challenge we address is \TODO{concise problem statement in 1-2 sentences}. This problem manifests across multiple dimensions:

\begin{itemize}
    \item \textbf{Scale:} \TODO{e.g., ``Affecting X million individuals across Y regions''}
    \item \textbf{Urgency:} \TODO{e.g., ``Current trajectory suggests Z\% degradation by [year]''}
    \item \textbf{Complexity:} \TODO{e.g., ``Involves trade-offs between competing stakeholder interests''}
\end{itemize}

% 背景段落3:量化问题严重性
To quantify the severity of this challenge, consider the following statistics:
% TODO: 插入关键统计数据或引用

\begin{figure}[H]
    \centering
    % \includegraphics[width=0.85\textwidth]{figures/problem_overview.pdf}
    \fbox{\parbox{0.8\textwidth}{\centering\vspace{3em}\TODO{Figure 1: Problem Overview Infographic}\vspace{3em}}}
    \caption{Overview of the policy challenge: \TODO{caption description}}
    \label{fig:problem_overview}
\end{figure}

%=== 1.2 文献回顾与研究缺口 ===
\subsection{Literature Review and Research Gap}
\label{sec:intro_literature}

% 文献分类综述
Previous research on \TODO{topic} can be categorized into three main streams:

\textbf{Stream 1: \TODO{Category Name, e.g., Theoretical Foundations}}

\TODO{2-3 sentences summarizing key works in this category, with citations like \cite{example2023policy}. Focus on what has been established.}

\textbf{Stream 2: \TODO{Category Name, e.g., Empirical Evidence}}

\TODO{2-3 sentences summarizing empirical studies. Highlight methodologies used (RCTs, quasi-experiments, observational studies).}

\textbf{Stream 3: \TODO{Category Name, e.g., Policy Modeling Approaches}}

\TODO{2-3 sentences on existing modeling approaches (system dynamics, agent-based, optimization). Identify their strengths and limitations.}

\vspace{0.5em}
\noindent\textbf{Research Gap.} Despite these contributions, the literature exhibits several gaps:

\begin{enumerate}[label=(\arabic*)]
    \item \TODO{Gap 1: e.g., ``Limited integration of mechanism modeling with policy optimization''}
    \item \TODO{Gap 2: e.g., ``Insufficient attention to distributional impacts and equity''}
    \item \TODO{Gap 3: e.g., ``Lack of robustness analysis under uncertainty''}
\end{enumerate}

%=== 1.3 本文贡献 ===
\subsection{Our Contributions}
\label{sec:intro_contributions}

This paper addresses the identified gaps through a \textbf{policy-driven modeling framework} that integrates mechanism analysis, optimization, and robust evaluation. Our specific contributions include:

\begin{enumerate}[label=\textbf{C\arabic*:}, leftmargin=2.5em]
    \item \textbf{System Mechanism Model.} We develop a \TODO{e.g., system dynamics / causal inference} model that captures the pathways through which policy interventions generate outcomes, enabling scenario analysis across varying policy intensities (\secref{sec:model1}).
    
    \item \textbf{Policy Optimization Framework.} We formulate a \TODO{e.g., multi-objective / robust} optimization problem that balances \TODO{competing objectives} subject to practical constraints, yielding Pareto-optimal policy portfolios (\secref{sec:model2}).
    
    \item \textbf{Spillover and Stability Analysis.} Beyond direct impacts, we analyze how policies propagate through interconnected systems, assessing \textbf{cross-domain spillovers} and \textbf{long-term stability} (\secref{sec:impacts}).
    
    \item \textbf{Actionable Policy Recommendations.} We translate quantitative findings into a structured \textbf{Policy Memorandum} with tiered, implementable recommendations aligned with stakeholder capacities.
\end{enumerate}

%=== 1.4 论文结构 ===
\subsection{Paper Organization}
\label{sec:intro_organization}

The remainder of this paper is organized as follows:

\begin{itemize}[leftmargin=*]
    \item \textbf{\secref{sec:assumptions}:} Defines the policy scope, formalizes assumptions, and establishes analytical boundaries.
    \item \textbf{\secref{sec:data}:} Describes data sources, preprocessing, and the notation system.
    \item \textbf{\secref{sec:model1}:} Develops the system mechanism model capturing causal pathways.
    \item \textbf{\secref{sec:model2}:} Formulates the policy optimization problem and solution approach.
    \item \textbf{\secref{sec:algorithm}:} Presents the computational algorithm with complexity analysis.
    \item \textbf{\secref{sec:results}:} Reports results and validates model predictions.
    \item \textbf{\secref{sec:sensitivity}:} Conducts sensitivity and robustness analysis.
    \item \textbf{\secref{sec:impacts}:} Discusses policy impacts, spillovers, and implementation considerations.
    \item \textbf{\secref{sec:strengths}:} Evaluates model strengths and limitations.
    \item \textbf{\secref{sec:conclusion}:} Summarizes findings and outlines future directions.
\end{itemize}
