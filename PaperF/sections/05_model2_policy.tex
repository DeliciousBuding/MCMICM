% 05_model2_policy.tex - 政策优化模型
% ICM Problem F - Policy Science

\section{Model II: Policy Optimization Framework}
\label{sec:model2}

% 【写作指导】F题政策优化需要:
% 1. 多目标平衡(效率、公平、可持续性)
% 2. 现实约束(预算、容量、政治可行性)
% 3. 鲁棒性考量(不确定性下的决策)

%=== 5.1 优化问题构建 ===
\subsection{Problem Formulation}
\label{sec:model2_formulation}

Building on the mechanism model (\secref{sec:model1}), we formulate a \textbf{multi-objective optimization} problem that identifies Pareto-optimal policy portfolios.

%--- 5.1.1 目标函数 ---
\subsubsection{Objective Functions}

Effective policy must balance multiple, potentially conflicting objectives. We consider:

\textbf{Objective 1: \TODO{e.g., Social Welfare Maximization}}
\begin{equation}
\objective_1 = \max_{\boldsymbol{\policy}} \sum_{t=1}^{T} \discount^{t-1} \cdot \welfare(\outcome_t, \mathbf{X}_t)
\label{eq:obj1}
\end{equation}

where $\welfare(\cdot)$ is the social welfare function aggregating individual utilities, and $\discount$ is the social discount factor.

\textbf{Objective 2: \TODO{e.g., Equity / Distributional Fairness}}
\begin{equation}
\objective_2 = \min_{\boldsymbol{\policy}} \; \text{Gini}(\outcome_{i,T}) = 1 - 2 \int_0^1 L(p) \, dp
\label{eq:obj2}
\end{equation}

where $L(p)$ is the Lorenz curve of outcomes across units.

\textbf{Objective 3: \TODO{e.g., Cost-Effectiveness / Budget Efficiency}}
\begin{equation}
\objective_3 = \min_{\boldsymbol{\policy}} \sum_{i,t} c_{i,t} \cdot \policy_{i,t}
\label{eq:obj3}
\end{equation}

%--- 5.1.2 约束条件 ---
\subsubsection{Constraints}

Policy decisions are subject to real-world constraints:

\textbf{Budget Constraint:}
\begin{equation}
\sum_{i \in \mathcal{I}} \sum_{t=1}^{T} c_{i,t} \cdot \policy_{i,t} \leq B_{\text{total}}
\label{eq:budget}
\end{equation}

\textbf{Capacity Constraint:}
\begin{equation}
\policy_{i,t} \leq \bar{\policy}_i \quad \forall i \in \mathcal{I}, \; t \in \mathcal{T}
\label{eq:capacity}
\end{equation}

\textbf{Temporal Smoothness (avoid drastic policy changes):}
\begin{equation}
|\policy_{i,t+1} - \policy_{i,t}| \leq \Delta_{\max} \quad \forall i, t
\label{eq:smoothness}
\end{equation}

\textbf{Non-negativity:}
\begin{equation}
\policy_{i,t} \geq 0 \quad \forall i \in \mathcal{I}, \; t \in \mathcal{T}
\label{eq:nonnegativity}
\end{equation}

%--- 5.1.3 综合优化问题 ---
\subsubsection{Integrated Optimization Problem}

The complete multi-objective optimization problem is:

\begin{equation}
\begin{aligned}
& \underset{\boldsymbol{\policy}}{\text{optimize}}
& & \left( \objective_1(\boldsymbol{\policy}), \; \objective_2(\boldsymbol{\policy}), \; \objective_3(\boldsymbol{\policy}) \right) \\
& \text{subject to}
& & \text{Budget constraint \eqnref{eq:budget}} \\
& & & \text{Capacity constraint \eqnref{eq:capacity}} \\
& & & \text{Smoothness constraint \eqnref{eq:smoothness}} \\
& & & \text{Non-negativity \eqnref{eq:nonnegativity}} \\
& & & \text{System dynamics \eqnref{eq:state_evolution}--\eqnref{eq:outcome}}
\end{aligned}
\label{eq:moop}
\end{equation}

%=== 5.2 权衡分析框架 ===
\subsection{Trade-off Analysis Framework}
\label{sec:model2_tradeoff}

Given multiple objectives, we employ the \textbf{$\epsilon$-constraint method} combined with \textbf{Pareto frontier visualization} to explore trade-offs.

\textbf{$\epsilon$-Constraint Formulation:}

Fix objectives 2 and 3 as constraints, optimize objective 1:
\begin{equation}
\begin{aligned}
& \max_{\boldsymbol{\policy}} && \objective_1(\boldsymbol{\policy}) \\
& \text{s.t.} && \objective_2(\boldsymbol{\policy}) \leq \epsilon_2 \\
& && \objective_3(\boldsymbol{\policy}) \leq \epsilon_3 \\
& && \text{Original constraints}
\end{aligned}
\label{eq:epsilon_constraint}
\end{equation}

By systematically varying $(\epsilon_2, \epsilon_3)$, we trace the Pareto frontier.

\begin{figure}[H]
    \centering
    % \includegraphics[width=0.7\textwidth]{figures/pareto_frontier.pdf}
    \fbox{\parbox{0.65\textwidth}{\centering\vspace{4em}\TODO{Figure: Pareto Frontier showing Efficiency-Equity Trade-off}\vspace{4em}}}
    \caption{Pareto frontier illustrating trade-offs between efficiency ($\objective_1$) and equity ($\objective_2$)}
    \label{fig:pareto}
\end{figure}

%=== 5.3 鲁棒优化扩展 ===
\subsection{Robust Optimization Extension}
\label{sec:model2_robust}

To account for parameter uncertainty, we extend the formulation to a \textbf{robust optimization} framework.

\textbf{Uncertainty Set:}
Let $\mathcal{U}$ denote the uncertainty set for key parameters:
\begin{equation}
\mathcal{U} = \left\{ \boldsymbol{\theta} : \| \boldsymbol{\theta} - \hat{\boldsymbol{\theta}} \|_2 \leq \Gamma \right\}
\label{eq:uncertainty_set}
\end{equation}

where $\hat{\boldsymbol{\theta}}$ is the nominal parameter estimate and $\Gamma$ is the uncertainty budget.

\textbf{Robust Counterpart:}
\begin{equation}
\max_{\boldsymbol{\policy}} \min_{\boldsymbol{\theta} \in \mathcal{U}} \objective_1(\boldsymbol{\policy}; \boldsymbol{\theta})
\label{eq:robust}
\end{equation}

This ensures policy performance even under adverse parameter realizations.

%=== 5.4 政策情景设计 ===
\subsection{Policy Scenario Design}
\label{sec:model2_scenarios}

We evaluate policy performance across \TODO{N} representative scenarios:

\begin{table}[H]
\centering
\caption{Policy Scenarios for Comparative Analysis}
\label{tab:scenarios}
\begin{tabular}{l p{5cm} c c}
\toprule
\textbf{Scenario} & \textbf{Description} & \textbf{Policy Intensity} & \textbf{Budget} \\
\midrule
Baseline (S0) & Status quo, no new intervention & -- & \$0 \\
Conservative (S1) & \TODO{Low-intensity intervention} & Low & \TODO{Budget} \\
Moderate (S2) & \TODO{Balanced approach} & Medium & \TODO{Budget} \\
Aggressive (S3) & \TODO{High-intensity intervention} & High & \TODO{Budget} \\
Targeted (S4) & \TODO{Focused on high-impact units} & Variable & \TODO{Budget} \\
\bottomrule
\end{tabular}
\end{table}

Scenario analysis results are presented in \secref{sec:results}.
