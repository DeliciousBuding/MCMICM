% _shared_macros.tex - 共享宏定义文件
% 用于多人协作时统一符号和变量命名,避免冲突
% 使用说明:在 main.tex 中通过 % _shared_macros.tex - 共享宏定义文件
% 用于多人协作时统一符号和变量命名,避免冲突
% 使用说明:在 main.tex 中通过 % _shared_macros.tex - 共享宏定义文件
% 用于多人协作时统一符号和变量命名,避免冲突
% 使用说明:在 main.tex 中通过 % _shared_macros.tex - 共享宏定义文件
% 用于多人协作时统一符号和变量命名,避免冲突
% 使用说明:在 main.tex 中通过 \input{sections/_shared_macros.tex} 引入

%=== 政策分析专用符号 ===

% 政策变量
\newcommand{\policy}{\pi}                    % 政策方案
\newcommand{\policyspace}{\Pi}               % 政策空间
\newcommand{\intervention}{I}                % 干预措施
\newcommand{\baseline}{B}                    % 基准情景

% 效果评估
\newcommand{\outcome}{Y}                     % 结果变量
\newcommand{\treatment}{T}                   % 处理变量
\newcommand{\effect}{\tau}                   % 处理效应
\newcommand{\ate}{\bar{\tau}}               % 平均处理效应 (ATE)
\newcommand{\att}{\tau_{\text{ATT}}}        % 处理组平均效应

% 利益相关者
\newcommand{\stakeholder}{S}                 % 利益相关者
\newcommand{\utility}{U}                     % 效用函数
\newcommand{\welfare}{W}                     % 社会福利

% 时间与动态
\newcommand{\horizon}{T}                     % 规划期
\newcommand{\discount}{\delta}               % 折现因子
\newcommand{\lagtime}{\tau_{\text{lag}}}    % 政策时滞

% 不确定性
\newcommand{\uncertainty}{\epsilon}          % 不确定性项
\newcommand{\robust}{\mathcal{R}}           % 鲁棒性指标

%=== 模型专用符号 ===

% 系统动力学
\newcommand{\stock}{S}                       % 存量
\newcommand{\flow}{F}                        % 流量
\newcommand{\feedback}{\lambda}              % 反馈系数

% 优化
\newcommand{\objective}{\mathcal{J}}         % 目标函数
\newcommand{\constraint}{\mathcal{C}}        % 约束集
\newcommand{\optimal}{^*}                    % 最优解标记

% 评估指标
\newcommand{\efficiency}{\eta}               % 效率
\newcommand{\equity}{\xi}                    % 公平性
\newcommand{\sustainability}{\sigma}         % 可持续性

%=== 常用缩写 ===

\newcommand{\ie}{\textit{i.e.}}
\newcommand{\eg}{\textit{e.g.}}
\newcommand{\etc}{\textit{etc.}}
\newcommand{\wrt}{w.r.t.}
\newcommand{\st}{\text{s.t.}}

%=== 数学环境 ===

% 期望、概率
\newcommand{\E}{\mathbb{E}}                  % 期望
\newcommand{\Prob}{\mathbb{P}}               % 概率
\newcommand{\Var}{\text{Var}}                % 方差

% 集合
\newcommand{\R}{\mathbb{R}}                  % 实数集
\newcommand{\N}{\mathbb{N}}                  % 自然数集
\newcommand{\Z}{\mathbb{Z}}                  % 整数集

%=== 颜色定义(用于图表一致性)===

\definecolor{policyblue}{RGB}{31, 119, 180}
\definecolor{resultgreen}{RGB}{44, 160, 44}
\definecolor{warningorange}{RGB}{255, 127, 14}
\definecolor{riskred}{RGB}{214, 39, 40}
\definecolor{neutralgray}{RGB}{127, 127, 127}
 引入

%=== 政策分析专用符号 ===

% 政策变量
\newcommand{\policy}{\pi}                    % 政策方案
\newcommand{\policyspace}{\Pi}               % 政策空间
\newcommand{\intervention}{I}                % 干预措施
\newcommand{\baseline}{B}                    % 基准情景

% 效果评估
\newcommand{\outcome}{Y}                     % 结果变量
\newcommand{\treatment}{T}                   % 处理变量
\newcommand{\effect}{\tau}                   % 处理效应
\newcommand{\ate}{\bar{\tau}}               % 平均处理效应 (ATE)
\newcommand{\att}{\tau_{\text{ATT}}}        % 处理组平均效应

% 利益相关者
\newcommand{\stakeholder}{S}                 % 利益相关者
\newcommand{\utility}{U}                     % 效用函数
\newcommand{\welfare}{W}                     % 社会福利

% 时间与动态
\newcommand{\horizon}{T}                     % 规划期
\newcommand{\discount}{\delta}               % 折现因子
\newcommand{\lagtime}{\tau_{\text{lag}}}    % 政策时滞

% 不确定性
\newcommand{\uncertainty}{\epsilon}          % 不确定性项
\newcommand{\robust}{\mathcal{R}}           % 鲁棒性指标

%=== 模型专用符号 ===

% 系统动力学
\newcommand{\stock}{S}                       % 存量
\newcommand{\flow}{F}                        % 流量
\newcommand{\feedback}{\lambda}              % 反馈系数

% 优化
\newcommand{\objective}{\mathcal{J}}         % 目标函数
\newcommand{\constraint}{\mathcal{C}}        % 约束集
\newcommand{\optimal}{^*}                    % 最优解标记

% 评估指标
\newcommand{\efficiency}{\eta}               % 效率
\newcommand{\equity}{\xi}                    % 公平性
\newcommand{\sustainability}{\sigma}         % 可持续性

%=== 常用缩写 ===

\newcommand{\ie}{\textit{i.e.}}
\newcommand{\eg}{\textit{e.g.}}
\newcommand{\etc}{\textit{etc.}}
\newcommand{\wrt}{w.r.t.}
\newcommand{\st}{\text{s.t.}}

%=== 数学环境 ===

% 期望、概率
\newcommand{\E}{\mathbb{E}}                  % 期望
\newcommand{\Prob}{\mathbb{P}}               % 概率
\newcommand{\Var}{\text{Var}}                % 方差

% 集合
\newcommand{\R}{\mathbb{R}}                  % 实数集
\newcommand{\N}{\mathbb{N}}                  % 自然数集
\newcommand{\Z}{\mathbb{Z}}                  % 整数集

%=== 颜色定义(用于图表一致性)===

\definecolor{policyblue}{RGB}{31, 119, 180}
\definecolor{resultgreen}{RGB}{44, 160, 44}
\definecolor{warningorange}{RGB}{255, 127, 14}
\definecolor{riskred}{RGB}{214, 39, 40}
\definecolor{neutralgray}{RGB}{127, 127, 127}
 引入

%=== 政策分析专用符号 ===

% 政策变量
\newcommand{\policy}{\pi}                    % 政策方案
\newcommand{\policyspace}{\Pi}               % 政策空间
\newcommand{\intervention}{I}                % 干预措施
\newcommand{\baseline}{B}                    % 基准情景

% 效果评估
\newcommand{\outcome}{Y}                     % 结果变量
\newcommand{\treatment}{T}                   % 处理变量
\newcommand{\effect}{\tau}                   % 处理效应
\newcommand{\ate}{\bar{\tau}}               % 平均处理效应 (ATE)
\newcommand{\att}{\tau_{\text{ATT}}}        % 处理组平均效应

% 利益相关者
\newcommand{\stakeholder}{S}                 % 利益相关者
\newcommand{\utility}{U}                     % 效用函数
\newcommand{\welfare}{W}                     % 社会福利

% 时间与动态
\newcommand{\horizon}{T}                     % 规划期
\newcommand{\discount}{\delta}               % 折现因子
\newcommand{\lagtime}{\tau_{\text{lag}}}    % 政策时滞

% 不确定性
\newcommand{\uncertainty}{\epsilon}          % 不确定性项
\newcommand{\robust}{\mathcal{R}}           % 鲁棒性指标

%=== 模型专用符号 ===

% 系统动力学
\newcommand{\stock}{S}                       % 存量
\newcommand{\flow}{F}                        % 流量
\newcommand{\feedback}{\lambda}              % 反馈系数

% 优化
\newcommand{\objective}{\mathcal{J}}         % 目标函数
\newcommand{\constraint}{\mathcal{C}}        % 约束集
\newcommand{\optimal}{^*}                    % 最优解标记

% 评估指标
\newcommand{\efficiency}{\eta}               % 效率
\newcommand{\equity}{\xi}                    % 公平性
\newcommand{\sustainability}{\sigma}         % 可持续性

%=== 常用缩写 ===

\newcommand{\ie}{\textit{i.e.}}
\newcommand{\eg}{\textit{e.g.}}
\newcommand{\etc}{\textit{etc.}}
\newcommand{\wrt}{w.r.t.}
\newcommand{\st}{\text{s.t.}}

%=== 数学环境 ===

% 期望、概率
\newcommand{\E}{\mathbb{E}}                  % 期望
\newcommand{\Prob}{\mathbb{P}}               % 概率
\newcommand{\Var}{\text{Var}}                % 方差

% 集合
\newcommand{\R}{\mathbb{R}}                  % 实数集
\newcommand{\N}{\mathbb{N}}                  % 自然数集
\newcommand{\Z}{\mathbb{Z}}                  % 整数集

%=== 颜色定义(用于图表一致性)===

\definecolor{policyblue}{RGB}{31, 119, 180}
\definecolor{resultgreen}{RGB}{44, 160, 44}
\definecolor{warningorange}{RGB}{255, 127, 14}
\definecolor{riskred}{RGB}{214, 39, 40}
\definecolor{neutralgray}{RGB}{127, 127, 127}
 引入

%=== 政策分析专用符号 ===

% 政策变量
\newcommand{\policy}{\pi}                    % 政策方案
\newcommand{\policyspace}{\Pi}               % 政策空间
\newcommand{\intervention}{I}                % 干预措施
\newcommand{\baseline}{B}                    % 基准情景

% 效果评估
\newcommand{\outcome}{Y}                     % 结果变量
\newcommand{\treatment}{T}                   % 处理变量
\newcommand{\effect}{\tau}                   % 处理效应
\newcommand{\ate}{\bar{\tau}}               % 平均处理效应 (ATE)
\newcommand{\att}{\tau_{\text{ATT}}}        % 处理组平均效应

% 利益相关者
\newcommand{\stakeholder}{S}                 % 利益相关者
\newcommand{\utility}{U}                     % 效用函数
\newcommand{\welfare}{W}                     % 社会福利

% 时间与动态
\newcommand{\horizon}{T}                     % 规划期
\newcommand{\discount}{\delta}               % 折现因子
\newcommand{\lagtime}{\tau_{\text{lag}}}    % 政策时滞

% 不确定性
\newcommand{\uncertainty}{\epsilon}          % 不确定性项
\newcommand{\robust}{\mathcal{R}}           % 鲁棒性指标

%=== 模型专用符号 ===

% 系统动力学
\newcommand{\stock}{S}                       % 存量
\newcommand{\flow}{F}                        % 流量
\newcommand{\feedback}{\lambda}              % 反馈系数

% 优化
\newcommand{\objective}{\mathcal{J}}         % 目标函数
\newcommand{\constraint}{\mathcal{C}}        % 约束集
\newcommand{\optimal}{^*}                    % 最优解标记

% 评估指标
\newcommand{\efficiency}{\eta}               % 效率
\newcommand{\equity}{\xi}                    % 公平性
\newcommand{\sustainability}{\sigma}         % 可持续性

%=== 常用缩写 ===

\newcommand{\ie}{\textit{i.e.}}
\newcommand{\eg}{\textit{e.g.}}
\newcommand{\etc}{\textit{etc.}}
\newcommand{\wrt}{w.r.t.}
\newcommand{\st}{\text{s.t.}}

%=== 数学环境 ===

% 期望、概率
\newcommand{\E}{\mathbb{E}}                  % 期望
\newcommand{\Prob}{\mathbb{P}}               % 概率
\newcommand{\Var}{\text{Var}}                % 方差

% 集合
\newcommand{\R}{\mathbb{R}}                  % 实数集
\newcommand{\N}{\mathbb{N}}                  % 自然数集
\newcommand{\Z}{\mathbb{Z}}                  % 整数集

%=== 颜色定义(用于图表一致性)===

\definecolor{policyblue}{RGB}{31, 119, 180}
\definecolor{resultgreen}{RGB}{44, 160, 44}
\definecolor{warningorange}{RGB}{255, 127, 14}
\definecolor{riskred}{RGB}{214, 39, 40}
\definecolor{neutralgray}{RGB}{127, 127, 127}
