% 06_solution_and_algorithm.tex - 求解算法
% ICM Problem F - Policy Science

\section{Solution Algorithm}
\label{sec:algorithm}

% 【写作指导】算法部分需要:
% 1. 清晰的算法流程描述
% 2. 复杂度分析
% 3. 收敛性/最优性讨论
% 4. 实现细节(便于复现)

%=== 6.1 算法设计思路 ===
\subsection{Algorithm Design Rationale}
\label{sec:algorithm_rationale}

The optimization problem \eqnref{eq:moop} presents several computational challenges:

\begin{itemize}
    \item \textbf{Non-convexity:} The objective functions involve nonlinear system dynamics.
    \item \textbf{Multi-objective nature:} No single optimal solution exists; we seek the Pareto frontier.
    \item \textbf{Large solution space:} $|\mathcal{I}| \times T$ decision variables with interdependencies.
\end{itemize}

To address these challenges, we develop a \textbf{\TODO{e.g., Hybrid Evolutionary-Local Search / Decomposition-Based / Simulation-Optimization}} algorithm.

%=== 6.2 算法流程 ===
\subsection{Algorithm Description}
\label{sec:algorithm_description}

\begin{algorithm}[H]
\caption{Policy Optimization Algorithm (\TODO{Algorithm Name})}
\label{alg:main}
\KwIn{Initial population $\mathcal{P}_0$, parameters $\{\text{pop\_size}, \text{max\_gen}, \text{crossover\_rate}, \text{mutation\_rate}\}$}
\KwOut{Pareto-optimal policy set $\mathcal{P}^*$}

\BlankLine
\tcp{Initialization}
$\mathcal{P} \leftarrow$ InitializePopulation(pop\_size)\;
EvaluateFitness($\mathcal{P}$) using mechanism model (\secref{sec:model1})\;
$\mathcal{A} \leftarrow \emptyset$ \tcp{Archive for non-dominated solutions}

\BlankLine
\tcp{Main evolutionary loop}
\For{$g = 1$ \KwTo max\_gen}{
    \tcp{Selection}
    $\mathcal{M} \leftarrow$ TournamentSelection($\mathcal{P}$, based on crowding distance)\;
    
    \tcp{Crossover and Mutation}
    $\mathcal{O} \leftarrow \emptyset$\;
    \For{each pair $(\boldsymbol{\policy}_1, \boldsymbol{\policy}_2)$ in $\mathcal{M}$}{
        \If{rand() $<$ crossover\_rate}{
            $(\boldsymbol{\policy}_1', \boldsymbol{\policy}_2') \leftarrow$ SBXCrossover($\boldsymbol{\policy}_1, \boldsymbol{\policy}_2$)\;
        }
        $\boldsymbol{\policy}_1' \leftarrow$ PolynomialMutation($\boldsymbol{\policy}_1'$, mutation\_rate)\;
        RepairConstraints($\boldsymbol{\policy}_1'$) \tcp{Ensure feasibility}
        $\mathcal{O} \leftarrow \mathcal{O} \cup \{\boldsymbol{\policy}_1'\}$\;
    }
    
    \tcp{Evaluation}
    EvaluateFitness($\mathcal{O}$)\;
    
    \tcp{Environmental Selection}
    $\mathcal{P} \leftarrow$ NSGA-II-Selection($\mathcal{P} \cup \mathcal{O}$, pop\_size)\;
    
    \tcp{Archive Update}
    $\mathcal{A} \leftarrow$ UpdateArchive($\mathcal{A}$, non-dominated solutions from $\mathcal{P}$)\;
    
    \tcp{Convergence Check}
    \If{ConvergenceCriterion($\mathcal{A}$)}{
        \textbf{break}\;
    }
}

\BlankLine
\tcp{Local Search Refinement}
$\mathcal{P}^* \leftarrow$ LocalSearch($\mathcal{A}$) \tcp{Gradient-based refinement}

\Return $\mathcal{P}^*$\;
\end{algorithm}

%=== 6.3 算法流程图 ===
\subsection{Algorithm Flowchart}
\label{sec:algorithm_flowchart}

\begin{figure}[H]
    \centering
    \begin{tikzpicture}[
        node distance=1.2cm,
        startstop/.style={rectangle, rounded corners, minimum width=3cm, minimum height=0.8cm, text centered, draw=black, fill=red!20},
        process/.style={rectangle, minimum width=3.5cm, minimum height=0.8cm, text centered, draw=black, fill=blue!10},
        decision/.style={diamond, aspect=2, minimum width=2cm, minimum height=0.8cm, text centered, draw=black, fill=green!10},
        arrow/.style={thick,->,>=stealth}
    ]
        % 节点
        \node (start) [startstop] {Start};
        \node (init) [process, below=of start] {Initialize Population};
        \node (eval) [process, below=of init] {Evaluate Fitness};
        \node (select) [process, below=of eval] {Selection \& Reproduction};
        \node (update) [process, below=of select] {Update Population};
        \node (converge) [decision, below=of update] {Converged?};
        \node (local) [process, below=of converge, yshift=-0.5cm] {Local Search Refinement};
        \node (end) [startstop, below=of local] {Output Pareto Set};
        
        % 箭头
        \draw [arrow] (start) -- (init);
        \draw [arrow] (init) -- (eval);
        \draw [arrow] (eval) -- (select);
        \draw [arrow] (select) -- (update);
        \draw [arrow] (update) -- (converge);
        \draw [arrow] (converge) -- node[anchor=east] {Yes} (local);
        \draw [arrow] (converge.east) -- ++(1.5,0) |- node[anchor=south, pos=0.25] {No} (eval.east);
        \draw [arrow] (local) -- (end);
    \end{tikzpicture}
    \caption{Flowchart of the policy optimization algorithm}
    \label{fig:flowchart}
\end{figure}

%=== 6.4 关键子程序 ===
\subsection{Key Subroutines}
\label{sec:algorithm_subroutines}

\textbf{Constraint Repair.} When genetic operators produce infeasible solutions, we apply:
\begin{equation}
\policy_{i,t}^{\text{repaired}} = \min\left( \max(\policy_{i,t}, 0), \bar{\policy}_i \right)
\label{eq:repair}
\end{equation}

\textbf{Crowding Distance.} To maintain solution diversity:
\begin{equation}
\text{CD}(\boldsymbol{\policy}) = \sum_{m=1}^{M} \frac{f_m^{(i+1)} - f_m^{(i-1)}}{f_m^{\max} - f_m^{\min}}
\label{eq:crowding}
\end{equation}

\textbf{Local Search.} For archive solutions, we apply gradient-based refinement:
\begin{equation}
\boldsymbol{\policy}^{(k+1)} = \boldsymbol{\policy}^{(k)} + \eta \cdot \nabla_{\boldsymbol{\policy}} \objective(\boldsymbol{\policy}^{(k)})
\label{eq:local_search}
\end{equation}

%=== 6.5 复杂度分析 ===
\subsection{Complexity Analysis}
\label{sec:algorithm_complexity}

\begin{table}[H]
\centering
\caption{Computational Complexity Analysis}
\label{tab:complexity}
\begin{tabular}{l l l}
\toprule
\textbf{Component} & \textbf{Complexity} & \textbf{Remarks} \\
\midrule
Fitness evaluation & $O(I \cdot T)$ & Per individual \\
Non-dominated sorting & $O(M \cdot N^2)$ & $M$ objectives, $N$ population \\
Crowding distance & $O(M \cdot N \log N)$ & Sorting required \\
Local search & $O(K \cdot I \cdot T)$ & $K$ iterations \\
\midrule
\textbf{Total per generation} & $O(N \cdot I \cdot T + M \cdot N^2)$ & \\
\bottomrule
\end{tabular}
\end{table}

\textbf{Convergence.} Empirically, the algorithm converges within \TODO{G} generations for problem instances with $|\mathcal{I}| \leq \TODO{I}$ and $T \leq \TODO{T}$. Formal convergence guarantees follow from standard evolutionary algorithm theory \cite{rudolph1998convergence}.

%=== 6.6 实现细节 ===
\subsection{Implementation Details}
\label{sec:algorithm_implementation}

\begin{itemize}
    \item \textbf{Programming Language:} Python 3.10 with NumPy, SciPy
    \item \textbf{Parallelization:} Fitness evaluations parallelized across \TODO{N} CPU cores
    \item \textbf{Parameter Settings:}
    \begin{itemize}
        \item Population size: \TODO{100}
        \item Maximum generations: \TODO{500}
        \item Crossover rate: \TODO{0.9}
        \item Mutation rate: \TODO{1/n} (where $n$ = number of decision variables)
    \end{itemize}
    \item \textbf{Runtime:} Approximately \TODO{X} minutes on \TODO{hardware specification}
\end{itemize}

Source code is provided in Appendix \ref{app:code}.
