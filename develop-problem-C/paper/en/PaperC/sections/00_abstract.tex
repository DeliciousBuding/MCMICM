% Abstract / Summary Sheet

\begin{abstract}

\textbf{Motivated by recurring judge--fan disagreements, we frame \emph{Dancing with the Stars} (DWTS) as an audit problem:} can published rules explain eliminations without observing fan vote shares'

We build a \textbf{Dual-Core Inversion Engine}: LP recovers feasible fan-vote intervals for percent seasons (S3--S27), while MILP infers latent fan ranks for rank seasons (S1--S2, S28--S34). On top of these intervals, we construct a \textbf{truncated Bayesian posterior} with MCMC and temporal smoothness; only samples that reproduce the observed eliminations are accepted.

Our posterior reconstruction yields 2,250 contestant-week estimates across 264 elimination weeks. Counterfactual evaluation reports mean Kendall's Tau $\approx 0.455$ and reversal rate $\approx 0.174$, indicating stable but nontrivial fan--judge tension. Feature attribution via \textbf{XGBoost + SHAP} shows judge-score statistics as the dominant survival driver, with partner effects and nonlinear age contributions.

For mechanism design, a \textbf{dynamic adaptive weighting} (judge-heavy early, fan-heavy late) is recommended. The Pareto frontier highlights the 60/40 rule as a balanced static point, while the dynamic schedule preserves audience influence without sacrificing fairness signals.

\begin{keywords}
Bayesian inference; MCMC; linear programming; mixed-integer programming; XGBoost; SHAP; mechanism design
\end{keywords}

\end{abstract}
