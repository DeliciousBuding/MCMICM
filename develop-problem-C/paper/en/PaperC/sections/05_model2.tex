% Section 5: Model 2 - Counterfactual Mechanism Evaluation

\section{Model 2: Counterfactual Mechanism Evaluation}
\label{sec:model2}

\noindent\textbf{\textit{This section addresses Task 2:}} We keep inferred fan intent and judge scores fixed, then replay eliminations under alternative weighting rules to quantify fairness and stability.

\subsection{Metrics}
We evaluate each counterfactual rule using three complementary metrics:
\begin{itemize}[itemsep=0.2em]
    \item \textbf{Kendall's Tau:} rank agreement between fan intent and mechanism outcome.
    \item \textbf{Reversal Rate:} fraction of weeks where the mechanism eliminates a different contestant than the fan-minimum.
    \item \textbf{Condorcet Consistency:} whether the mechanism avoids eliminating the fan winner.
\end{itemize}

\subsection{Results}
\begin{table}[H]
    \centering
    \caption{Counterfactual Evaluation Summary}
    \label{tab:counterfactual_summary}
    \begin{tabular}{lccc}
        \toprule
        \textbf{Rule} & \textbf{Kendall's Tau} & \textbf{Reversal Rate} & \textbf{Condorcet Consistency} \\
        \midrule
        Fixed 60/40 ($\alpha=0.6$) & 0.455 & 0.174 & 1.000 \\
        Dynamic Weighting & 0.409 & 0.174 & 1.000 \\
        \bottomrule
    \end{tabular}
\end{table}

\noindent\textbf{Interpretation.} The dynamic schedule slightly lowers Kendall's Tau (more fan-driven variation) while maintaining identical reversal and Condorcet performance. This motivates a dynamic mechanism that preserves audience influence without degrading fairness signals.
