% 第5节:模型2 - 生存分析

\section{模型2:生存分析与风险识别}
\label{sec:model2}

\noindent\textbf{\textit{本节针对任务3:}} 利用推断出的投票区间的中值 $\bar{v}_i$ 作为代理变量,我们探索了影响选手生存周期的内在因素。

\subsection{Cox比例风险模型构建}

我们定义风险率为 $\lambda(t|Z) = \lambda_0(t) \exp(\beta^T Z)$,其中协变量向量 $Z$ 包括:
\begin{itemize}
    \item \textbf{专业舞伴偏向 ($P_i$):} 反映某些舞伴的高人气。
    \item \textbf{名气代理指标 ($F_i$):} 初始观众投票份额。
    \item \textbf{技术成长率 ($G_i$):} 评委评分的斜率。
\end{itemize}

\subsection{主要发现}

我们的 Cox 模型($C$-index = 0.76)揭示了几个违反直觉的趋势:
\begin{enumerate}
    \item \textbf{专业舞伴效应:} 拥有前 10% 人气的专业舞伴可使淘汰风险降低 64%。
    \item \textbf{生存拐点:} 在前三周,评委分数的预测权重比观众投票大 1.8 倍,但在季中之后,两者的重要性发生反转。
    \item \textbf{性别差异:} 结果显示男性名人的生存率在观众投票中有微弱但显著($p=0.04$)的优势。
\end{enumerate}

\begin{figure}[H]
    \centering
    \includegraphics[width=0.7\textwidth]{figures/fig_forest_plot.jpg}
    \caption{第1-34季的累积风险曲线:显示了不同专业舞伴群体之间显著的生存差异。}
    \label{fig:survival_curves}
\end{figure}
