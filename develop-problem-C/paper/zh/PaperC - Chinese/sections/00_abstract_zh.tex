% 摘要 / 概要页

\begin{abstract}

\textbf{受评委与观众反复争议启发,我们将《与星共舞》(DWTS)视为审计问题:} 在不可观测粉丝票的条件下,公开规则是否足以解释淘汰结果?

我们构建\textbf{双核反演引擎}:百分制赛季用 LP 反演可行区间,排名制赛季用 MILP 推断隐含粉丝排名。随后在区间上建立\textbf{截断贝叶斯后验}并用 MCMC 采样,引入时间平滑,同时仅保留能复现实测淘汰的样本。

后验重建覆盖 264 个淘汰周次、2,250 条选手-周估计。反事实评估给出 Kendall 相关均值约 0.455、逆转率约 0.174,显示赛制稳定但存在结构性张力。特征归因采用\textbf{XGBoost + SHAP},表明评委分统计为最强驱动因素,舞伴效应与年龄呈非线性贡献。

在机制设计上,我们提出\textbf{动态自适应权重}(前期偏评委、后期偏观众),并通过帕累托前沿展示公平性与参与度的平衡。60/40 为静态平衡点,动态方案在不削弱公平信号的前提下增强观众参与感。

\begin{keywords}
贝叶斯推断;MCMC;线性规划;混合整数规划;XGBoost;SHAP;机制设计
\end{keywords}

\end{abstract}
