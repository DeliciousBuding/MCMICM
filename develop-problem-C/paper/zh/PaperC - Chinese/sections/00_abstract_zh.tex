% 摘要 / 概要页
% 摘要页(占第0页,不计入正文)

\begin{abstract}

\textbf{受评委与观众之间反复出现的争议(如第27季)启发,我们将《与星共舞》(DWTS)视为一个“审计问题”:} 能否在\textbf{从未观测到个人观众投票份额}的情况下,通过节目公布的评分规则解释观测到的淘汰结果?

为了回答这一问题,我们构建了一个\textbf{双核反演引擎}:(i) 通过\textbf{线性规划(LP)}重构百分比规则季(S3--S27)下的“可行观众投票区间”;(ii) 通过\textbf{混合整数线性规划(MILP)}推断排名规则季(S1--S2, S28--S34)下的“仅排名观众置换”。随后,我们利用最小松弛指标 $S^*$ 来量化\textbf{模型与数据的不匹配},从而识别出无法通过任何可行投票分配来复现观测结果的情形。

在对全部34季的研究中,我们的反演引擎在1%的容差范围内解释了\textbf{32个赛季(松弛量为零)};仅有\textbf{S32--S33}表现出结构性不匹配(在声明的规则下不存在可行解)。可能的原因包括未记录的规则变动、未建模的评委拯救机制或多支舞蹈聚合方式的差异。

基于推断出的投票数据,\textbf{Cox生存分析}揭示:(1) 专业舞伴是主导性的生存预测因子(似然比检验 $\chi^2 = 47.3, p < 0.001$;Harrell's C指数提升 $\Delta C = 0.09$);(2) 评委评分在早期轮次占主导,而观众投票势头则驱动后期结果。

在机制设计方面,我们推荐的\textbf{加权百分比}方案($0.6 \times \text{评委百分比} + 0.4 \times \text{观众百分比}$)在保持帕累托拐点附近观众影响力的同时,将\textbf{“确定性错误淘汰”}从40人减少到3人(降幅高达92.5%)。

\begin{keywords}
反问题;线性规划;混合整数规划;生存分析;机制设计
\end{keywords}

\end{abstract}
