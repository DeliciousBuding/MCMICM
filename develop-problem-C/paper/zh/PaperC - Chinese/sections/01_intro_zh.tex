% 引言(中文)

\section{引言}
\label{sec:intro_zh}

\subsection{背景}

\textbf{受评委与观众争议启发}(如 S27 Bobby Bones),我们将《与星共舞》视为审计问题:在不可观测粉丝票的条件下,公开规则是否足以解释淘汰结果?

节目规则在 34 季中多次变化(见图 \ref{fig:flowchart}),因此需要在不同信息结构下进行反演。

\begin{figure}[h]
    \centering
    \includegraphics[width=1.0\textwidth]{figures/figure1.jpg}
    \vspace{-1.5em}
    \caption{\textbf{分析框架。} 双核反演引擎在百分制与排名制赛季中重建隐藏粉丝票。}
    \vspace{-1.0em}
    \label{fig:flowchart}
\end{figure}

\subsection{问题描述}
\begin{enumerate}[label=\arabic*., nosep, leftmargin=*]
    \item \textbf{反演粉丝票(任务1--2):} 基于淘汰结果推断可行区间。
    \item \textbf{规则变动影响(任务2):} 免疫、双淘汰、评委拯救的影响。
    \item \textbf{成功因素(任务3):} 何种属性影响生存周数。
    \item \textbf{机制评估(任务4):} 现有赛制是否公平。
    \item \textbf{机制设计(任务5):} 如何优化公平性与参与度。
\end{enumerate}

\subsection{贡献}
\begin{enumerate}[itemsep=0.1em]
    \item \textbf{双核反演 + 后验重建:} LP/MILP 反演区间,截断贝叶斯 MCMC 输出均值与 HDI。
    \item \textbf{规则透明度审计:} 通过松弛指标验证规则一致性。
    \item \textbf{反事实评估:} Kendall 相关、逆转率与孔多塞一致性。
    \item \textbf{特征归因:} XGBoost + SHAP 揭示关键生存因素。
    \item \textbf{动态机制设计:} 评委权重前高后低,兼顾公平与参与。
\end{enumerate}
