% Section 4: Model 1 (Chinese)

\section{模型1:粉丝票反演与规则审计}
\label{sec:model1}

\noindent\textbf{本节对应任务1--2:} 从淘汰结果反演粉丝票区间,并识别规则无法解释的周次。

\subsection{问题表述}
设第 $s$ 赛季第 $w$ 周活跃选手集合为 $\mathcal{C}_{s,w}$,粉丝票份额 $\mathbf{v}$ 满足:
\begin{equation}
    \sum_{i=1}^{n} v_i = 1, \quad v_i \ge \epsilon.
\end{equation}

\begin{figure}[H]
    \centering
    \includegraphics[width=0.9\textwidth]{figures/fig_dual_engine.jpg}
    \caption{\textbf{双核反演架构}:百分制赛季使用 LP,排名制赛季使用 MILP/CP。}
    \label{fig:dual_engine}
\end{figure}

\subsection{百分制赛季:LP 反演}
在 S3--S27,合成分数 $C_i = J_i^{\%} + v_i$,被淘汰者应具有最低 $C_i$。当严格可行性失效时,引入松弛变量,得到最小不一致度 $S^*$,用于规则透明度审计。

\subsection{排名制赛季:MILP/CP 反演}
在 S1--S2 与 S28+,粉丝票排名为潜变量,需通过 MILP/CP 反演。评委拯救赛季仅要求淘汰者处于 bottom-two 集合。

\subsection{规则透明度审计}
我们用 $S^*$ 度量规则无法解释的程度,作为“规则透明度”指标。$S^* > 0$ 表示公开规则无法完全解释淘汰结果。

\begin{figure}[H]
    \centering
    \includegraphics[width=0.85\textwidth]{figures/fig_anomaly_detection.pdf}
    \caption{\textbf{规则透明度审计}:每赛季的最大不一致度 $S^*$。}
    \label{fig:money_plot}
\end{figure}

\subsection{约束一致性采样(拒绝采样)}
在区间内提出候选样本,仅保留能复现实测淘汰的样本:
\begin{enumerate}[itemsep=0.1em]
    \item \textbf{Proposal:} 在 $[L_i, U_i]$ 超矩形中生成候选 $V$;
    \item \textbf{Accept:} 仅保留与实测淘汰一致的样本;
    \item \textbf{Audit:} 在通过样本中计算人气偏离概率。
\end{enumerate}

\subsection{启发式正则(保守估计)}
在排名制赛季中,区间可能极宽。我们提供可选的区间收缩用于敏感性分析,主结果默认不启用。

\subsection{小结}
\begin{itemize}[itemsep=0.2em]
    \item 双核反演统一了不同赛制的粉丝票反演问题。
    \item $S^*$ 提供可解释的规则透明度审计。
    \item 约束一致性采样为“人气偏离”提供概率级评估。
\end{itemize}
