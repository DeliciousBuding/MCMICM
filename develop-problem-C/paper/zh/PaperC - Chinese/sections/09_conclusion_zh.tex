% 第9节:结论与建议

\section{结论}
\label{sec:conclusion}

\subsection{研究发现总结}

本文开发了一个综合框架,用于从《与星共舞》(DWTS)的淘汰数据中反演观众投票。主要发现如下:

\begin{enumerate}[itemsep=0.2em]
    \item \textbf{双核反演引擎:} 使用混合 LP/MILP 框架成功重构了 34 个赛季中 32 个赛季的观众投票分布,一致性达 100\%。
    
    \item \textbf{模型-数据不匹配监测:} 识别出第32和33赛季存在正松弛量($S^* > 0$),揭示了评委拯救机制引入的数学复杂性。这种不匹配通过模型量化了决策的不透明性。
    
    \item \textbf{生存预测因子:} 生存分析证实,专业舞伴对成绩的提升效应($\chi^2 = 47.3, p < 0.001$)远大于选手的背景职业或初始知名度。
    
    \item \textbf{机制改进效果:} 提议的加权百分比机制($\alpha = 0.6$)在区间稳健分类下,将确认的非预期淘汰从 40 例大幅缩减至 3 例  \textbf{实现了 92.5\% 的公平性提升}。
\end{enumerate}

\subsection{对制片方的建议}

\begin{enumerate}[itemsep=0.2em]
    \item \textbf{实施审计协议:} 在每集播出前运行反演引擎。标记 $S^* > 0.5$ 的结果以供制作人二次审核,确保规则执行的一致性。
    
    \item \textbf{采用加权百分比评分:} 转为 $0.6 \times \text{评委\%} + 0.4 \times \text{观众\%}$ 的加权系统。其帕累托优越性在于保留 68\% 观众影响力的同时,消除了绝大部分的非预期淘汰风险。
    
    \item \textbf{量化评委拯救标准:} 明确并记录导致评委拯救决策的定性属性(如本周进步排名),以缩减审计不确定性区间。
\end{enumerate}

\subsection{核心要点}

当前 DWTS 投票系统在 230 轮淘汰中产生了 40 例确认的非预期淘汰(占 17.4\%)。\textbf{我们的加权百分比系统($\alpha = 0.6$)将其降至仅 3 例(占 1.3\%)}  这一改进在允许观众保留强力发声权的同时,最大限度地捍卫了竞赛的技术严谨性。
