% Section 6: Model 3 (Chinese)

\section{模型三:特征归因与赛制设计}
\label{sec:model3_zh}

\noindent\textbf{\textit{对应问题 3--4:}} 使用 XGBoost+SHAP 解释选手生存因素,并设计动态加权赛制。

\subsection{特征归因(XGBoost + SHAP)}

\begin{figure}[H]
    \centering
    \includegraphics[width=0.85\textwidth]{figures/fig_shap_summary.pdf}
    \caption{\textbf{SHAP 总览。} 评委分统计占主导,舞伴效应与年龄呈非线性影响。}
    \label{fig:shap_summary_zh}
\end{figure}

\begin{figure}[H]
    \centering
    \includegraphics[width=0.7\textwidth]{figures/fig_shap_age_dependence.pdf}
    \caption{\textbf{年龄依赖图。} 年龄贡献呈非线性。}
    \label{fig:shap_age_zh}
\end{figure}

\subsection{动态权重机制}

\begin{equation}
\alpha_w =
\begin{cases}
\alpha_0, & w \le 5,\\
\alpha_0 - (\alpha_0-\alpha_T)\dfrac{w-5}{W-5}, & w > 5,
\end{cases}
\end{equation}
其中 $(\alpha_0,\alpha_T)=(0.7,0.4)$。

\begin{figure}[H]
    \centering
    \includegraphics[width=0.7\textwidth]{figures/fig_dynamic_weight_schedule.pdf}
    \caption{\textbf{动态权重曲线。} 前期偏评委,后期偏观众。}
    \label{fig:dynamic_schedule_zh}
\end{figure}

\subsection{帕累托前沿}

\begin{figure}[H]
    \centering
    \includegraphics[width=0.75\textwidth]{figures/fig_pareto_frontier.pdf}
    \caption{\textbf{帕累托前沿。} $\alpha=0.6$ 时评委一致性约 0.50、观众影响约 0.83。}
    \label{fig:pareto_zh}
\end{figure}

\noindent\textbf{建议:} 采用动态权重以兼顾技术公平与观众参与感。
