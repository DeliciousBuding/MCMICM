% Section 6: Model 3 (Chinese)

\section{模型3:机制设计}
\label{sec:model3}

\noindent\textbf{本节对应任务4--5:} 在公平性与观众参与之间寻找机制优化方案。

\subsection{区间鲁棒分类}
基于 $[v_i^{\min}, v_i^{\max}]$,将淘汰分为三类:
\begin{enumerate}[itemsep=0.2em]
    \item \textbf{明确安全($\mathcal{D}_S$):} 任意可行 $v_i$ 下,淘汰者均为低人气/低评委分。
    \item \textbf{明确不公($\mathcal{D}_W$):} 任意可行 $v_i$ 下,淘汰者在人气与评委分上均高于中位数。
    \item \textbf{可能不公($\mathcal{P}_W$):} 介于两者之间,依赖真实票值。
\end{enumerate}

\subsection{候选机制}
我们比较现行 50/50、软阈值(Soft Floor)与加权百分比:
\begin{equation}
    C_i^{\text{new}} = \alpha J_i^{\%} + (1-\alpha) v_i, \quad \alpha=0.6.
\end{equation}

\subsection{公平-参与度权衡(Pareto)}
\begin{figure}[H]
    \centering
    \includegraphics[width=0.85\textwidth]{figures/fig_pareto_frontier.jpg}
    \caption{\textbf{Pareto 前沿:} 兼顾公平与参与的“拐点”配置。}
    \label{fig:pareto}
\end{figure}

\begin{table}[H]
    \centering
    \caption{机制对比(区间鲁棒分类)}
    \label{tab:mechanism_comparison}
    \begin{tabular}{lcccc}
        \toprule
        \textbf{机制} & \textbf{参数} & $|\mathcal{D}_W|$ & $|\mathcal{P}_W|$ & $|\mathcal{D}_S|$ \\
        \midrule
        现行机制 & 50/50 & 40 & -- & 190 \\
        Soft Floor & $\theta=5\%$ & 94 & -- & 136 \\
        \textbf{Weighted Percent} & $\alpha=0.6$ & \textbf{3} & 37 & 190 \\
        \bottomrule
    \end{tabular}
\end{table}

\subsection{结论}
加权百分比(60/40)在公平性提升与粉丝参与之间取得最优平衡,具有可解释性与可推广性。
