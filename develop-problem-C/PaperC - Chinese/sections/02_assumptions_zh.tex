% 第2节:模型假设

\section{模型假设与合理性建议}
\label{sec:assumptions}

为了简化分析并捕捉问题的本质,我们基于对DWTS竞赛动态的观察,提出以下核心假设:

\begin{itemize}
    \item \textbf{诚实投票。} 观众为自己喜欢的选手投票,不存在因策略性考虑而投票给弱者以排除强敌的情况(符合娱乐节目的一般观察)。
    \item \textbf{真实淘汰。} 公布的淘汰结果准确反映了即便是在最不利的可能观众分布下,该选手也无法晋级的事实(作为审计的基准前提)。
    \item \textbf{投票下限。} 每位选手获得最低份额 $\epsilon = 0.001$ 的观众投票份额,以反映极少出现零票的情况。
    \item \textbf{规则准确性。} 记录在案的制作规则(百分比 vs. 排名)在对应的赛季中是一致执行的。
    \item \textbf{评委独立性。} 评委在投票统计前独立给出评分,不受实时观众投票率的影响。
    \item \textbf{时间独立性。} 观众无法根据前几周的推断结果改变本周的投票偏好(即本模型暂不考虑跨周的内生反馈循环)。
    \item \textbf{模型-数据不匹配检测。} 未建模的因素(如突发退赛、评委临时改变决策逻辑)将通过目标函数中的松弛项 $\mathbf{s}$ 自动识别。
\end{itemize}
