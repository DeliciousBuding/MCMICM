% 第1节:引言

\section{引言}
\label{sec:intro}

\subsection{研究背景}

《与星共舞》(DWTS)是一项将名人与专业舞者配对的长期竞赛。尽管评委的评分是公开的,但决定淘汰结果的\textbf{个人观众投票份额从未被披露}。这种信息的不对称性催生了诸如第27季(Bobby Bones获胜引发巨大争议)及第32季评委拯救规则引发的长期公众讨论。从公开的计分规则和观测到的淘汰结果中逆推背后的观众偏好,构成了一个典型的\textbf{数学反问题}。

\begin{figure}[H]
    \centering
    \includegraphics[width=1.0\textwidth]{figures/figure1.jpg}
    \vspace{-1em}
    \caption{我们的反演引擎概览(左侧)与主要研究成果(右侧):展示了从基于线性规划(LP)+ 混合整数规划(MILP)的观众投票推断到生存分析及机制设计的完整流程。}
    \label{fig:intro_summary}
\end{figure}

\subsection{问题陈述}

本研究旨在解决以下五个核心任务:
\begin{itemize}[nosep]
    \item \textbf{观众投票推断:} 如何从各季(S1--S34)观测到的淘汰结果中反演开出的合理观众投票区间?
    \item \textbf{规则演进评估:} 豁免权、双淘汰及评委拯救机制如何改变计分博弈的约束空间?
    \item \textbf{生存因子识别:} 通过对推断数据的生存分析,哪些因素(舞伴、名气、评分)最能预测生存周期?
    \item \textbf{机制效能审计:} 历史上哪些淘汰案例被视为“不公平”?这种不公平是系统性的吗?
    \item \textbf{评分方案优化:} 能否通过机制设计降低“错误淘汰率”,同时兼顾收视率与专业公平性?
\end{itemize}

\vfill

\subsection{我们的贡献}

\begin{itemize}[nosep]
    \item \textbf{双核反演引擎:} 开发了针对百分比规则(LP)和排名规则(MILP)的专用推断核。
    \item \textbf{结构性不匹配识别:} 通过松弛指标分析,首次量化并指出了S32--S33季中存在的模型-数据不匹配现象。
    \item \textbf{关键生存因子:} 发现专业舞伴的影响力($\chi^2=47.3$)显著高于名人初始名气。
    \item \textbf{帕累托优化机制:} 提出了加权百分比规则,将“确认的错误淘汰”数量减少了92.5%。
\end{itemize}

\clearpage
