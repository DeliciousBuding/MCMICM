% 第1节:引言

\section{引言}
\label{sec:intro}

\subsection{研究背景}

《与星共舞》(DWTS)自2005年以来已经播出34季,将名人与专业舞者进行配对。评委评分是公开的,但\textbf{个人观众投票份额从未被直接公开}。相反,部分信息通过投票系统间接透露:百分比规则季(第3-27季)采用基于百分比的评分方式,综合得分由规则定义;而排名规则季(第1-2季、第28季起)仅公开综合排名。这形成了一个\emph{反问题}:从可观测的淘汰结果中重构潜在的观众投票分布。

\subsection{问题陈述}

我们解决五个相互关联的问题(MCM问题C任务):
\begin{enumerate}[itemsep=0.1em]
    \item \textbf{观众投票推断:}从淘汰结果重构合理的观众投票分布。
    \item \textbf{规则变更影响:}豁免权、双淘汰和评委拯救如何影响推断?
    \item \textbf{成功因素:}什么因素预测生存(职业、舞伴、名气)?
    \item \textbf{机制评估:}当前系统是否产生``公平''的结果?
    \item \textbf{机制设计:}我们能否在保持观众参与度的同时提高公平性?
\end{enumerate}

\subsection{主要贡献}

\begin{enumerate}[itemsep=0.1em]
    \item \textbf{双核引擎:}LP(百分比规则季)+ MILP(排名规则季)用于观众投票推断。
    \item \textbf{模型-数据不匹配检测:}识别出第32/33季存在非零约束松弛量——揭示了第28季以后规则变更带来的数学复杂性。
    \item \textbf{生存分析:}专业舞者是主导预测因子(LRT $\chi^2 = 47.3$,$\Delta C = 0.09$)。
    \item \textbf{加权百分比系统:}将确定性错误淘汰从40减少到3。
    \item \textbf{透明协议:}提供平衡娱乐性和公平性的可操作建议。
\end{enumerate}

\subsection{论文结构}

\secref{sec:assumptions}:假设;\secref{sec:notations}:符号说明;\secref{sec:model1}:观众投票反演;\secref{sec:model2}:生存分析;\secref{sec:model3}:机制设计;\secref{sec:sensitivity}:敏感性分析;\secref{sec:evaluation}:模型评估;\secref{sec:conclusion}:结论。
