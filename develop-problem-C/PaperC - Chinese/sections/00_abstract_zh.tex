% 摘要 / 概要页
% 摘要页(占第0页,不计入正文)

\begin{abstract}

《与星共舞》(DWTS)节目将评委评分与观众投票相结合来决定每周的淘汰结果,但个人观众投票份额从未被直接公开。我们开发了一个\textbf{双核反演引擎},用于从美国全部34季的淘汰结果中重构潜在的观众投票分布。

对于\emph{百分比规则季}(第3-27季),我们将观众投票推断表述为带有松弛变量的\textbf{线性规划(LP)}问题,其中评委百分比计算公式为 $J_i / \sum_k J_k$(使用求和而非均值)。对于\emph{排名规则季}(第1-2季、第28-34季),观众投票排名被视为\textbf{潜在决策变量},通过带有排列约束的\textbf{混合整数线性规划(MILP)}求解。两种求解器都在影响可行性的规则变更处进行了处理(豁免权、双淘汰);对于评委拯救机制(第28季起),我们建模了\emph{垫底二人可行性},但未建模现场拯救标准,将未记录的实施细节视为建模不确定性。

我们的框架识别出第32季和第33季存在\textbf{模型与数据不匹配}——约束松弛量 $S^* > 0$ 表明存在无法解释的结果。可能原因包括:未记录的规则变化、未建模的评委拯救标准或外部因素。

基于推断的投票数据,\textbf{Cox生存分析}揭示:(1)专业舞伴是主导性的生存预测因子(似然比检验 $\chi^2 = 47.3$,$p < 0.001$;Harrell's C指数提升 $\Delta C = 0.09$);(2)评委评分主导早期回合,而观众支持势头推动后期阶段的结果。

我们提出了一种\textbf{加权百分比}机制($0.6 \times \text{评委百分比} + 0.4 \times \text{观众百分比}$),在基于区间的稳健分类下,将确定性错误淘汰数从40降至\textbf{3}。Jerry Rice(第2季)和Bobby Bones(第27季)的案例研究展示了关键争议。

\begin{keywords}
反问题;线性规划;混合整数规划;生存分析;机制设计
\end{keywords}

\end{abstract}
