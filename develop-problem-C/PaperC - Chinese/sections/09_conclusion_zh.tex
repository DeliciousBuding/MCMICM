% 第9节:结论与建议

\section{结论}
\label{sec:conclusion}

\subsection{研究发现总结}

本文开发了一个综合框架,用于从《与星共舞》的淘汰数据中推断观众投票情况。我们的主要发现如下:

\begin{enumerate}[itemsep=0.2em]
    \item \textbf{双核反演:} 使用混合 LP/MILP 成功重构了 34个赛季中 32个赛季的观众投票分布,约束满足率达 100\%,中位边界宽度为 8.3\%。
    
    \item \textbf{模型-数据不匹配检测:} 识别出第32/33季约束松弛量 $S^* > 0$,揭示了评委拯救机制引入的数学复杂性。这\emph{不是}操纵证据,但突显了第28季起规则中的建模挑战。
    
    \item \textbf{专业舞者效应:} 生存分析表明专业舞伴是最主要的预测因子(似然比检验 $\chi^2 = 47.3$,$p < 0.001$;C指数改进量 $\Delta C = 0.09$)——其影响大于名人类型或先前名气。
    
    \item \textbf{加权百分比系统:} 提出的机制 ($0.6J^\% + 0.4V$) 在区间稳健分类下将确定性错误淘汰从 \textbf{40次减少到3次}。
\end{enumerate}

\subsection{建议}

\begin{enumerate}[itemsep=0.2em]
    \item \textbf{播前透明化:} 在播出前运行 LP/MILP 求解器;标记 $S^* > 0.5$ 的期次供制作人审核。
    \item \textbf{采用加权百分比:} $\text{得分}_i = 0.6 \cdot J_i^\% + 0.4 \cdot V_i$ —— 透明的``60\%技术,40\%人气''。
    \item \textbf{评委拯救透明度:} 记录评委拯救决策的标准,以减少模型-数据不匹配。
\end{enumerate}

\subsection{局限性}

\begin{itemize}[itemsep=0.1em]
    \item 未观测到的异质性(社交媒体活动、新闻事件)。
    \item 排名规则季(第28-34季)的边界较宽。
    \item 无法确定不匹配的\emph{原因}(仅能检测)。
    \item 反事实模拟假设投票行为不发生改变。
\end{itemize}

\subsection{未来工作}

未来的扩展:(1) 为制作人开发实时仪表盘,(2) 国际化扩展至《舞动奇迹》(Strictly Come Dancing) 等其他授权节目,(3) 结合社交媒体情感分析以缩小边界。

\vspace{0.5em}
\noindent\textbf{核心要点:} 当前的 DWTS 系统在 230 轮中产生了 40次确定性错误淘汰 (17.4\%)。我们的加权百分比系统将其减少到 3次 (1.3\%) —— 这一稳健的改进已通过考虑观众投票估计不确定性的区间分类法得到验证。
