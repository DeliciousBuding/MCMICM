% 第8节:模型评价

\section{模型评价}
\label{sec:evaluation}

\subsection{优势}
\begin{itemize}[itemsep=0.1em]
    \item \textbf{双核灵活性:} 自动适应 DWTS 历史上反复变动的百分比规则与排名规则。
    \item \textbf{异常自动监测:} 松弛量 $s$ 的引入使模型不仅能解释数据,还能识别规则执行中的异常(如 S32--33)。
    \item \textbf{机制优化路径:} 提供的帕累托前沿分析为如何在改进公平性的同时最大限度减少对“观众主权”的侵蚀提供了定量依据。
\end{itemize}

\subsection{局限性}
\begin{itemize}[itemsep=0.1em]
    \item \textbf{社交媒体忽略:} 本模型未整合实时社交媒体声量数据,这可能是一个重要的观众情绪领先指标。
    \item \textbf{跨周反馈假设:} 我们假设各周投票是独立的,忽略了“由于某人表现不佳而引发的下周同情票”等复杂心理动力学。
\end{itemize}

\subsection{未来改进方向}
\begin{itemize}[itemsep=0.1em]
    \item 整合 NLP 分析观众评论的情感倾向,以精确估计 $\epsilon$。
    \item 开发分布式模拟器,以评估不同观众地理分布对权力的影响。
\end{itemize}
