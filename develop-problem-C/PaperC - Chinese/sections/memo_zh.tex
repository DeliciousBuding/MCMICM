% 备忘录:审计备忘录 (Audit Memorandum)

\newpage
\thispagestyle{empty}

\begin{center}
\Large\textbf{审\quad 计\quad 备\quad 忘\quad 录}
\end{center}

\vspace{0.5em}
\noindent\rule{\textwidth}{1.5pt}
\vspace{0.3em}

\begin{tabular}{@{}ll}
\textbf{收件人:}  & 《与星共舞》(DWTS)执行制片人 \\
\textbf{发件人:}  & Team \#2617892 --- 建模与审计小组 \\
\textbf{日期:}    & 2026年1月30日 \\
\textbf{主题:}    & \textbf{透明度风险评估与规则改进建议}
\end{tabular}

\vspace{0.3em}
\noindent\rule{\textwidth}{1.5pt}
\vspace{0.8em}

\subsection*{执行摘要}

我们对《与星共舞》全部 34 个赛季的淘汰结果进行了独立审计。本次分析旨在回答一个核心问题:\textbf{在不观测具体观众投票份额的情况下,节目发布的评分规则能否解释所观测到的淘汰结果?}

在 34 个赛季中,有 32 个赛季的回答是\textbf{肯定}的  我们的数学模型找到了与所有结果一致的可行观众投票分布。然而,\textbf{第32和33赛季}存在结构性不匹配:在既定规则下,没有任何可行的投票分配能完全复现观测到的淘汰结果。

\subsection*{核心发现(审计视角)}

自第28赛季引入评委拯救(Judges' Save)环节以来,节目产生了一层\textbf{决策不透明性}:虽然播出的倒数两名程序是规则驱动的,但最终淘汰取决于未以可验证形式披露的定性标准。这并不意味着存在违规操作,但它通过使某些结果在单纯基于规则的模型下\textbf{不可复现},从而\textbf{增加了审计风险}。

\subsection*{三大主要发现}

\begin{enumerate}[itemsep=0.2em]
    \item \textbf{近期赛季的模型-数据不匹配} \\
    第32和33赛季在我们的优化框架中表现出正松弛量($S^* > 0$),表明在文档记录的规则下,无法通过任何可行的投票分配来解释其实际结果。可能原因:评委拯救标准基于未披露的因素(如进步程度、艺术表现力),而评分逻辑未能捕捉这些因素。\textit{影响:这些赛季的审计追踪不完整。}
    
    \item \textbf{粉丝动员带来的结构性脆弱性} \\
    当前 50/50 的评委-观众比例允许即使技术得分极低的选手,也能凭借动员起来的粉丝群生存。第27赛季(Bobby Bones)便是明证:推断的投票区间显示,如果对手票数分散,选手仅凭约 25--30\% 的观众支持率即可晋级。\textit{影响:当前机制下存在竞赛公正性风险。}
    
    \item \textbf{识别出 40 例确认的非预期淘汰} \\
    利用区间稳健分类法,我们识别出历史上有 40 例淘汰案例,其中被淘汰选手在所有可行投票分配中,其评委得分\emph{及}估计观众支持率均高于中位数。通过机制调整,这一数字可降至 \textbf{3} 例。
\end{enumerate}

\subsection*{政策建议:加权百分比机制}

为了在不牺牲观众参与度的前提下提高透明度,我们建议:

\begin{itemize}[itemsep=0.1em]
    \item \textbf{采用 60/40 权重:} $0.6 \times \text{评委百分比} + 0.4 \times \text{观众百分比}$
    \item \textbf{预期影响:} 将确认的非预期淘汰从 40 例减少到 3 例(降幅 \textbf{92.5\%})。
    \item \textbf{稳健性:} 性能在 $\alpha \in [0.55, 0.65]$ 范围内保持稳定(位于帕累托拐点)。
    \item \textbf{保留粉丝参与度:} 保留了 68\% 的历史观众影响力指数,确保观众依然拥有决定性。
\end{itemize}

\noindent 此外,建议报告最低限度的可验证信息(如倒数两名的具体构成),以提高审计追踪的完整性。

\subsection*{范围与局限性说明}

\begin{itemize}[itemsep=0.1em]
    \item 我们从未观测到实际票数  分析产生的是\emph{可行区间}而非精确值。
    \item 不匹配事件($S^* > 0$)标志着\textbf{模型与数据的偏差},而非操纵的证据。
    \item 第27赛季后的规则细节主要根据问题描述推断,官方内部文档可能有所不同。
\end{itemize}

\vspace{0.8em}
\hfill\textbf{--- Team \#2617892, 建模与审计小组}

\vspace{0.3em}
\hfill\textit{详细技术方法见随附报告主体。}

