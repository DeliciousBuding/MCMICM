% 第7节:敏感性分析

\section{敏感性与稳健性分析}
\label{sec:sensitivity}

\subsection{针对投票单性形约束的扰动}

我们通过对推断的观众投票份额 $v_i$ 引入 $\pm 5\%$ 的随机摄动来测试模型 1 的稳定性。结果显示,对于超过 95% 的单周案例,可行区间的结构保持不变(即晋级/淘汰的逻辑顺序未发生逆转),这证明了我们的 LP 框架具有良好的稳健性。

\subsection{留一法交叉验证 (LOSO)}

为了验证模型 2 的泛化能力,我们对 34 个赛季进行了留一法验证。
\begin{itemize}
    \item \textbf{C-index 稳定性:} 在不同赛季中,$C$-index 维持在 $0.74-0.78$ 之间。
    \item \textbf{最敏感因素:} 专业舞伴的人气指标表现出最高的预测稳健性,而技术成长率在地 30 季之后受评委拯救机制的影响表现出较大的波动。
\end{itemize}

\subsection{Bootstrap 显著性检验}

对 Cox 模型的系数进行 1000 次 Bootstrap 采样后的结果(如图 \ref{fig:bootstrap} 所示)表明,主要生存因子的 95% 置信区间均不包含零,证实了我们的风险识别是可靠的。

\begin{figure}[H]
    \centering
    \includegraphics[width=0.6\textwidth]{figures/fig_forest_plot.jpg}
    \caption{Bootstrap 采样分布:展示了各协变量对生存期影响的稳健性。}
    \label{fig:bootstrap}
\end{figure}
