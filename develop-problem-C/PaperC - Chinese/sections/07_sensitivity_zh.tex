% 第7节:敏感性与鲁棒性分析

\section{敏感性与鲁棒性分析}
\label{sec:sensitivity}

我们进行了系统的敏感性分析,以验证我们的模型在不同参数选择和数据扰动下的鲁棒性。

% 7.1 最小投票份额敏感性
\subsection{最小投票份额下限 ($\epsilon$)}
\label{subsec:epsilon_sensitivity_zh}

下限参数 $\epsilon$ 控制着模型1中的最小投票份额约束。我们将 $\epsilon$ 从 0\% 变化到 10\%,并检查:

\begin{itemize}[itemsep=0.2em]
    \item \textbf{可行率:} $S^* = 0$ 的赛季比例。
    \item \textbf{边界紧凑度:} 平均区间宽度 $v_i^{\max} - v_i^{\min}$。
    \item \textbf{不匹配检测:} 在每个 $\epsilon$ 下哪些赛季表现出 $S^* > 0$。
\end{itemize}

\noindent\textbf{主要发现:} 在 $\epsilon = 1\%$(我们的默认值)时,94.1\% 的赛季是可行的,中位边界宽度为 8.3\%。增加到 $\epsilon = 5\%$ 时,边界缩小到 5.1\%,但可行率降低到 82.4\%。

% 7.2 机制权重敏感性
\subsection{机制权重 ($\alpha$) 敏感性}
\label{subsec:alpha_sensitivity_zh}

对于加权百分比机制,我们将 $\alpha$(评委权重)从 0.3 到 0.8 进行变化:

\begin{table}[H]
    \centering
    \caption{确定性错误淘汰 ($|\mathcal{D}_W|$) vs.\ 评委权重 $\alpha$}
    \label{tab:alpha_sensitivity_zh}
    \begin{tabular}{cccc}
        \toprule
        $\alpha$ & $|\mathcal{D}_W|$ & 比例 & 相比当前系统的减少 \\
        \midrule
        0.30 & 19 & 8.3\% & 52.5\% \\
        0.40 & 12 & 5.2\% & 70.0\% \\
        0.50 & 6 & 2.6\% & 85.0\% \\
        0.55 & 4 & 1.7\% & 90.0\% \\
        \textbf{0.60} & \textbf{3} & \textbf{1.3\%} & \textbf{92.5\%} \\
        0.65 & 4 & 1.7\% & 90.0\% \\
        0.70 & 8 & 3.5\% & 80.0\% \\
        0.80 & 18 & 7.8\% & 55.0\% \\
        \bottomrule
    \end{tabular}
\end{table}

\noindent\textbf{主要发现:} 最优范围是 $\alpha \in [0.55, 0.65]$,其中 $\alpha = 0.6$ 将确定性错误淘汰数降至 3。该机制对 $\alpha$ 的微小扰动具有鲁棒性。

% 7.3 Bootstrap 稳定性
\subsection{Bootstrap 稳定性分析}
\label{subsec:bootstrap_stability_zh}

为了评估估计的稳定性,我们进行了 1,000 次赛季级别的 Bootstrap 重采样:

\begin{itemize}[itemsep=0.2em]
    \item \textbf{观众投票边界:} 中位边界宽度的 95\% 置信区间为 $[7.6\%, 9.1\%]$。
    \item \textbf{专业舞者效应:} C指数改进量 $\Delta C$ 的 95\% 置信区间为 $[0.06, 0.12]$。
    \item \textbf{加权百分比错误淘汰率:} 95\% 置信区间为 $[0.4\%, 2.6\%]$(相比当前系统有稳健的提升)。
\end{itemize}

% 7.4 Leave-One-Season-Out 交叉验证
\subsection{留一赛季交叉验证 (Leave-One-Season-Out)}
\label{subsec:loso_zh}

我们使用留一赛季交叉验证来验证模型2(生存分析):

\begin{enumerate}
    \item 在33个赛季上训练 Cox 模型。
    \item 预测预留赛季的生存曲线。
    \item 计算预留数据上的一致性指数 (C-index)。
\end{enumerate}

\begin{table}[H]
    \centering
    \caption{留一赛季验证结果}
    \label{tab:loso_zh}
    \begin{tabular}{lc}
        \toprule
        \textbf{指标} & \textbf{值} \\
        \midrule
        平均 C-index & 0.72 \\
        标准差 C-index & 0.08 \\
        最差赛季 (第32季) & 0.54 \\
        最佳赛季 (第15季) & 0.86 \\
        \bottomrule
    \end{tabular}
\end{table}

\noindent\textbf{主要发现:} 模型达到了 0.72 的平均 C-index,表明了良好的区分能力。值得注意的是,不匹配的第32季具有最低的 C-index (0.54),证实了我们的不匹配检测能够正确识别出标准模型表现不佳的赛季。

% 7.5 数据扰动鲁棒性
\subsection{数据扰动鲁棒性}
\label{subsec:perturbation_zh}

我们在评委评分中加入高斯噪声($\sigma = 1$分)并重新运行反演:

\begin{itemize}[itemsep=0.2em]
    \item \textbf{边界稳定性:} 边界的中位绝对变化为 1.2 个百分点。
    \item \textbf{排名一致性:} 94.3\% 的选手排名得以保持。
    \item \textbf{不匹配持久性:} 在所有扰动试验中,第32季和第33季始终被标记。
\end{itemize}
