% 第6节:模型3 - 机制设计(加权百分比)

\section{模型3:机制设计}
\label{sec:model3}

我们探讨:\textbf{我们能否设计一种投票机制,在保持观众参与度的同时提高公平性?}

\subsection{区间稳健公平性分类}

点估计 $\hat{v}_i$ 继承了来自 LP 边界的不确定性。为了避免过度自信的结论,我们使用\textbf{可行区间} $[v_i^{\min}, v_i^{\max}]$ 将淘汰结果分为三类:

\begin{enumerate}[itemsep=0.2em]
    \item \textbf{确定性安全 ($\mathcal{D}_S$):} 对于\emph{所有}可行的 $v_i$,被淘汰选手的观众支持率低于中位数 \emph{或} 评委百分比低于中位数。
    
    \item \textbf{确定性错误淘汰 ($\mathcal{D}_W$):} 对于\emph{所有}可行的 $v_i$,被淘汰选手在两个维度上均高于中位数支持率:
    \[
        \mathcal{D}_W: \quad v_E^{\min} > \text{median}(\mathbf{v}) \;\land\; J_E^\% > \text{median}(\mathbf{J}^\%)
    \]
    
    \item \textbf{可能性错误淘汰 ($\mathcal{P}_W$):} 既不属于 $\mathcal{D}_S$ 也不属于 $\mathcal{D}_W$——分类结果取决于真实的(未知的)观众投票。
\end{enumerate}

\noindent\textbf{实施情况:} 对于每一次淘汰,我们依据公平性标准检查区间端点。这产生了稳健的计数:3次确定性错误淘汰 ($\mathcal{D}_W$),37次可能性错误淘汰 ($\mathcal{P}_W$),190次确定性安全 ($\mathcal{D}_S$)。

\subsection{候选机制}

\begin{enumerate}[itemsep=0.1em]
    \item \textbf{软下限:} $C_i^{\text{floor}} = \max(C_i, \theta)$ —— 保证最低得分。
    \item \textbf{加权百分比(提议):} 
\end{enumerate}
\begin{equation}
    \label{eq:weighted_pct_zh}
    C_i^{\text{new}} = \alpha \cdot J_i^{\%} + (1-\alpha) \cdot v_i, \quad J_i^{\%} = \frac{J_i}{\sum_k J_k},
\end{equation}
其中 $\alpha = 0.6$。这\emph{不是} Borda 计数法(使用排名);我们使用的是归一化的百分比。

\subsection{反事实模拟}

我们在每种机制下重演了全部 230 轮淘汰:

\begin{table}[H]
    \centering
    \caption{机制比较(区间稳健分类)}
    \label{tab:mechanism_comparison_zh}
    \begin{tabular}{lcccc}
        \toprule
        \textbf{机制} & \textbf{参数} & $|\mathcal{D}_W|$ & $|\mathcal{P}_W|$ & $|\mathcal{D}_S|$ \\
        \midrule
        当前系统 & 50/50 & 40 & -- & 190 \\
        软下限 & $\theta=5\%$ & 94 & -- & 136 \\
        \textbf{加权百分比} & $\alpha=0.6$ & \textbf{3} & 37 & 190 \\
        \bottomrule
    \end{tabular}
\end{table}

\noindent\textbf{主要发现:} 在我们的区间稳健分类下,加权百分比 ($\alpha=0.6$) 仅产生 \textbf{3个确定性错误淘汰}(低于当前系统的 40个)。根据可行边界内的真实投票值,另有 37个属于``可能性错误淘汰''。

\begin{figure}[H]
    \centering
    \includegraphics[width=0.8\textwidth]{figures/fig_task5_borda_comparison.pdf}
    \caption{不同机制下的错误淘汰分类。加权百分比 ($\alpha=0.6$) 将确定性错误淘汰数从 40 降至 3。}
    \label{fig:mechanism_comparison_zh}
\end{figure}

\subsection{$\alpha$ 的敏感性}

\begin{table}[H]
    \centering
    \caption{确定性错误淘汰数 vs. $\alpha$}
    \label{tab:alpha_dw_zh}
    \begin{tabular}{lccccc}
        \toprule
        $\alpha$ & 0.3 & 0.4 & 0.5 & \textbf{0.6} & 0.7 \\
        \midrule
        $|\mathcal{D}_W|$ & 19 & 12 & 6 & \textbf{3} & 8 \\
        \bottomrule
    \end{tabular}
\end{table}

\subsection{实施建议}

我们建议采用 $\alpha = 0.6$ 的加权百分比:(1) 将确定性错误淘汰降至 3(从 40 降下),(2) 透明的``60\%技术,40\%人气''口号,(3) 在 $\alpha \in [0.5, 0.7]$ 范围内具有稳健性。
