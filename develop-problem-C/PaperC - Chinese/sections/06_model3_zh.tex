% 第6节:机制设计(加权百分比)

\section{模型3:机制设计}
\label{sec:model3}

\noindent\textbf{\textit{本节针对任务4--5:}} 我们利用区间稳健分类法评估当前机制的公平性,随后提出并优化了一种加权百分比(Weighted Percent)机制。该机制在保留观众参与度的同时,将确认的非预期淘汰减少了92.5\%。

我们探讨:\textbf{能否设计一种既能提高公平性又能保持观众参与度的投票机制?}

\subsection{区间稳健的公平性分类}

点估计 $\hat{v}_i$ 继承了来自LP边界的不确定性。为了避免过于武断的结论,我们利用\textbf{可行区间} $[v_i^{\min}, v_i^{\max}]$ 将淘汰案例分为三类:

\begin{enumerate}[itemsep=0.2em]
    \item \textbf{确认安全 ($\mathcal{D}_S$):} 对于\emph{所有}可行的 $v_i$,被淘汰选手的观众支持率\emph{或}评委百分比均低于中位数。
    
    \item \textbf{确认的非预期淘汰 ($\mathcal{D}_W$):} 对于\emph{所有}可行的 $v_i$,被淘汰选手在两个维度上的支持率均高于中位数:
    \[
        \mathcal{D}_W: \quad v_E^{\min} > \text{median}(\mathbf{v}) \;\land\; J_E^\% > \text{median}(\mathbf{J}^\%)
    \]
    
    \item \textbf{可能的非预期淘汰 ($\mathcal{P}_W$):} 既不属于 $\mathcal{D}_S$ 也不属于 $\mathcal{D}_W$ 分类取决于未知的真实观众票数。
\end{enumerate}

\noindent\textbf{实施情况:} 我们对历史每场淘汰进行了双端检查。结果显示:40例确认的非预期淘汰($\mathcal{D}_W$),37例可能的非预期淘汰($\mathcal{P}_W$),190例确认安全案例($\mathcal{D}_S$)。

\subsection{候选机制}

\begin{enumerate}[itemsep=0.1em]
    \item \textbf{软下限制(Soft Floor):} $C_i^{\text{floor}} = \max(C_i, \theta)$  保证最低得分。
    \item \textbf{加权百分比(Weighted Percent,本文建议):} 
\end{enumerate}
\begin{equation}
    \label{eq:weighted_pct}
    C_i^{\text{new}} = \alpha \cdot J_i^{\%} + (1-\alpha) \cdot v_i, \quad J_i^{\%} = \frac{J_i}{\sum_k J_k},
\end{equation}
其中 $\alpha = 0.6$。这与Borda计数法不同(后者使用排名);我们使用的是标准化的百分比。

\subsection{公平性与参与度的权衡:帕累托分析}
\label{subsec:pareto}

一个关键问题是:\textbf{为什么选择 $\alpha = 0.6$?} 我们通过映射两个竞争目标之间的\textbf{帕累托前沿(Pareto frontier)}来回答:

\begin{itemize}[itemsep=0.1em]
    \item \textbf{公平性:} 最小化非预期淘汰($|\mathcal{D}_W|$)。
    \item \textbf{观众参与度:} 保留结果的悬念  如果评委拥有一票否决权,观众将失去投票动力。
\end{itemize}

\noindent 我们通过\textbf{观众影响力指数}来衡量参与度:即观众投票(而不仅仅是评委)决定最终淘汰结果的轮次比例。

\begin{figure}[H]
    \centering
    \includegraphics[width=0.85\textwidth]{figures/fig_pareto_frontier.jpg}
    \caption{\textbf{平衡点:机制设计的帕累托前沿。} 横轴为非预期淘汰率,纵轴为观众影响力指数。每个点代表不同的 $\alpha$ 值。\textbf{$\alpha = 0.6$ 的配置位于帕累托曲线的拐点位置}  在保留 68\% 历史观众影响力的同时,实现了非预期淘汰量 92.5\% 的降幅。}
    \label{fig:pareto}
\end{figure}

\begin{tcolorbox}[colback=blue!5!white, colframe=blue!80!black, title=\textbf{60/40 平衡点解析}]
在 $\alpha = 0.6$ 时,我们实现了:
\begin{itemize}[itemsep=0.1em]
    \item \textbf{92.5\% 的降幅}(从 40 例减少到 3 例确认的非预期淘汰)。
    \item \textbf{保留了 68\% 的观众影响力}  观众依然重要。
    \item \textbf{稳健性:} 在 $\alpha \in [0.55, 0.65]$ 范围内性能稳定。
\end{itemize}
这是\textbf{帕累托最优平衡点}  进一步增加评委权重带来的公平性收益将递减,而观众参与度却会迅速流失。
\end{tcolorbox}

\subsection{反事实模拟}

我们在各机制下重新运行了所有 230 轮淘汰:

\begin{table}[H]
    \centering
    \caption{机制比较(区间稳健分类)}
    \label{tab:mechanism_comparison}
    \begin{tabular}{lcccc}
        \toprule
        \textbf{机制} & \textbf{参数} & $|\mathcal{D}_W|$ & $|\mathcal{P}_W|$ & $|\mathcal{D}_S|$\\
        \midrule
        当前系统 & 50/50 & 40 & -- & 190 \\
        软下限制 & $\theta=5\%$ & 94 & -- & 136 \\
        \textbf{加权百分比} & $\alpha=0.6$ & \textbf{3} & 37 & 190 \\
        \bottomrule
    \end{tabular}
\end{table}

\begin{figure}[H]
    \centering
    \includegraphics[width=0.8\textwidth]{figures/fig_task5_borda_comparison.pdf}
    \caption{\textbf{确认的非预期淘汰降幅。} 不同机制的横向对比。加权百分比系统 ($\alpha=0.6$) 将确认的非预期淘汰数从 40 降至 3 —— 即 \textbf{92.5\% 的提升}。}
    \label{fig:mechanism_comparison}
\end{figure}
