% 第4节:模型1 - 观众投票反演

\section{模型1:双核反演引擎}
\label{sec:model1}

\noindent\textbf{\textit{本节针对任务1--2:}} 我们从观测到的淘汰结果逆推其背后的可行观众投票空间。

\subsection{百分比规则:基于松弛量的稳健LP}
\label{subsec:lp_core}

在第3至27季,总分 $T_i$ 由评委百分比 $J_i^\%$ 与观众百分比 $v_i$ 的等权重和决定:
\[ T_i = 0.5 \cdot J_i^\% + 0.5 \cdot v_i \quad \text{其中 } J_i^\% = \frac{J_i}{\sum_j J_j} \]

给定在某周中被淘汰的选手集合 $\mathcal{K}$(通常 $|\mathcal{K}|=1$)和晋级的选手集合 $\mathcal{U}$,必须满足以下单一淘汰条件:
\begin{equation}
    \forall e \in \mathcal{K}, \forall u \in \mathcal{U}: \quad T_u \ge T_e \implies 0.5(J_u^\% + v_u) \ge 0.5(J_e^\% + v_e)
\end{equation}

为了检测潜在的不匹配,我们引入了\textbf{投票单性形约束}下的最小松弛优化问题:
\begin{equation}
\label{eq:robust_lp}
\begin{aligned}
\min_{v, s} \quad & \sum_{e \in \mathcal{K}} s_e \\
\text{s.t.} \quad & v_u - v_e + s_e \ge J_e^\% - J_u^\%, \quad \forall e \in \mathcal{K}, u \in \mathcal{U} \\
& \sum_{i \in \mathcal{K} \cup \mathcal{U}} v_i = 1, \quad v_i \ge \epsilon
\end{aligned}
\end{equation}
其中 $s_e$ 捕捉了观测到的淘汰结果在逻辑上是不可能发生的情况(即 $s_e > 0$)。

\subsection{排名规则:基于置换的MILP}
\label{subsec:milp_core}

在第1, 2, 28--34季中,排名规则将 $T_i$ 简化为两个排名的算术和。我们使用混合整数线性规划(MILP)通过引入指示变量 $x_{ik}$(若选手 $i$ 的观众排名为 $k$ 则为1)来寻找可行的置换。

对于每一对晋级选手 $(u, e)$,约束条件如下:
\begin{equation}
    R_{J,u} + R_{v,u} \le R_{J,e} + R_{v,e}
\end{equation}
其中评委排名 $R_{J}$ 也是从 $J_i$ 计算而来的。通过遍历所有可行的观众置换,我们定义反演的\textbf{可行观众排名空间}。

\begin{figure}[H]
    \centering
    \includegraphics[width=0.85\textwidth]{figures/fig_dual_engine.jpg}
    \caption{反演引擎分析:左图展示了第14季(LP)中被淘汰选手的投票区间上限;右图展示了第32季中 MILP 捕捉到的严重不匹配(松弛量显影)。}
    \label{fig:inversion_results}
\end{figure}

\subsection{关于 S32--S33 不匹配的发现}

如图 \ref{fig:inversion_results} 右侧所示,我们的模型在 S32--S33 中一致性地检测到了正的松弛值。这表明,要么是由于未记录的评委拯救机制,要么是因为同时计算多组评分产生的交互作用,导致了单一周次的排名逻辑失效。这一发现有力地支持了《审计备忘录》中关于透明度风险的论点。
