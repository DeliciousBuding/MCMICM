% 附录 A:核心代码
\section*{附录 A:核心代码}
\addcontentsline{toc}{section}{附录 A:核心代码}
\label{appendix:code_zh}

核心算法总结如下;完整代码见补充材料。

\subsection*{A.1 LP 反演引擎(简化版)}

\begin{lstlisting}[language=Python, caption={LP 观众投票边界计算}]
def compute_bounds(judge_scores, eliminated_idx, eps=0.01):
    n = len(judge_scores)
    j_sum = np.sum(judge_scores)  # 使用求和,而非均值
    A_ub, b_ub = [], []
    for i in range(n):
        if i != eliminated_idx:
            c = np.zeros(n); c[eliminated_idx], c[i] = 1, -1
            A_ub.append(c)
            b_ub.append((judge_scores[i]-judge_scores[eliminated_idx])/j_sum)
    bounds = [(eps, 1.0) for _ in range(n)]
    results = {}
    for i in range(n):
        c_min = np.zeros(n); c_min[i] = 1
        res = linprog(c_min, A_ub, b_ub, [[1]*n], [1], bounds)
        results[i] = res.fun if res.success else None
    return results
\end{lstlisting}

\subsection*{A.2 Cox 比例风险模型}

\begin{lstlisting}[language=Python, caption={生存分析}]
from lifelines import CoxPHFitter
model = CoxPHFitter()
model.fit(df[['weeks','eliminated','judge_score','fan_vote','celeb_type']],
          duration_col='weeks', event_col='eliminated')
hazard_ratios = np.exp(model.summary['coef'])
\end{lstlisting}

\subsection*{A.3 加权百分比机制}

\begin{lstlisting}[language=Python, caption={加权百分比反事实模拟}]
def weighted_percent_score(j_scores, v_shares, alpha=0.6):
    # 不使用最大最小值:按照式(3)使用百分比归一化
    j_pct = j_scores / j_scores.sum()  # J_i / sum_k(J_k)
    return alpha * j_pct + (1-alpha) * v_shares
\end{lstlisting}
