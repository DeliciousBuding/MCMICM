% Section 4: Model 1 - Fan Vote Inversion (Dual-Core Engine)
% 模型一:粉丝投票反演(LP/CP 双核引擎)

\section{Model 1: Fan Vote Inversion via Dual-Core Engine}
\label{sec:model1}

% 4.1 问题形式化
\subsection{Problem Formulation}
\label{subsec:model1_formulation}

Let $\mathcal{C}_{s,w} = \{1, 2, \ldots, n\}$ denote the set of active contestants in season $s$, week $w$. We observe:
\begin{itemize}[itemsep=0.2em]
    \item Judge scores $\mathbf{J} = (J_1, J_2, \ldots, J_n)$
    \item Elimination outcome $E \in \mathcal{C}_{s,w}$ (the contestant sent home)
    \item For rank seasons: fan vote ranks $\mathbf{r} = (r_1, r_2, \ldots, r_n)$
\end{itemize}

\noindent We seek to infer the latent fan vote shares $\mathbf{v} = (v_1, v_2, \ldots, v_n)$ satisfying:
\begin{equation}
    \label{eq:vote_simplex}
    \sum_{i=1}^{n} v_i = 1, \quad v_i \geq \epsilon \quad \forall i,
\end{equation}
where $\epsilon > 0$ is a minimum share floor (default: 1\%).

% 4.2 百分比披露季(LP 核心)
\subsection{Percent Seasons: Linear Programming Core}
\label{subsec:lp_core}

For seasons S3--S27, where production disclosed some vote percentage information, we formulate fan vote inference as a \textbf{Linear Programming (LP) feasibility problem}.

\subsubsection{Elimination Constraint}
Under the standard DWTS rule, the contestant with the lowest combined score (judge + fan) is eliminated:
\begin{equation}
    \label{eq:combined_score}
    C_i = \frac{J_i}{\bar{J}} + v_i,
\end{equation}
where $\bar{J} = \frac{1}{n}\sum_j J_j$ normalizes judge scores. The elimination constraint becomes:
\begin{equation}
    \label{eq:elimination_constraint}
    C_E \leq C_i \quad \forall i \neq E.
\end{equation}

Substituting \eqnref{eq:combined_score} into \eqnref{eq:elimination_constraint}:
\begin{equation}
    \label{eq:lp_elimination}
    v_E - v_i \leq \frac{J_i - J_E}{\bar{J}} \quad \forall i \neq E.
\end{equation}

\subsubsection{LP Formulation}
To find tight bounds on each $v_i$, we solve:
\begin{equation}
    \label{eq:lp_bounds}
    v_i^{\min} = \min_{\mathbf{v}} v_i, \quad v_i^{\max} = \max_{\mathbf{v}} v_i
\end{equation}
subject to constraints \eqnref{eq:vote_simplex} and \eqnref{eq:lp_elimination}.

\begin{algorithm}[H]
    \caption{LP-Based Fan Vote Bound Computation}
    \label{alg:lp_bounds}
    \KwIn{Judge scores $\mathbf{J}$, eliminated contestant $E$, floor $\epsilon$}
    \KwOut{Feasible intervals $\{[v_i^{\min}, v_i^{\max}]\}_{i=1}^{n}$}
    
    Construct constraint matrix $\mathbf{A}$ from \eqnref{eq:lp_elimination}\;
    Add simplex constraint: $\mathbf{1}^\top \mathbf{v} = 1$\;
    Add floor constraints: $v_i \geq \epsilon$ for all $i$\;
    
    \For{$i \leftarrow 1$ \KwTo $n$}{
        $v_i^{\min} \leftarrow \text{LP\_solve}(\min v_i \mid \mathbf{A}\mathbf{v} \leq \mathbf{b})$\;
        $v_i^{\max} \leftarrow \text{LP\_solve}(\max v_i \mid \mathbf{A}\mathbf{v} \leq \mathbf{b})$\;
    }
    \Return $\{[v_i^{\min}, v_i^{\max}]\}$\;
\end{algorithm}

% 4.3 排名披露季(CP 核心)
\subsection{Rank Seasons: Constraint Programming Core}
\label{subsec:cp_core}

For seasons S1--S2 and S28--S34, only ordinal rankings are disclosed (e.g., ``Contestant A was 1st in fan votes, B was 2nd, ...'').

\subsubsection{Rank Ordering Constraints}
Let $r_i$ denote contestant $i$'s fan vote rank (1 = highest). The ordering constraints are:
\begin{equation}
    \label{eq:rank_ordering}
    r_i < r_j \implies v_i > v_j.
\end{equation}

\subsubsection{CP Formulation}
We use \textbf{Constraint Programming (CP)} with interval arithmetic to enumerate feasible vote distributions:

\begin{equation}
    \label{eq:cp_formulation}
    \begin{aligned}
        \text{Find:} \quad & \mathbf{v} \in [0, 1]^n \\
        \text{s.t.} \quad & \sum_i v_i = 1 \\
        & v_i \geq \epsilon \quad \forall i \\
        & v_i > v_j \quad \forall (i, j): r_i < r_j \\
        & C_E \leq C_i \quad \forall i \neq E
    \end{aligned}
\end{equation}

The CP solver propagates constraints to prune the search space, returning either a feasible assignment or proving infeasibility.

% 4.4 规则自适应约束生成
\subsection{Rule-Adaptive Constraint Generation}
\label{subsec:rule_adaptive}

DWTS production rules have evolved significantly across 34 seasons. Our engine incorporates:

\begin{table}[H]
    \centering
    \caption{Rule Changes Across DWTS Seasons}
    \label{tab:rule_changes}
    \begin{tabular}{llp{8cm}}
        \toprule
        \textbf{Seasons} & \textbf{Rule} & \textbf{Constraint Modification} \\
        \midrule
        S1--S2 & Rank-only & Use CP core instead of LP \\
        S3--S27 & Percent disclosed & Standard LP with elimination constraints \\
        S28+ & Rank-only + Judge Save & Add save veto: if $r_E \leq 2$, judges may override \\
        Various & Double elimination & Two contestants eliminated per week \\
        Various & Immunity & First-week contestants exempt from elimination \\
        \bottomrule
    \end{tabular}
\end{table}

% 4.5 异常检测:Money Plot
\subsection{Anomaly Detection via Constraint Sensitivity}
\label{subsec:anomaly_detection}

We define the \textbf{slack variable} $S^*(\epsilon)$ as the minimum constraint violation when the LP/CP system is infeasible:
\begin{equation}
    \label{eq:slack}
    S^*(\epsilon) = \min_{\mathbf{v}, s \geq 0} s \quad \text{s.t.} \quad \mathbf{A}\mathbf{v} \leq \mathbf{b} + s \cdot \mathbf{1}.
\end{equation}

\noindent\textbf{Interpretation:}
\begin{itemize}[itemsep=0.2em]
    \item $S^*(\epsilon) = 0$: Feasible---the season's eliminations are explainable by standard voting logic.
    \item $S^*(\epsilon) > 0$: Infeasible---some external factor (production intervention, voting irregularity) is required to explain outcomes.
\end{itemize}

\figref{fig:money_plot} shows $S^*(\epsilon)$ across all 34 seasons for $\epsilon \in [0\%, 10\%]$:
\begin{itemize}[itemsep=0.2em]
    \item \textbf{Normal seasons:} $S^* = 0$ throughout the range (green lines).
    \item \textbf{S32:} $S^* = 2.0$ even at $\epsilon = 0\%$---a ``hard anomaly.''
    \item \textbf{S33:} $S^* = 1.0$ at $\epsilon = 0\%$---another hard anomaly.
\end{itemize}

\begin{figure}[H]
    \centering
    \includegraphics[width=0.85\textwidth]{figures/fig_anomaly_detection.pdf}
    \caption{The ``Money Plot'': Constraint slack $S^*(\epsilon)$ vs.\ minimum vote floor $\epsilon$ for all 34 seasons. S32 and S33 exhibit positive slack even at $\epsilon = 0$, indicating infeasibility that standard elimination logic cannot explain.}
    \label{fig:money_plot}
\end{figure}

% 4.6 结果汇总
\subsection{Inversion Results Summary}
\label{subsec:model1_results}

Applying our dual-core engine to all 34 seasons yields:

\begin{itemize}[itemsep=0.3em]
    \item \textbf{Coverage:} 32 of 34 seasons (94.1\%) produce feasible fan vote distributions.
    \item \textbf{Anomalies:} S32 and S33 are flagged as hard anomalies requiring external explanation.
    \item \textbf{Bound Tightness:} Median interval width $v_i^{\max} - v_i^{\min} = 8.3\%$ across all contestant-weeks.
    \item \textbf{Point Estimates:} We use interval midpoints $\hat{v}_i = (v_i^{\min} + v_i^{\max}) / 2$ for downstream analysis.
\end{itemize}
