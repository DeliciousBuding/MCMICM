% Section 4: Model 1 - Fan Vote Inversion (Dual-Core Engine)
% 模型一:粉丝投票反演(LP/CP 双核引擎)

\section{Model 1: Fan Vote Inversion via Dual-Core Engine}
\label{sec:model1}

% 4.1 问题形式化
\subsection{Problem Formulation}
\label{subsec:model1_formulation}

Let $\mathcal{C}_{s,w} = \{1, 2, \ldots, n\}$ denote the set of active contestants in season $s$, week $w$. We observe:
\begin{itemize}[itemsep=0.2em]
    \item Judge scores $\mathbf{J} = (J_1, J_2, \ldots, J_n)$
    \item Elimination outcome $E \in \mathcal{C}_{s,w}$ (the contestant sent home)
    \item For percent seasons (S3--S27): aggregate vote share information
    \item For rank seasons (S1--S2, S28+): only judge ranks disclosed; \textbf{fan vote ranks are latent}
\end{itemize}

\noindent We seek to infer the latent fan vote shares $\mathbf{v} = (v_1, v_2, \ldots, v_n)$ satisfying:
\begin{equation}
    \label{eq:vote_simplex}
    \sum_{i=1}^{n} v_i = 1, \quad v_i \geq \epsilon \quad \forall i,
\end{equation}
where $\epsilon > 0$ is a minimum share floor (default: 1\%).

% 4.2 百分比披露季(LP 核心)
\subsection{Percent Seasons: Linear Programming Core}
\label{subsec:lp_core}

For seasons S3--S27, where production disclosed some vote percentage information, we formulate fan vote inference as a \textbf{Linear Programming (LP) feasibility problem}.

\subsubsection{Elimination Constraint}
Under the standard DWTS rule, the contestant with the lowest combined score (judge percentage + fan percentage) is eliminated. The judge percentage for contestant $i$ is:
\begin{equation}
    \label{eq:judge_pct}
    J_i^{\%} = \frac{J_i}{\sum_{k=1}^{n} J_k},
\end{equation}
where the denominator is the \textbf{sum of all active contestants' judge scores}. The combined score is:
\begin{equation}
    \label{eq:combined_score}
    C_i = J_i^{\%} + v_i = \frac{J_i}{\sum_{k} J_k} + v_i.
\end{equation}
The elimination constraint ($E$ has the lowest combined score):
\begin{equation}
    \label{eq:elimination_constraint}
    C_E \leq C_i \quad \forall i \neq E.
\end{equation}

Substituting and rearranging:
\begin{equation}
    \label{eq:lp_elimination}
    v_E - v_i \leq \frac{J_i - J_E}{\sum_{k} J_k} \quad \forall i \neq E.
\end{equation}

\subsubsection{Robust LP Formulation}
\label{subsubsec:robust_lp}

In practice, strict feasibility may fail due to measurement noise, undisclosed rule modifications, or model specification gaps. We introduce \textbf{slack variables} $\mathbf{s} = (s_1, s_2, \ldots, s_m)$ to relax the $m$ elimination constraints, formulating the \textbf{Robust LP}:

\begin{equation}
    \label{eq:robust_lp}
    \boxed{
    \begin{aligned}
        S^* = \min_{\mathbf{v}, \mathbf{s}} \quad & \sum_{k=1}^{m} |s_k| \\
        \text{s.t.} \quad & \sum_{i=1}^{n} v_i = 1 \\
        & v_i \geq \epsilon \quad \forall i \\
        & v_E - v_i \leq \frac{J_i - J_E}{\sum_{k} J_k} + s_k \quad \forall k \in \{1, \ldots, m\}
    \end{aligned}
    }
\end{equation}

\noindent where:
\begin{itemize}[itemsep=0.2em]
    \item $S^* = \sum_k |s_k|$ is the \textbf{Minimum Inconsistency Score}---the smallest total constraint violation required to achieve feasibility.
    \item $\epsilon > 0$ is the \textbf{relaxation parameter} (minimum vote floor, default 1\%).
    \item $s_k \geq 0$ allows the $k$-th elimination constraint to be violated by amount $s_k$.
\end{itemize}

\noindent\textbf{Interpretation:}
\begin{itemize}[itemsep=0.2em]
    \item $S^* = 0$: The system is \emph{feasible}---eliminations are fully explainable by fan votes and judge scores.
    \item $S^* > 0$: The system exhibits \emph{model-data mismatch}---some factor not captured by our model is required to explain the observed outcomes. This could be undocumented rule variations, judge save subjectivity (S28+), or data aggregation differences.
\end{itemize}

\subsubsection{Bounding via LP}
Given a feasible (or minimally relaxed) constraint set, we compute tight bounds on each $v_i$:
\begin{equation}
    \label{eq:lp_bounds}
    v_i^{\min} = \min_{\mathbf{v}} v_i, \quad v_i^{\max} = \max_{\mathbf{v}} v_i
\end{equation}
subject to constraints \eqnref{eq:vote_simplex}, \eqnref{eq:lp_elimination}, and $\sum |s_k| \leq S^* + \delta$.

% 4.3 排名披露季(CP 核心)
\subsection{Rank Seasons: Mixed-Integer Linear Programming}
\label{subsec:milp_core}

For seasons S1--S2 and S28--S34, only judge rankings are disclosed; \textbf{fan vote ranks are latent variables}. We model fan ranks as \textbf{decision variables} using MILP.

\subsubsection{MILP Formulation}
Let $x_{ik} \in \{0,1\}$ indicate whether contestant $i$ has fan rank $k$. The constraints:
\begin{align}
    \sum_{k=1}^{n} x_{ik} &= 1 \quad \forall i \quad \text{(each contestant has exactly one rank)} \label{eq:one_rank}\\
    \sum_{i=1}^{n} x_{ik} &= 1 \quad \forall k \quad \text{(each rank assigned to exactly one contestant)} \label{eq:rank_unique}\\
    r_i^{\text{fan}} &= \sum_{k=1}^{n} k \cdot x_{ik} \quad \text{(define fan rank)} \label{eq:fan_rank_def}
\end{align}

The \textbf{elimination constraint} under rank rules (combined rank $R_i = r_i^{\text{judge}} + r_i^{\text{fan}}$):
\begin{equation}
    \label{eq:rank_elim}
    R_E \geq R_i \quad \forall i \neq E \quad \text{(eliminated has highest combined rank)}
\end{equation}

\noindent We model contestant eligibility via an FSM with states $\{\texttt{ACTIVE}, \texttt{IMMUNE}, \texttt{SAVED}, \texttt{ELIMINATED}\}$.

% 4.4 规则自适应约束生成
\subsection{Rule-Adaptive Constraints}

\begin{table}[H]
    \centering
    \caption{Rule Changes Across DWTS Seasons}
    \label{tab:rule_changes}
    \begin{tabular}{llp{7cm}}
        \toprule
        \textbf{Seasons} & \textbf{Rule} & \textbf{Constraint Modification} \\
        \midrule
        S1--S2 & Rank-only & Use CP core \\
        S3--S27 & Percent & Standard LP \\
        S28+ & Rank + Judge Save & Add save veto \\
        Various & Double elim/Immunity & Adjust active set \\
        \bottomrule
    \end{tabular}
\end{table}

% 4.5 异常检测:Money Plot
\subsection{Anomaly Detection}
\label{subsec:anomaly_detection}

Using the Robust LP (\eqnref{eq:robust_lp}), we compute $S^*(\epsilon)$---the minimum inconsistency as a function of floor $\epsilon$:
\begin{itemize}[itemsep=0.1em]
    \item $S^*(\epsilon) = 0$: \textbf{Feasible}---eliminations are explainable.
    \item $S^*(\epsilon) > 0$: \textbf{Infeasible}---external factor required.
\end{itemize}

\figref{fig:money_plot} shows $S^*(\epsilon)$ across all 34 seasons: normal seasons remain at $S^* = 0$, while \textbf{S32} ($S^* = 2.0$) and \textbf{S33} ($S^* = 1.0$) are ``hard anomalies.''

\begin{figure}[H]
    \centering
    \includegraphics[width=0.8\textwidth]{figures/fig_anomaly_detection.pdf}
    \caption{The ``Money Plot'': S32 and S33 exhibit positive slack at $\epsilon = 0$, indicating structural infeasibility.}
    \label{fig:money_plot}
\end{figure}

\begin{figure}[H]
    \centering
    \includegraphics[width=0.85\textwidth]{figures/fig1_inconsistency_spectrum.pdf}
    \caption{Inconsistency Spectrum: structural breaks in S32--S33.}
    \label{fig:inconsistency}
\end{figure}

\noindent\textbf{Key Finding:} Seasons 1--31 achieve $S^* \approx 0$ (model success); S32--33 spike to $S^* > 1$ (structural break). This is not model failure but \textbf{system failure}---the Judges' Save mechanic created mathematically unexplainable outcomes.

% 4.6 结果汇总
\subsection{Summary}

\begin{itemize}[itemsep=0.2em]
    \item \textbf{Coverage:} 32 of 34 seasons (94.1\%) produce feasible fan vote distributions.
    \item \textbf{Anomalies:} S32 and S33 are flagged as hard anomalies requiring external explanation.
    \item \textbf{Bound Tightness:} Median interval width $v_i^{\max} - v_i^{\min} = 8.3\%$ across all contestant-weeks.
    \item \textbf{Point Estimates:} We use interval midpoints $\hat{v}_i = (v_i^{\min} + v_i^{\max}) / 2$ for downstream analysis.
\end{itemize}


