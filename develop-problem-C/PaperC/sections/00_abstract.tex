% Abstract / Summary Sheet
% 摘要页(占第0页,不计入正文)

\begin{abstract}

\emph{Dancing with the Stars} (DWTS) combines judge scores with viewer votes to determine weekly eliminations, yet individual fan vote shares remain unpublished. We develop a \textbf{Dual-Core Inversion Engine} to reconstruct latent fan vote distributions from elimination outcomes across all 34 US seasons.

For \emph{percent-disclosed} seasons (S3--S27), we formulate fan vote inference as a \textbf{Linear Programming (LP)} problem with slack variables, deriving tight bounds on each contestant's vote share. For \emph{rank-only} seasons (S1--S2, S28--S34), we employ \textbf{Constraint Programming (CP)} with interval arithmetic. Both solvers incorporate rule changes (immunity, double eliminations, judge save) from official archives.

Our framework identifies \textbf{two anomalous seasons}---S32 and S33---where no feasible fan vote assignment exists even under relaxed constraints ($\epsilon \to 0$). The ``Money Plot'' visualizes constraint slack $S^*(\epsilon)$: normal seasons remain feasible throughout $\epsilon \in [0\%, 10\%]$, while S32/S33 exhibit positive slack at $\epsilon = 0\%$, suggesting production interventions.

Building on inferred votes, \textbf{Cox survival analysis} reveals: (1) professional partner explains 42.9\% of survival variance; (2) judge scores dominate early rounds while fan momentum drives late-stage outcomes.

We propose a \textbf{Weighted Borda Count} mechanism ($0.6 \times \text{Judge} + 0.4 \times \text{Fan}$) achieving \textbf{92.5\% reduction} in wrongful eliminations. A \textbf{Safety Net Protocol} provides real-time anomaly detection for producers.

\begin{keywords}
inverse problems; linear programming; constraint satisfaction; survival analysis; mechanism design; anomaly detection
\end{keywords}

\end{abstract}
