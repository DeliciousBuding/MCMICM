% Abstract / Summary Sheet
% 摘要页(占第0页,不计入正文)

\begin{abstract}

\textbf{Motivated by recurring judge--fan disagreements (e.g., Season 27), we treat \emph{Dancing with the Stars} (DWTS) as an audit problem:} can the show's published scoring rules explain the observed eliminations \textbf{without ever observing individual fan vote shares}?

To answer this, we build a \textbf{Dual-Core Inversion Engine} that (i) reconstructs \emph{feasible fan-vote intervals} under percent-rule seasons (S3--S27) via \textbf{Linear Programming}, and (ii) infers \emph{rank-only fan permutations} under rank-rule seasons (S1--S2, S28--S34) via \textbf{Mixed-Integer Linear Programming}. We then quantify \textbf{model--data mismatch} using a minimal-slack metric $S^*$, identifying when no feasible vote assignment can reproduce the observed outcome.

Across 34 seasons, our inversion engine explains \textbf{32 seasons with zero slack} under a 1\% tolerance; only \textbf{S32--S33} exhibit structural mismatch (no feasible assignment exists under stated rules). Possible causes include undocumented rule variations, unmodeled judge-save criteria, or multi-dance aggregation differences.

Building on inferred votes, \textbf{Cox survival analysis} reveals: (1) professional partner is the dominant survival predictor (likelihood ratio test $\chi^2 = 47.3$, $p < 0.001$; Harrell's C-index improvement $\Delta C = 0.09$); (2) judge scores dominate early rounds while fan momentum drives late-stage outcomes.

For mechanism design, our \textbf{Weighted Percent} recommendation ($0.6 \times \text{JudgePct} + 0.4 \times \text{FanPct}$) reduces \textbf{Definite-Wrongful eliminations from 40 to 3} ($-92.5\%$) while preserving fan influence near the Pareto knee.

\begin{keywords}
inverse problems; linear programming; mixed-integer programming; survival analysis; mechanism design
\end{keywords}

\end{abstract}
