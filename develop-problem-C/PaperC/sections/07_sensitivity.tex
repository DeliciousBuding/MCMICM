% Section 7: Sensitivity and Robustness Analysis
% 敏感性与鲁棒性分析

\section{Sensitivity and Robustness Analysis}
\label{sec:sensitivity}

We conduct systematic sensitivity analysis to validate our models' robustness across parameter choices and data perturbations.

% 7.1 最小投票份额敏感性
\subsection{Minimum Vote Share Floor ($\epsilon$)}
\label{subsec:epsilon_sensitivity}

The floor parameter $\epsilon$ controls the minimum vote share constraint in Model 1. We vary $\epsilon$ from 0\% to 10\% and examine:

\begin{itemize}[itemsep=0.2em]
    \item \textbf{Feasibility rate:} Proportion of seasons with feasible vote distributions.
    \item \textbf{Bound tightness:} Average interval width $v_i^{\max} - v_i^{\min}$.
    \item \textbf{Anomaly detection:} Which seasons become infeasible at each $\epsilon$.
\end{itemize}

% \begin{figure}[H]
%     \centering
%     \includegraphics[width=0.85\textwidth]{figures/fig_epsilon_sensitivity.pdf}
%     \caption{Sensitivity analysis for minimum vote floor $\epsilon$. Left: Feasibility rate decreases as $\epsilon$ increases. Right: Bound tightness improves (narrower intervals) with higher $\epsilon$ among feasible cases.}
%     \label{fig:epsilon_sensitivity}
% \end{figure}

\noindent\textbf{Key Finding:} At $\epsilon = 1\%$ (our default), 94.1\% of seasons are feasible with median bound width 8.3\%. Increasing to $\epsilon = 5\%$ narrows bounds to 5.1\% but reduces feasibility to 82.4\%.

% 7.2 机制权重敏感性
\subsection{Mechanism Weight ($\alpha$) Sensitivity}
\label{subsec:alpha_sensitivity}

For the Weighted Percent mechanism, we vary $\alpha$ (judge weight) from 0.3 to 0.8:

\begin{table}[H]
    \centering
    \caption{Wrongful Eliminations vs.\ Judge Weight $\alpha$}
    \label{tab:alpha_sensitivity}
    \begin{tabular}{cccc}
        \toprule
        $\alpha$ & Wrongful Elim. & Rate & Change vs.\ Current \\
        \midrule
        0.30 & 12 & 5.2\% & -70.0\% \\
        0.40 & 7 & 3.0\% & -82.5\% \\
        0.50 & 5 & 2.2\% & -87.5\% \\
        0.55 & 4 & 1.7\% & -90.0\% \\
        \textbf{0.60} & \textbf{3} & \textbf{1.3\%} & \textbf{-92.5\%} \\
        0.65 & 4 & 1.7\% & -90.0\% \\
        0.70 & 8 & 3.5\% & -80.0\% \\
        0.80 & 18 & 7.8\% & -55.0\% \\
        \bottomrule
    \end{tabular}
\end{table}

\noindent\textbf{Key Finding:} The optimal range is $\alpha \in [0.55, 0.65]$, with $\alpha = 0.6$ achieving the global minimum. The mechanism is robust to small perturbations in $\alpha$.

% 7.3 Bootstrap 稳定性
\subsection{Bootstrap Stability Analysis}
\label{subsec:bootstrap_stability}

To assess estimation stability, we perform 1,000 bootstrap resamples at the season level:

\begin{itemize}[itemsep=0.2em]
    \item \textbf{Fan vote bounds:} 95\% CI for median bound width is $[7.6\%, 9.1\%]$.
    \item \textbf{Professional dancer effect:} 95\% CI for variance explained is $[38.2\%, 47.6\%]$.
    \item \textbf{Weighted Percent wrongful rate:} 95\% CI is $[0.4\%, 2.6\%]$ (robust improvement over current system).
\end{itemize}

% 7.4 Leave-One-Season-Out 交叉验证
\subsection{Leave-One-Season-Out Validation}
\label{subsec:loso}

We validate Model 2 (survival analysis) using leave-one-season-out cross-validation:

\begin{enumerate}
    \item Train Cox model on 33 seasons.
    \item Predict survival curves for held-out season.
    \item Compute concordance index (C-index) on held-out data.
\end{enumerate}

\begin{table}[H]
    \centering
    \caption{Leave-One-Season-Out Validation Results}
    \label{tab:loso}
    \begin{tabular}{lc}
        \toprule
        \textbf{Metric} & \textbf{Value} \\
        \midrule
        Mean C-index & 0.72 \\
        Std C-index & 0.08 \\
        Worst season (S32) & 0.54 \\
        Best season (S15) & 0.86 \\
        \bottomrule
    \end{tabular}
\end{table}

\noindent\textbf{Key Finding:} The model achieves a mean C-index of 0.72, indicating good discriminative ability. Notably, the anomalous season S32 has the lowest C-index (0.54), confirming that our anomaly detection correctly identifies seasons where standard models fail.

% 7.5 数据扰动鲁棒性
\subsection{Data Perturbation Robustness}
\label{subsec:perturbation}

We add Gaussian noise to judge scores ($\sigma = 1$ point) and re-run the inversion:

\begin{itemize}[itemsep=0.2em]
    \item \textbf{Bound stability:} Median absolute change in bounds is 1.2 percentage points.
    \item \textbf{Ranking consistency:} 94.3\% of contestant rankings are preserved.
    \item \textbf{Anomaly persistence:} S32 and S33 remain flagged as anomalies in all perturbation trials.
\end{itemize}
