% Appendix A: Key Code
% 附录A:核心代码

\section*{Appendix A: Key Code}
\addcontentsline{toc}{section}{Appendix A: Key Code}
\label{appendix:code}

This appendix presents the core implementation of our analytical framework. Full code is available in our supplementary materials.

% A.1 数据预处理
\subsection*{A.1 Data Loading and Preprocessing (ETL Module)}

\begin{lstlisting}[language=Python, caption={DWTS data loader with FSM-based cleaning}]
import pandas as pd
import numpy as np
from typing import Dict, List, Tuple

class DWTSDataLoader:
    """Load and clean DWTS competition data using FSM rules."""
    
    def __init__(self, filepath: str):
        self.raw_df = pd.read_csv(filepath)
        self.cleaned_df = None
        
    def clean_data(self) -> pd.DataFrame:
        """Apply FSM-based data cleaning."""
        df = self.raw_df.copy()
        
        # Standardize column names
        df.columns = df.columns.str.lower().str.replace(' ', '_')
        
        # Parse season number
        df['season_num'] = df['season'].str.extract(r'(\d+)').astype(int)
        
        # Determine disclosure type
        df['disclosure_type'] = np.where(
            df['season_num'].between(3, 27), 'percent', 'rank'
        )
        
        # Parse judge scores (handle various formats)
        df['judge_total'] = df['judges_scores'].apply(self._parse_judge_score)
        
        self.cleaned_df = df
        return df
    
    def _parse_judge_score(self, score_str: str) -> float:
        """Parse judge scores from various string formats."""
        if pd.isna(score_str):
            return np.nan
        # Handle "8+7+8+9" or "32/40" formats
        if '+' in str(score_str):
            return sum(map(float, str(score_str).split('+')))
        elif '/' in str(score_str):
            return float(str(score_str).split('/')[0])
        return float(score_str)
\end{lstlisting}

% A.2 LP 核心引擎
\subsection*{A.2 Linear Programming Inversion Engine}

\begin{lstlisting}[language=Python, caption={LP-based fan vote bound computation}]
from scipy.optimize import linprog
import numpy as np

class LPInversionEngine:
    """Infer fan vote bounds via Linear Programming."""
    
    def __init__(self, epsilon: float = 0.01):
        self.epsilon = epsilon  # Minimum vote floor
        
    def compute_bounds(self, judge_scores: np.ndarray, 
                       eliminated_idx: int) -> Dict:
        """Compute min/max bounds for each contestant's vote share."""
        n = len(judge_scores)
        j_bar = np.mean(judge_scores)
        
        # Build elimination constraints: v_E - v_i <= (J_i - J_E) / J_bar
        A_ub = []
        b_ub = []
        for i in range(n):
            if i != eliminated_idx:
                constraint = np.zeros(n)
                constraint[eliminated_idx] = 1
                constraint[i] = -1
                A_ub.append(constraint)
                b_ub.append((judge_scores[i] - judge_scores[eliminated_idx]) / j_bar)
        
        # Equality constraint: sum(v) = 1
        A_eq = np.ones((1, n))
        b_eq = np.array([1.0])
        
        # Bounds: v_i >= epsilon
        bounds = [(self.epsilon, 1.0) for _ in range(n)]
        
        # Compute bounds for each contestant
        results = {'bounds': [], 'feasible': True}
        for i in range(n):
            c_min = np.zeros(n); c_min[i] = 1
            c_max = np.zeros(n); c_max[i] = -1
            
            res_min = linprog(c_min, A_ub, b_ub, A_eq, b_eq, bounds)
            res_max = linprog(c_max, A_ub, b_ub, A_eq, b_eq, bounds)
            
            if res_min.success and res_max.success:
                results['bounds'].append((res_min.fun, -res_max.fun))
            else:
                results['feasible'] = False
                results['bounds'].append((None, None))
        
        return results
\end{lstlisting}

% A.3 生存分析
\subsection*{A.3 Cox Proportional Hazards Analysis}

\begin{lstlisting}[language=Python, caption={Survival analysis with lifelines}]
from lifelines import CoxPHFitter
import pandas as pd

class SurvivalAnalyzer:
    """Cox PH model for elimination risk analysis."""
    
    def __init__(self):
        self.model = CoxPHFitter()
        
    def fit(self, df: pd.DataFrame, 
            duration_col: str = 'weeks_survived',
            event_col: str = 'eliminated') -> None:
        """Fit Cox model to DWTS survival data."""
        covariates = ['avg_judge_score', 'avg_fan_vote', 
                     'celebrity_type', 'prior_fame', 'pro_dancer_id']
        
        # One-hot encode categoricals
        df_model = pd.get_dummies(df[covariates + [duration_col, event_col]], 
                                   drop_first=True)
        
        self.model.fit(df_model, duration_col=duration_col, 
                       event_col=event_col)
        
    def get_hazard_ratios(self) -> pd.DataFrame:
        """Extract hazard ratios with confidence intervals."""
        summary = self.model.summary
        summary['HR'] = np.exp(summary['coef'])
        summary['HR_lower'] = np.exp(summary['coef'] - 1.96 * summary['se(coef)'])
        summary['HR_upper'] = np.exp(summary['coef'] + 1.96 * summary['se(coef)'])
        return summary[['HR', 'HR_lower', 'HR_upper', 'p']]
\end{lstlisting}

% A.4 机制设计
\subsection*{A.4 Weighted Borda Count Mechanism}

\begin{lstlisting}[language=Python, caption={Borda mechanism with counterfactual simulation}]
import numpy as np

class BordaMechanism:
    """Weighted Borda Count voting mechanism."""
    
    def __init__(self, alpha: float = 0.6):
        self.alpha = alpha  # Judge weight
        
    def compute_scores(self, judge_scores: np.ndarray, 
                       fan_votes: np.ndarray) -> np.ndarray:
        """Compute normalized Borda scores."""
        # Min-max normalize
        j_norm = (judge_scores - judge_scores.min()) / \
                 (judge_scores.max() - judge_scores.min() + 1e-8)
        v_norm = (fan_votes - fan_votes.min()) / \
                 (fan_votes.max() - fan_votes.min() + 1e-8)
        
        return self.alpha * j_norm + (1 - self.alpha) * v_norm
    
    def simulate_elimination(self, judge_scores: np.ndarray,
                            fan_votes: np.ndarray) -> int:
        """Return index of contestant to eliminate (lowest score)."""
        scores = self.compute_scores(judge_scores, fan_votes)
        return np.argmin(scores)
    
    def counterfactual_analysis(self, historical_data: List[Dict],
                                current_system_elim: List[int]) -> Dict:
        """Compare wrongful eliminations: current vs Borda."""
        wrongful_current = 0
        wrongful_borda = 0
        
        for week_data, actual_elim in zip(historical_data, current_system_elim):
            borda_elim = self.simulate_elimination(
                week_data['judge'], week_data['fan'])
            
            # Check if actual elimination was "wrongful"
            fan_rank_actual = np.argsort(week_data['fan'])[::-1].tolist()
            if actual_elim != borda_elim:
                if fan_rank_actual.index(actual_elim) < \
                   fan_rank_actual.index(borda_elim):
                    wrongful_current += 1
        
        return {
            'wrongful_current': wrongful_current,
            'wrongful_borda': wrongful_borda,
            'reduction': 1 - wrongful_borda / max(wrongful_current, 1)
        }
\end{lstlisting}

% A.5 异常检测
\subsection*{A.5 Anomaly Detection via Constraint Slack}

\begin{lstlisting}[language=Python, caption={Money Plot: constraint slack analysis}]
def compute_slack(judge_scores, eliminated_idx, epsilon):
    """Compute minimum constraint violation (slack) at given epsilon."""
    n = len(judge_scores)
    j_bar = np.mean(judge_scores)
    
    # Add slack variable to make LP always feasible
    # Minimize: s (slack)
    # Subject to: original constraints + s * 1 >= violation
    
    c = np.zeros(n + 1)
    c[-1] = 1  # Minimize slack
    
    A_ub = []
    b_ub = []
    for i in range(n):
        if i != eliminated_idx:
            constraint = np.zeros(n + 1)
            constraint[eliminated_idx] = 1
            constraint[i] = -1
            constraint[-1] = -1  # Slack relaxation
            A_ub.append(constraint)
            b_ub.append((judge_scores[i] - judge_scores[eliminated_idx]) / j_bar)
    
    bounds = [(epsilon, 1.0) for _ in range(n)] + [(0, None)]
    A_eq = np.zeros((1, n + 1))
    A_eq[0, :n] = 1
    b_eq = np.array([1.0])
    
    res = linprog(c, A_ub, b_ub, A_eq, b_eq, bounds)
    return res.x[-1] if res.success else float('inf')

def generate_money_plot(all_seasons_data, epsilon_range):
    """Generate slack vs epsilon plot for all seasons."""
    results = {}
    for season, data in all_seasons_data.items():
        slacks = []
        for eps in epsilon_range:
            slack = compute_slack(data['judge'], data['elim_idx'], eps)
            slacks.append(slack)
        results[season] = slacks
    return results
\end{lstlisting}
